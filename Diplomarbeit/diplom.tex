% Diplomarbeit deutsch Matthias Kreier
% weitere Informationen unter http://people.physik.hu-berlin.de/~kreier/
% Originalversion 24.01.2008
% Recreated for arXiv 31.12.2025

\documentclass[11pt,twoside,german,openany]{book}
%\nofiles % verhindert Ausgabe einer neuen TOC
\usepackage{palatino}
\usepackage[OT2,T1]{fontenc}
% \usepackage{ucs}                % - Modern LaTeX (2020+) already assumes UTF‑8 by default, so you don’t need ucs or utf8x.
% \usepackage[utf8x]{inputenc}
\usepackage{geometry}
\geometry{verbose,a4paper,tmargin=35mm,bmargin=25mm,lmargin=25mm,rmargin=25mm,headheight=10mm,headsep=10mm,footskip=10mm}
\pagestyle{headings}
\setcounter{secnumdepth}{3}
\setcounter{tocdepth}{3}
\usepackage{graphicx}
\usepackage{amsmath}
\usepackage{amssymb}
\usepackage{changepage}
\usepackage[russian,ngerman]{babel}
\newcommand\ru[1]{\foreignlanguage{russian}{#1}}  % um russische Texte einzubinden
%\usepackage{ae}   % Die Umlaute werden dann wieder unschön, das ß geht verloren ... (in Palatino)
\usepackage[pdftex,pdfpagelabels=true]{hyperref}
\usepackage{booktabs}
%\usepackage[pdftex]{thumbpdf}

\makeatletter
  \newcommand{\noun}[1]{\textsc{#1}}

  % \DeclareRobustCommand*\textsubscript[1]{%
  %   \@textsubscript{\selectfont#1}}
  % \newcommand{\@textsubscript}[1]{%
  %   {\m@th\ensuremath{_{\mbox{\fontsize\sf@size\z@#1}}}}}

  \renewcommand{\chaptermark}[1]{%
  \markboth{#1}{}}
  \renewcommand{\sectionmark}[1]{%
  \markright{#1}{}}

% \bibliographystyle{Nplain} % Stil der Phys. Rev. Ausgaben ist amsplain

\hypersetup{
 colorlinks=true,
 linkcolor=blue,
 citecolor=blue,
 urlcolor=blue,
 pdftitle	    = {Electronic Properties of CdHgTe},
 pdfsubject	  = {Diplomarbeit Institut für Physik},
 pdfauthor	  = {Matthias Kreier},
 pdfkeywords	= {ARPES,CdHgTe},
 pdfcreator	  = {Adobe-Acrobat-Distiller},
 pdfproducer	= {LaTeX with hyperref and thumbpdf}
}

\newenvironment{mytable}[1]{
  \begin{table}[h]
  \centering
  \begin{tabular}{#1}
  \toprule\toprule
}{
  \bottomrule\bottomrule
  \end{tabular}
  \end{table}
}

\makeatother

\begin{document}
\frontmatter
\pagenumbering{Roman}
\begin{titlepage}
\begin{center}
\vspace*{2mm}
{\Huge
Elektronische Struktur der\\
ternären II/VI-Halbleiter\\[1.2mm]
Cd\textsubscript{\large1-X}Hg\textsubscript{\large X}Te}\\[13mm]
{\Large Diplomarbeit}\\[20mm]
\includegraphics[width=40mm, height=40mm]{pic/husiegel_bw_rgb}\\[20mm]
\noun{
Humboldt-Universität zu Berlin\\
%Mathematisch-Naturwissenschaftliche Fakultät I\\
Institut für Physik\\
AG Elektronische Eigenschaften und Supraleitung}\\[15mm]
eingereicht von\\[10mm]
{
Matthias Kreier\\
geb. am 3.10.1975 in Nauen}\\[25mm]
Berlin, 24. 1. 2008
\end{center}
\end{titlepage}

% \newpage
% \thispagestyle{empty}
% \mbox{}
% \newpage

\setcounter{page}{0}

\newpage
\thispagestyle{empty}
\mbox{}
\newpage

\tableofcontents{}

\mainmatter

% ****************************************************************** Kapitel 1: Einleitung *******************************************************************

\chapter{Einleitung}

Aufgrund militärischer Anforderungen war man in den 40er und 50er Jahren des 20. Jahrhunderts intensiv auf der Suche nach einem direkten intrinsischen Halbleiter für den langwelligen Infrarot-Wellenlängenbereich (LWIR, 8-14 µm). Ein Ergebnis dieser Bemühungen war die Synthese der ternären Legierung HgCdTe im Jahre 1958 von der Lawson Forschungsgruppe \cite{lawson} am Royal Radar Establishment in England. Die Bedeutung dieser Arbeit wurde schon früh erkannt. Das führte zu einer intensiven Entwicklung in zahlreichen Ländern wie England, Frankreich, Polen, Deutschland, der Sowjetunion sowie den USA [2]. Doch wurde über die Entwicklung in den ersten Jahren wenig veröffentlicht. Die Arbeiten in den USA zum Beispiel unterlagen der Geheimhaltung bis in die späten 60er Jahre. Die ersten Photodetektoren wurden in den USA bereits 1964 von Texas Instruments hergestellt.

\begin{figure}
\begin{center}
\includegraphics[width=70mm, keepaspectratio]{pic/1-sensor}
\caption[width=90mm]{\label{Rockwell}Abbildung 1.1: HgCdTe in Infrarotkameras: Das Rockwell 2x2 2Kx2K Infrarot-Array Hawaii-2RG}
% \captionsetup{width=85mm}
\end{center}
\end{figure}

Verschiedene Eigenschaften machen HgCdTe zum idealen Material als IR Detektor. Einige von diesen sind eine einstellbare Bandlücke von 0,7 to 25 µm, die direkte Bandlücke mit hohem Absorptionskoeffizienten, moderate Dielektrizitätskonstante und Brechungsindex sowie einen geringen thermischen Ausdehnungskoeffizienten. Weiterhin gibt es eine Auswahl passender Substrate für epitaktisches Wachstum über einen großen Wellenlängenbereich (z. B. $Cd_{0.96}Zn_{0.04}Te$)

Aufgrund seiner Eigenschaften ist HgCdTe im Bereich der IR-Detektoren das Material der Wahl [3], [4]. In Abbildung 1.1 ist das 2x2 2Kx2K Infrarotarray Hawaii-2RG (16 Megapixel) zu sehen. Es wurde für das 6,5 m James Webb Space Telescope (JWST), den Nachfolger des Hubble-Teleskopes, entwickelt. Es wird unter anderem im Very Large Telescope (VLT) der ESO im Experiment SINFONI verwendet [5]. Das Material wird schon seit langem bei Weltraummissionen für die Infrarotastronomie eingesetzt. Zu erwähnen wäre hier das Experiment NICMOS (Near Infrared Camera and Multi-Object Spectrometer), welches aus Kameras und Spektrometern für das nahe Infrarot bis 2,5 µm Wellenlänge am Hubble-Weltraumteleskop besteht und seit 1997 im Einsatz ist.

Seine besonderen Eigenschaften erhält die ternäre Verbindung HgCdTe durch eine Mischung der Eigenschaften seiner beiden Bestandteile CdTe und HgTe. Beide Materialien lassen sich in beliebigen Verhältnissen mischen. Das Verhältnis von Cd zu Hg wird mit 0 < x < 1 in Hg1-XCdXTe angegeben. Bei CdTe handelt es sich um einen klassischen Halbleiter mit einer direkten Bandlücke von 1,6 eV. Die Legierung HgTe hingegen ist nicht einfach einzuordnen. Sie besitzt metallische Eigenschaften, weshalb einige sie als Semimetall bezeichnen. Andererseits zeigt sie Halbleitereigenschaften, die allerdings nur richtig erklärt werden können, wenn man von einer negativen Bandlücke ausgeht. Gemäß der Definition der Bandlücke als Energiedifferenz zwischen $\Gamma_8$ und $\Gamma_6$ beträgt die Bandlücke daher -0.283 eV. Dies ist in Übereinstimmung mit magnetooptischen Transportmessungen [6].

Ziel dieser Diplomarbeit ist die Untersuchung der elektronischen Struktur von CdxHg1-xTe mit x-Werten von 0,07 bis 0,4. Uns interessiert, wie sich die elektronische Bandstruktur beim Übergang vom Halbleiter zum ’zero-gap’-Halbleiter zum Halbleiter mit negativer Bandlücke verhält. Als verheißungsvoll erwies sich die Arbeit von N. Orlowski [7] an HgTe, in der das unter das VBM geklappte $\Gamma_6$-Band spektroskopisch aufgelöst werden konnte. Seine Ergebnisse von 2000 motivierte zweifellos zur Ausschreibung der folgenden Diplomarbeit im Jahre 2004:

\begin{quote}
Elektronische Eigenschaften der II-VI-Halbleiterverbindungen Cd\textsubscript{1-x}Hg\textsubscript{x}Te und Pb\textsubscript{1-x}Sn\textsubscript{x}Te

Die in der Arbeit zu untersuchenden ternären Halbleiterverbindungen erlauben in Abhängigkeit von der Zusammensetzung eine weite Variation der fundamentalen Bandlücke, die nicht nur zu Null sondern sogar auch negativ gemacht werden kann (wird bei Nachfrage näher erläutert). Dabei wird das prinzipielle Auftreten einer negativen Bandlücke bei verschiedenen Halbleitern heute noch kontrovers diskutiert und eine direkte Untersuchungsmethode wie die winkelaufgelöste Photoemission könnte hier die Klärung bringen.

Die Aufgabe der/des Diplomandin/en besteht u.a. darin, die in einem kooperierenden Mos-
kauer Institut hergestellten Einkristalle zu charakterisieren (z.B. SEM/Röntgenemission, Laue-
Beugung, LEED) und anschließend mit winkelaufgelöster Photoemission und evtl. auch inverser
Photoemission die experimentelle Bandstruktur zu bestimmen und die Ergebnisse mit Rechnun-
gen verschiedener Modelle zu vergleichen und zu diskutieren. Die Messungen sollen sowohl im
Berliner Institut als auch teilweise mit Synchrotronstrahlung (BESSY in Berlin oder HASYLAB in
Hamburg) durchgeführt werden.
\end{quote}


% **************************************************** Kapitel 2: Eigenschaften von II-VI Halbleitern *******************************************************

\chapter{Eigenschaften von II-VI Halbleitern}
\section{Kristallstruktur}

Die Struktur von kristallinen Festkörpern wird durch das Gitter und die Basis beschrieben. Das Gitter ist eine dreidimensionale Anordnung von Punkten, deren kleinste Einheit die Elementarzelle ist. Es wird durch die entsprechenden Gitterkonstanten sowie primitive Vektoren beschrieben. Die Basis definiert die Anordnung der Atome in der Elementarzelle. Zu jeder Struktur gibt es ein entsprechendes reziprokes Gitter im reziproken Raum. In diesem wird die elektronische Struktur beschrieben. Die Wigner-Seitz-Zelle des reziproken Gitters heißt erste Brillouin-Zone.

\begin{figure}
  \centering
  \includegraphics[width=80mm, keepaspectratio]{pic/zinkblende}
  \caption[width=110mm]{ Zinkblendestruktur von HgCdTe. Die gelben Punkte repräsentieren das Tellur. An den grauen Plätzen befindet sich Cadmium mit einem Anteil von „x“ oder Quecksilber mit einem Anteil „1-x“.}
  \label{fig:Zinkblende}
\end{figure}

Die räumliche Struktur von HgCdTe wird als „Zinkblende-Struktur“ bezeichnet (Abbildung~\ref{fig:Zinkblende}). Diese Struktur wird durch zwei kubisch-flächenzentrierte (fcc) Elementarzellen gebildet, die um ein Viertel ihrer Raumdiagonalen gegeneinander verschoben sind. Hier bildet ein Tellur-Atom die Basis des ersten Gitters. Die Basis des zweiten Gitters wird zu x aus Cadmiumatomen gebildet und zu $1 - x$ aus Atomen von Quecksilber.

\begin{table}[h]
\centering
\begin{tabular}{|c|c|c|c|c|}
d    & $d_f$ / $\pi / a$   & d(HgTe) / \AA & d(CdTe) / \AA & d(X) / \AA \\
\midrule
$|\Gamma \Lambda L|$ & $\sqrt{3}/2$ & 1,478 & 1,485 & 1,4815 \\
$|\Gamma \Delta X|$ & $1/2$ & 1,204 & 1,209 & 1,2065 \\
$|\Gamma \Sigma K|$ & $\sqrt{2}/2$ & 1,703 & 1,711 & 1,707 \\
$|\Gamma W|$ & $\sqrt{11}/4$ & 1,482 & 1,489 & 1,4855 \\
$|\Gamma U|$ & $\sqrt{2}/4$ & 0,852 & 0,856 & 0,854 \\
$|\Gamma K X|$ & $\sqrt{3}/4$ & 1,043 & 1,048 & 1,0455 \\
\end{tabular}
\caption{Entfernungen der hochsymmetrischen Punkte vom Zentrum der Volumen-Brillouin-Zone des fcc-Gitters. Die formalen Werte für df sind angegeben, um aus einer beliebigen Gitterkonstante $a$ die Entfernungen berechnen zu können.}
\label{tab:brillouin}
\end{table}


% Tabelle 2.1: Entfernungen der hochsymmetrischen Punkte vom Zentrum der Volumen-Brillouin-Zone des fcc-Gitters. Die formalen Werte für df sind angegeben, um aus einer beliebigen Gitterkonstante $a$ die Entfernungen berechnen zu können.

% |$\Gamma$ΛL|
% |$\Gamma$∆X|
% |$\Gamma$ΣK|
% |$\Gamma$W|
% |$\Gamma$U|
% |$\Gamma$K X |

In dieser Kristallstruktur ist jedes Atom tetraedrisch von vier nächsten Nachbarn der anderen Atomsorte umgeben. Die Ursache ist die Hybridisierung der s- und p-Orbitale der Valenzelektronen zu sp3-Hybridorbitalen. Sie schließen einen Bindungswinkel von 109,5◦ ein.

Die Gitterkonstanten von HgTe und CdTe unterscheiden sich um weniger als 0.7\% voneinander. Für die ternären Verbindung Hg1-xCdxTe folgt sie entsprechend dem Kompositionsparameter $x$ der empirischen Formel $a = 6, 4614 + 0, 00084x + 0, 0168x2 - 0, 0057x3$ [8] .

\begin{table}[h]
\centering
\begin{tabular}{|c|c}
\hline\hline
Substanz & $a$ / \AA \\ \hline
HgTe & 6,445 [9] \\ 
Hg\textsubscript{0.8}Cd\textsubscript{0.2}Te & 6,464 [10] \\ 
CdTe & 6,488 [9] \\ 
\end{tabular}
\end{table}

Das Kristallgitter wird durch folgende primitive Vektoren beschrieben:

$$\vec{a_1} = \frac{a}{2} \begin{pmatrix} 0 \\ 1 \\ 1 \end{pmatrix}, \quad
\vec{a_2} = \frac{a}{2} \begin{pmatrix} 1 \\ 0 \\ 1 \end{pmatrix}, \quad
\vec{a_3} = \frac{a}{2} \begin{pmatrix} 1 \\ 1 \\ 0 \end{pmatrix}$$

Gemäß Definition ergeben sich daraus die zugehörigen reziproken Gittervektoren:

$$\vec{b_1} = \frac{2\pi}{a} \begin{pmatrix} -1 \\ 1 \\ 1 \end{pmatrix}, \quad
\vec{b_2} = \frac{2\pi}{a} \begin{pmatrix} 1 \\ -1 \\ 1 \end{pmatrix}, \quad
\vec{b_3} = \frac{2\pi}{a} \begin{pmatrix} 1 \\ 1 \\ -1 \end{pmatrix}$$

\section{Volumen-Brillouin-Zone}

Aus diesen reziproken Gittervektoren resultiert ein reziprokes Gitter, das kubisch raumzentriert (bcc, body centered cubic) ist. Die erste Brillouin-Zone ist das in Abbildung 2.2 dargestellte abgestumpfte Oktaeder. Die Abbildung zeigt einen Schnitt durch die Volumen-Brillouin-Zone sowie die Oberflächen-Brillouin-Zonen, die den Flächen (110) und (001) entspechen.

Ebenfalls eingezeichnet sind die hochsymmetrischen Punkte $\mathrm{\Gamma}$, X, L, K, W und U. Es finden sich auch die drei hochsymmetrischen Richtungen $\Delta$, $\Sigma$ und $\Lambda$. Sie entsprechen den Richtungen [001],[110] und [111]. Die jeweiligen Abstände können in Tabelle 2.1 abgelesen werden.



Abbildung 2.2: Volumen-Brillouin-Zone des fcc-Gitters sowie Oberflächen-Brillouin-
Zonen der idealen (001) und (110) Oberflächen. Einige hochsymmetrische Punkte
sind eingezeichnet. Die Richtungen $\Delta$, $\Sigma$ und $\Lambda$ entsprechen jeweils der Richtung
[001], [110] und [111].

\section{Oberflächeneigenschaften}

An der Oberfläche kommt es zu einem Bruch der Translationssymmetrie des Volumenkristalls. Man kann daher die Oberfläche als eine Störung auffassen; das Kristall nimmt einen Zustand minimaler Energie an. In der idealen Oberfläche besetzen die Oberflächenatome eine wohldefinierte Gitterebene des Volumenkristalls, ihre Periodizität wäre somit festgelegt. Dieser Zustand ist jedoch äußerst selten der Fall.

Die einfachste Abweichung von der idealen Oberfläche wird als Relaxation bezeichnet. Hierbei treten einheitliche Verschiebung der obersten oder der oberen Lagen gegenüber dem Volumen auf. Von einer Rekonstruktion spricht man, wenn die Atome der obersten Lage periodisch gegeneinander verschoben sind und eine Überstruktur bilden [11].

Die natürliche Spaltfläche der II-VI-Halbleiter (wie auch HgCdTe) ist die (110)-Fläche. Durch ein Spalten der Probe im Vakuum lässt sich diese Oberfläche am einfachsten generieren. Alle Messungen dieser Arbeit wurden daher an dieser Oberfläche durchgeführt. Andere Oberflächen sind weitaus schwieriger herzustellen, wie zum Beispiel die (001)-Oberfläche [7].

Abbildung 2.3: Geometrie der (110)-Oberfläche der Zinkblende-Struktur [12] im Querschnitt und in
der Draufsicht. Die Parameter der Oberfläche sind ax = a

Die (110)-Oberfläche enthält ebenso viele Anionen wie Kationen und ist daher energetisch begünstigt. Es handelt sich um eine unpolare Oberfläche. Bei der Spaltung relaxiert die (110)-Oberfläche. Die Anionen entfernen sich von der Oberfläche, während sich die Kationen auf das Kristall zu bewegen. Für diese Oberfläche wurde ein universales Modell entwickelt, das unabhängig von der Substanz gültig ist [13]. Hierbei bleiben die Bindungslängen nahezu erhalten, die Symmetrie parallel zur Oberfläche wird nicht geändert.

An der Oberfläche wird die Anion-Kation-Bindung aufgebrochen. Bei der Relaxation kommt es zu einem Ladungstransfer vom Kation-’dangling bond’ zum Anion-’dangling-bond’. Das
’dangling-bond’ am Kation wird vollständig geleert und am Anion entsprechend vollständig besetzt. Jedes Oberflächenatom besitzt nur noch drei der ursprünglich vier Nachbarn. So hat jedes Anion (Kation) der Oberfläche zwei Bindungen zu einem Oberflächen-Kation (Anion), eine zu einem Kation (Anion) im Volumen gerichtete Bindung sowie eine ins Vakuum gerichtete freie Valenz (’dangling bond’). Es kommt daher zu einer Dehybridisierung der sp\textsuperscript{3}-Orbitale. Die Anionen nehmen eine s\textsuperscript{2}p\textsuperscript{3} Koordination an, die Kationen erhalten eine planare sp\textsubscript{2} Koordination. Das vom Anion-’dangling-bond’-abgeleitete Band senkt sich energetisch ab. Es verschiebt sich zum oberen Rand des Volumen-Valenzbandes.


% ***************************************************** Kapitel 3: Theoretische Grundlagen der Photoemission **************************************************

\chapter{Theoretische Grundlagen der Photoemission}


Um geeignete Materialien in Bauelementen wie Solarzellen, Detektoren oder integrierten Schaltkreisen verwenden zu können, muss ihre elektronische und strukturelle Charakteristik bekannt sein. Dazu gehören ebenfalls die Transporteigenschaften des Materials. Die Bestimmung der elektronischen Bandstruktur ist hierzu eines der wichtigsten Hilfsmittel.

Eine leistungsfähige Methode, die direkte Zustandsdichte und die impulsaufgelöste Energiebandstruktur zu bestimmen, ist die Photoelektronspektroskopie. Im Rahmen dieser Diplomarbeit findet insbesondere die winkelaufgelöste Photoelektronspektroskopie oder ARPES\footnote{ARPES - \underline{A}ngle \underline{R}esoved \underline{P}hoto \underline{E}lectron \underline{S}pectroscopy} Anwendung. Im Folgenden werden ihre physikalischen Grundlagen erläutert.


\section{Messprinzip der Photoelektronspektroskopie}

Die Photoelektronspektroskopie basiert auf dem so genannten äußeren Photoeffekt. Dieses physikalische Phänomen wurde bereits 1887 von H. Hertz [14] und 1888 von W. Hallwachs [15] entdeckt und untersucht. Die richtige Deutung erfolgte durch A. Einstein im Jahre 1905 ([16], Nobelpreis 1921). Der Effekt wird durch die folgende Formel beschrieben:

$$
E^{max}_{kin} = h\nu - \Phi.
$$

Sie gibt die maximale kinetische Energie $E^{max}_{kin}$ an, mit der Elektronen bei Anregung mit Strahlung der Energie $h\nu$ aus einem Metall austreten. Hier ist h das Plancksche Wirkungsquantum, $\nu$ die Frequenz des ionisierenden Photons und $\Phi$ die Austrittsarbeit des angeregten Materials.

Abhängig von der Energie der anregenden Photonen spricht man in der Photoelektronenspektroskopie von UPS\footnote{UPS - Ultraviolet Photoemission Spectroscopy, UV-Photoemission} oder XPS\footnote{XPS - X-Ray Photoemission Spectroscopy, Röntgenphotoemission}, wobei UPS Photonenenergien im UV-Bereich (10 bis 100
eV) bezeichnet und XPS Photonen im Röntgenbereich (> 1000 eV). Auf Grund ihrer höheren Energie werden durch XPS auch Rumpfelektronen angeregt. In Abhängigkeit der chemischen Umgebung zeigen XPS-Spektren Unterschiede in den Bindungsenergien eines Rumpfelektrons. In vielen Fällen kann aus der Form der Spektren Aufschluss über den Valenzzustand eines Elementes gegeben werden. Diese chemische Analyse ist unter der Bezeichnung ECSA (Electron Spectroscopy for Chemical Analysis) bekannt.

Demgegenüber werden in UPS nur Valenzelektronen und Elektronen aus hochgelegten Rumpfniveaus angeregt. UPS eignet sich daher zur Untersuchung der Valenzbandstruktur von
Halbleitern. In ARPES werden die emittierten Elektronen winkel- und energieaufgelöst detektiert. Das Prinzip wird in Abbildung 3.1 verdeutlicht.

\section{Das Drei-Stufen-Modell}

Es gibt eine Vielzahl erfolgreicher theoretischer Beschreibungen des Photoemissionsprozesses. In der Praxis hat sich das anschauliche Drei-Stufen-Modell [17] durchgesetzt. Im Einteilchenbild gliedert sich der Prozess damit in die folgenden drei unabhängigen Schritte [18]:

\begin{itemize}
  \item Absorption des Photons und Anregung eines Elektrons aus einem Anfangszustand im Valenzband in einen Endzustand im Leitungsband
  \item Transport des angeregten Elektrons zur Oberfläche des Festkörpers
  \item Austritt des Elektrons durch die Oberfläche ins Vakuum
\end{itemize}

Die einzelnen Schritte dieses Modells werden im Folgenden einzeln beschrieben.

\subsection*{Schritt 1: Anregung des Elektrons}

Der erste Schritt beschreibt die Photoionisation. Lokal wird ein Photon absorbiert und ein Elektron angeregt. Dieser Prozess lässt sich mit der zeitabhängigen Störungstheorie erklären [19].

Die Übergangsrate $T_{f →i}$ für ein Elektron von einem Anfangszustand $\left| \Phi_i \right\rangle$ mit einer Anfangsenergie $E_i$ in einen Endzustand $\left| \Phi_f \right\rangle$ mit der Energie $E_f$ ist durch Fermis Goldene Regel gegeben:

$$T_{f →i} = \frac{2\pi}{\hbar} \left| \left\langle \Phi_f \right| H_{int} \left| \Phi_i \right\rangle \right|^2 \delta(E_f - E_i - h\nu)$$

Hierbei bezeichnet $\hbar \omega$ die Energie des Photons. Die Wechselwirkung zwischen Elektron und Photon wird durch den Hamiltonoperator $H_{WW}$ beschrieben. In Coulomb-Eichung und linearer Näherung lautet dieser

$$H_{WW} = \frac{e}{2mc} \vec{A} \cdot \vec{p}$$

Das Vektorpotential der einfallenden elektromagnetischen Strahlung ist durch (cid:126)A gegeben
und beinhaltet Eigenschaften wie Frequenz, Phase und Polarisation. (cid:126)p ist der quantenmecha-
nische Impulsoperator (cid:126)p = i(cid:126). Die Energieerhaltung fordert, dass nur Übergänge vorkom-
men, die der Relation Ef = Ei + (cid:126) genügen. Diese Bedingung wird in Formel 3.2 durch die
-Funktion berücksichtigt.

Es seien die Wellenvektoren des Anfangs- und Endzustandes mit (cid:126)ki und (cid:126)kf gegeben. Der
Photonenimpuls kann bei den geringen Anregungsenergien von ARPES gegenüber dem Elek-
tronenimpuls vernachlässigt werden. Wir erhalten daher wellenvektorerhaltende direkte Über-
gänge. Die Impulserhaltung lautet somit:

$$\vec{k_i} = \vec{k_f} + \vec{k}$$

Die Übergangswahrscheinlichkeit wird im wesentlichen durch das Betragsquadrat des Über-
gangsmatrixelementes in Gleichung 3.2 bestimmt. Dieses hängt sowohl vom Anfangszustand
|i(cid:105) als auch vom Endzustand |f (cid:105) ab. In der Photoemission wird daher eine Kombination aus
beiden Zustandsdichten gemessen.

\subsection*{Schritt 2: Transport des angeregten Elektrons zur Oberfläche}

Die Photonen dringen mehrere 100 Å in den Festkörper ein und regen Elektronen an. Beim
Transport zur Oberfläche verlieren einige Elektronen durch inelastische Stöße kinetische Ener-
gie. Dabei geht die Information über den Anfangszustand verloren. Die mittlere freie Weglänge
der Elektronen in Abhängigkeit von ihrer kinetischen Energie ist in Abbildung 3.3 gezeigt. Die-
se „universelle Kurve“ ergibt sich aus Messungen, die an verschiedenen Materialien durchge-
führt wurden. Die eingetragenen Messpunkte verdeutlichen, dass die mittlere freie Weglänge
weitgehend unabhängig vom Material ist.

Aus der Abbildung ist ersichtlich, dass die mittlere freie Weglänge bei den in UPS übli-
chen Anregungsenergien nur einige Angström beträgt. Damit ist der Transport zur Oberfläche
der limitierende Schritt der PES und begründet seine Oberflächensensitivität. Messungen der
Photoemission repräsentieren nur die obersten Atomschichten. Sie erfordern ein gutes Vaku-
um, um eine Bedeckung der Oberfläche mit Fremdatomen zu verhindern. Weiterhin werden
besondere Anforderungen gestellt, um zuvor eine geeignete Oberfläche zu erhalten.

Abbildung 3.2: Photoemissionsprozess und zugehörige Energieverhältnisse für Probe und Analysator.
Die einzelnen Energien werden im Text erläutert.

Abbildung 3.3: Energieabhängige mittlere freie Weglänge
von Elektronen im Festkörper [20].

Abbildung 3.4: Brechung der Elektro-
nen an der Kristalloberfläche. (cid:126)K Wellen-
vektor im Vakuum, (cid:126)kf Wellenvektor des
Endzustandes im Kristall

Der dominierende Stoßprozess ist die Elektron-Elektron Streuung. Durch diesen Prozess
wird ein Spektrum von niederenergetischen Sekundärelektronen generiert, die später im Pho-
toemissionsspektrum sichtbar sind (siehe Abbildung 3.2). Die Elektron-Phonon Wechselwir-
kung hat nur bei sehr geringen Energien eine Bedeutung [18].

\subsection*{Schritt 3: Durchtritt des Elektrons vom Festkörper ins Vakuum}

Der dritte Schritt ist mit einer Brechung verbunden. Wir können das angeregte Elektron inner-
halb des Kristalls als quasi-frei betrachten. Die Energie-Impuls-Beziehung für den Endzustand
(cid:126)kf vor dem Durchtritt durch die Oberfläche lautet daher

$$E_f = \frac{\hbar^2 k_f^2}{2m} - |E_0|$$

Der Wellenvektor (cid:126)kf zerlegt sich in seine Anteile (cid:126)kf (cid:107) parallel zur Probenoberfläche und (cid:126)kf 
senkrecht zur Oberfläche (siehe Abbildung 3.4). Aufgrund der Translationsinvarianz bleibt die
parallele Komponente beim Durchtritt bis auf einen reziproken Gittervektor oder Oberflächen-
gittervektor (cid:126)g(cid:107) erhalten:

$$\vec{k_{||}} = \vec{k_f{||}} + \vec{g}$$

Die Addition eines solchen reziproken Gittervektors bezeichnet man als Umklappprozess.
Betrachten wir die Emission unter einem Winkel $\vartheta$ zur Probennormalen und vernachlässigen
Umklappprozesse (d. h. (cid:126)g(cid:107) = 0), so erhalten wir für die Parallelkomponente des Wellenvektors:

$$k_{||} = k_f \sin \vartheta = \frac{1}{\hbar} \sqrt{2m(E_{kin} + |E_0|)} \sin \vartheta$$

Die senkrechte Komponente bleibt nicht erhalten, da das Kristallpotential V0 außerhalb des
Festkörpers nicht vorhanden ist. Dies führt zu einer Verkürzung der senkrechten Komponente.
Dennoch können wir über sie eine Aussage machen, wenn wir als Endzustände freie Elektro-
nenparabeln annehmen. Die kinetische Energie ist dann wie folgt gegeben:

$$E_{kin} = \frac{\hbar^2 (k_{||}^2 + k_{\perp}^2)}{2m} - |E_0|$$

Die reziproken Volumengittervektoren (cid:126)G repräsentieren die jeweiligen Elektronenparabeln.
Der Wellenvektor (cid:126)kf lässt sich in seine Komponenten (cid:126)kf = (cid:126)kf (cid:107) + (cid:126)$k_{f\perp}$ zerlegen. Das Ergebnis
von Gleichung (3.6) kann in diesen Ansatz eingesetzt werden. In senkrechter Emission ($\vartheta = 0$)
erhalten wir:

$$k_{\perp} = \frac{1}{\hbar} \sqrt{2m(E_{kin} + |E_0|)}$$

Hier wurde der Volumengittervektor (cid:126)G in seine Komponenten zerlegt. Jedoch kann (cid:126)G(cid:107) auch
ein reziproker Oberflächengittervektor sein. Aufgrund der Impulserhaltung (3.4) kann so aus
(3.6) und (3.8) die gemessene Zustandsdichte einem bestimmten Punkt in der Brillouin-Zone
zugeordnet werden. Durch ein systematisches Abrastern des k-Raumes kann auf diese Weise
die Bandstruktur E((cid:126)k, (cid:126)k(cid:107)) = E((cid:126)k) bestimmt werden. In der Praxis erfolgt dies entlang hoch-
symmetrischer Richtungen.

Ei
Ef
EV BM Valenzbandmaximum
Evac,S Vakuumenergie der Probe
Evac,A Vakuumenergie des Analysators
(cid:126)$\omega$
Ub
Ethr
Ekin,S kinetische Energie der Elektrons

Energie des Photons
Bindungsenergie
Photoemissionsschwellwert

bezüglich der Probe (Sample)

Ekin,A kinetische Energie des Elektrons
bezüglich des Analysators
Kontaktpotential
Austrittsarbeit der Probe
Austrittsarbeit des Analysators
Fermi-Energie

Abbildung 3.5: Energieniveauschema für die Photoemission an Probe und Analysator

Aus der kinetischen Energie der emittierten Elektronen lassen sich Informationen über die
Bindungsenergien im Kristall erhalten. Ist die Energie Uthr des Photoemissionsschwellwertes
bekannt, so kann die Bindungsenergie bezüglich des Valenzbandmaximums durch folgende
Formel bestimmt werden:

$$E_B = h\nu - (E_{kin,S} + U_{thr})$$

Hier muss zwischen der kinetischen Energie bezüglich der Probe Ekin,S und bezüglich des
Analysators Ekin,A unterschieden werden. Die relativen Energieverhältnisse sind in Abbildung
3.5 noch einmal gesondert dargestellt. Der Unterschied zwischen den beiden kinetischen Ener-
gien entspricht dem Kontaktpotential UK. Es ergibt sich aus der Differenz der Austrittsarbeiten
von Probe und Analysator

$$E_{kin,S} = E_{kin,A} - U_K$$

In den Gleichungen (3.6), (3.8) und (3.9) bezeichnet Ekin die kinetische Energie bezüglich
der Probe. Letzten Endes wird die kinetische Energie jedoch im Analysator bestimmt. Die maxi-
mal gemessene kinetische Energie hängt damit nur von der Energie der anregenden Photonen
und der Austrittsarbeit des Analysators ab - siehe auch die eingangs erwähnte Formel (3.1). Im
Falle eines untersuchten Metalls entspricht diese der Fermi-Energie. Somit ist die gemessene
energetische Lage der Fermi-Kante unabhängig von der Austrittsarbeit der Probe.

Dieser spezielle Umstand des elektrostatischen Elektronenanalysators wird z. B. in der Un-
tersuchung von HTSCs\footnote{HTSC - \underline{H}igh \underline{T}emperature \underline{S}upra\underline{C}onductor, Hochtemperatursupraleiter oder High-T\textsubscript{C}-Supraleiter
} genutzt. Es werden die Spektren einer Probe aufgenommen, die von
besonderem Interesse sind. Anschließend wird die Probe an einer speziellen Stelle der Vaku-
umkammer mit amorphen Gold bedampft. Danach wird die Fermi-Kante der so präparierten
Oberfläche gemessen. Dies dient als Referenzpunkt für Bindungsenergien. Im Rahmen dieser
Diplomarbeit wurden die gemessenen Spektren ebenfalls auf die Fermi-Energie bezogen, deren
Lage mit einer amorphen Goldprobe bestimmt wurde.

Außerdem ist es möglich, auf diese Weise die Austrittsarbeit der untersuchten Probe selbst
zu bestimmen. Die bestimmte Fermi-Energie ist bei jeder Messung an derselben Stelle, weil
das gesamte Spektrum beim Eintritt in den Analysator verschoben wird. Der Betrag der Ver-
schiebung ist durch die Kontaktspannung gegeben (Abbildung 3.5). Die meisten Analysatoren
sind innen mit Graphit beschichtet, welches eine geringe Austrittsarbeit von 4,14 eV besitzt.
Die minimale kinetische Energie der Sekundärelektronen beim Austritt aus der Probe beträgt
0 eV. Diese Kante wird ebenfalls um die Kontaktspannung UK verschoben. Aus der mit dem
Analysator bestimmten niederenergetischen Grenze der Sekundärelektronen kann damit die
Austrittsarbeit der Probe bestimmt werden.

\section{Auswertung der Messdaten und Spektren}
\subsection{Konstanter Untergrund}

Die zur Detektion der Elektronen eingesetzten elektronischen Bauelemente (u. a. Channeltron,
Operationsverstärker) besitzen ein charakteristisches Eigenrauschen. Dieses Rauschen verur-
sacht einen energieunabhängigen konstanten Untergrund IB. Betrachten wir die Intensität Ii
an einem Messpunkt oder Kanal i. Dieser Wert Ii kann der Intensität bei einer bestimmten
Energie I(E) zugeordnet werden. Diese Intensität sollte bei kinetischen Energien oberhalb der
Fermi-Energie Null betragen. Damit lässt sich der konstante Untergrund IB bestimmen und
von den Messdaten abziehen [21]:

$$I^{'}_{i,corr} = I_i - I_B$$

In den letzten Jahrzehnten sind große Fortschritte in der Technologie der Detektoren sowie
ihrer Elektronik gemacht worden. Oftmals besitzen sie eine Einstellmöglichkeit für die Rausch-
unterdrückung [19]. So kann dieser Anpassungsschritt heute meist entfallen.

\subsection{Subtraktion des inelastischen Untergrundes}

Insbesondere bei Anregung durch die He-Gasentladungslampe ist das gemessene Spektrum
von einem deutlichen inelastischen Untergrund überlagert (siehe Abbildung 3.6 im Vergleich
zu Abbildung 7.4). Ein wichtiger Unterschied ist die Größe des Fokus, in dem die Anregung
erfolgt. An der BUS5-Beamline liegt der Durchmesser des bestrahlten Fleckes im Bereich von
100 µm. Das ermöglicht bei empfindlichen Proben ein „Abrastern“ der Oberfläche [23]. Dem-
gegenüber regen die Photonen der He-Lampe HIS 13 trotz Fokussierspiegel einen Bereich von
ca. 1mm Durchmesser an [19].

Abbildung 3.6: Valenzbandspektrum von HgCdTe mit inelastischem Unter-
grund, berechnetem Untergrundsignal und dem Spektrum ohne Untergrund.


Wir betrachten wieder die Intensität I (cid:48)

i an einem Messpunkt i. Zur Korrektur der inelasti-
schen Streuung wird von der Intensität I (cid:48)
i ein Betrag abgezogen, der proportional zur integrier-
ten Intensität des Valenzbandspektrums bei höheren Energien E > E(i) ist. Die korrigierte
Intensität I (cid:48)(cid:48)

i ergibt sich daher wie folgt:

$$I_{i,corr}^{''} = I_i - \kappa \int_{E(i)}^{E_{max}} I(E') dE'$$

Die Intensität $I_0^{'}$  in Kanal 0 steht für eine Energie unterhalb des Valenzbandes und wurde mit $I^{''}_0 = I^{'}_{i_{max}}$ festgelegt.

\subsection{Glättung der Spektren nach Savitzky-Golay}

Wie jedes andere analoge Signal sind die gemessenen Intensitäten mit einem statistischen Feh-
ler behaftet, der sich in einem Rauschen des Messsignals äußert. Dieses lässt sich selbstver-
ständlich durch eine Erhöhung der Statistik verringern. Damit ist aber auch ein Anstieg der
nötigen Messzeit verbunden; der möglichen Zählrate sind Grenzen gesetzt. In der Praxis muss
ein Kompromiss zwischen akzeptabler Messzeit und hoher Zählrate gefunden werden.

Die Spektren der Photoemission werden meist mit sehr hoher Auflösung aufgenommen, so
dass die einzelnen Werte von Datenpunkt zu Datenpunkt nur wenig variieren. Diese sind mit
einem Rauschen überlagert. Für diese Art Daten eignet sich der Glättungsalgorithmus nach
Savitzky-Golay [24] sehr gut. Hierbei ergibt sich der geglättete Wert I (cid:48)
i eines Messpunktes aus
der gewichteten Mittelung über seine Nachbarwerte:

$$I_{i,glatt} = \sum_{n=-m}^{m} C_n I_{i+n}$$

Die Mittelung im Intervall k $\in$ [i - nL, i + nR] soll gerade dem Wert des Least-Square-
Polynomfits durch die Punkte Ik am Punkt Ii entsprechen, die Koeffizienten cn werden ent-
sprechend gewählt. Zur Auswertung der Spektren wurde die Software Origin 7.5 verwendet
[25]. Sie bietet einen eingebauten Algorithmus zur Datenglättung nach Savitzky-Golay. Er fand
bei den ausgewerteten Spektren Anwendung.

\section{Trennung von Oberflächen und Volumenbandstruktur}

Im Allgemeinen interessiert bei Photoemissionsmessungen die elektronische Struktur des Vo-
lumens. Bedingt durch die starke Oberflächensensitivität dieser Messmethode spielen ener-
getische Zustände der Oberfläche in den gemessenen Spektren eine große Rolle. Die Wellen-
funktion dieser Zustände fällt auf beiden Seiten der Oberfläche exponentiell ab. Im Fall einer
Entartung mit Volumenzuständen fällt sie im Kristall nur auf einen endlichen Wert ab, man
spricht von einer Oberflächenresonanz.

Der Ursprung dieser abweichenden Zustände liegt in der abweichenden Bindungsstruktur
der Oberfläche. Dazu zählen sogenannte ’dangling bonds’ - freie Valenzen - sowie Brückenbin-
dungen (’bridge bonds’) und Rückbindungen (’back bonds’). Oberflächenabgeleitete Zustände
besitzen einige Eigenschaften, die sie von Volumenzuständen unterscheiden. In der folgenden
Liste sind einige aufgeführt [7]. Dabei ermöglicht ein einzelner Punkt allein keine eindeutige
Zuordnung [11]. Vielmehr sollten immer mehrere dieser Kriterien erfüllt sein:

\begin{itemize}
  \item Die energetische Lage von Oberflächenzuständen ist unabhängig von der Photonenenergie. Bei einer Messung in normaler Emission (auch: $k\perp$-Messung, $k\parallel = 0$) zeigen sie keine Dispersion.
  \item Ihre Periodizität ist an die Oberflächen-Brillouin-Zone gekoppelt. Sie zeigen daher ein anderes Dispersionsverhalten bezüglch $k\parallel$.
  \item Oberflächenzustände liegen teilweise in einer Energielücke des Volumens. Oberflächenresonanzen fallen dagegen mit Volumenzuständen zusammen.
  \item Oberflächenzustände reagieren empfindlich auf Adsorbate. Sie verschwinden vielfach bei Adsorption von weniger als einer Monolage.
  \item Volumenzustände können häufig durch den Vergleich mit Volumenbandstrukturrechnungen identifiziert werden.
  \item Wenn ebenso Bandstrukturrechnungen für die Oberfläche vorliegen, kann ein Vergleich weitere Hinweise auf Oberflächenzustände liefern.
  \item Teilweise besitzen Oberflächenzustände Orbitalcharakter. Durch gezielte Variation des Übergangsmatrixelementes oder der Symmetrieeigenschaften in Abhängigkeit vom Azimutalwinkel kann dieser Charakter untersucht werden.
\end{itemize}




% ***************************************************** Kapitel 4: Experimentelles **************************************************

\chapter{Experimentelles}

Alle Photoemissionsmessungen wurden mit der hochauflösenden AR65-Anlage durchgeführt.
Sie ist transportabel gestaltet, daher konnten Messungen sowohl im Labor mit einer He-Lampe
als auch bei BESSY an einem Synchrotronmessplatz durchgeführt werden.

\section{Die Photoemissionsanlage AR65}

Die UHV-Anlage AR65 wurde in unserer Arbeitsgruppe bis 1999 aufgebaut [26], um hochauf-
lösende Messungen am neu gebauten Synchrotron BESSY II zu ermöglichen [27]. Die 750kg
schwere Anlage wurde kompakt und fahrbar aufgebaut, damit an verschiedenen Beamlines
oder auch im Labor gemessen werden kann. Sie besteht aus einer Hauptkammer, in der die
eigentlichen Messungen durchgeführt werden, und einer Einschleuskammer.

In Kapitel 2 wurde die hohe Oberflächensensitivität der Photoemission erläutert. Weite-
re Aspekte HgCdTe betreffend werden in Kapitel 6 erörtert. Um diesen hohen Anforderungen
gerecht zu werden, ist die Hauptkammer mit einer Vielzahl unterschiedlicher Pumpen verbun-
den. Diese laufen an der Pumpkammer zusammen, dort findet auch die Druckmessung statt.
Die Pumpkammer ist über ein 200mm-Pumprohr von 1 Meter Länge mit der Messkammer ver-
bunden. Auf diese Weise sollen störende Einflüsse der Pumpsysteme und des Druckmesskop-
fes auf die Photoemissionsmessung minimiert werden. Insbesondere magnetische Streufelder
der Ionengetterpumpe könnten das Messergebnis beeinflussen.

An die Pumpkammer sind eine 500l-Turbomolekularpumpe, eine 500l-Ionengetterpumpe,
eine 1200l-Titansublimationspumpe sowie eine 1500l-Kryopumpe angeschlossen. Die angege-
benen Pumpleistungen pro Sekunde beziehen sich mit Ausnahme der TSP (hier H2) auf Stick-
stoff. Mit ihnen kann am Pumpenkreuz ein Druck von $p = 1.3 · 10^{-10}mbar$ erreicht werden
(2007), ohne Kryopumpe werden noch $p = 3 · 10^{-10}mbar$ erreicht. Der entsprechende Druck in
der Hauptkammer liegt um den Faktor 2 höher [26]. Die Druckmessung der IGP unterschreitet
im ersteren Fall ihren Grenzbereich von $6.4 · 10^{-11}mbar$ und zeigt low pressure... an.

Um störende Magnetfelder (wie auch das Erdmagnetfeld) vom Messort fernzuhalten, ist
die Messkammer innen mit einer Abschirmung aus µ-Metall versehen. Dadurch können äußere
Magnetfelder auf 0,5 \% reduziert werden.

Die winkelaufgelöste Untersuchung der emittierten Photonen erfolgt mit dem Photoelektronenanalysator AR65 der Firma Omicron. Er besitzt einen 180◦ Kugelkondensator oder SDA1 mit einem Sollbahnradius von 65 Millimetern und ist damit der Namengeber der gesamten Anlage. Dieser ist auf einem 2-Achsen-Goniometer montiert. Damit können Winkel im Bereich von $10^\circ < \Phi <-90^\circ$ in der Vertikalen und 0◦< $\theta < 350^\circ$ in der Horizontalen angefahren werden.

Allgemein kann die spektrale Auflösung $\Delta E_A$ eines Kugelkondensators mithilfe der folgenden Formel [28] bestimmt werden:

$$\Delta E_A = \frac{E_P \Delta R}{2R_0} + E_P \frac{\Delta \alpha^2}{4} + \Delta E_{det}$$

Sie hängt für eine Passenergie $E_{pass}$ der eintretenden Elektronen unter einem Öffnungswinkel $\alpha$ vom Radius $r_0$ des Kondensators sowie der Breite s des Eingangs- und Ausgangsspaltes
ab. Für die AR65 sind die Spaltbreiten mit s = 1mm gegeben, die Winkelakzeptanz beträgt
$\alpha = \pm 1◦$. Für eine standardmäßig verwendete Passenergie von $E_{pass} = 10eV$ ergibt sich damit
eine Energieauflösung von $\Delta E_A \approx 80meV$.

\section{Heliumlampe Focus HIS 13}

\section{Automatische N\textsubscript{2} Nachfüllanlage}

\section{BUS-Beamline bei BESSY}

\section{Röntgenuntersuchung mittels Laue}



\begin{center}
\begin{tabular}{|c|c|c|c|}
\hline 
photon energy&
kink at {[}eV{]}&
vec&
transmission\tabularnewline
\hline
\hline 
22.12&
3.546&
12.1&
14.3\tabularnewline
\hline 
22.12&
6.543&
12.4&
35.3\tabularnewline
\hline
\end{tabular}
\end{center}

The slight shift in intensity is connected to the thin bilayer splitting in HTSC based on YBCO. Bi2212 dosn't shows this behavior.


\newpage

\begin{figure}
\begin{center}
\includegraphics[width=100mm, keepaspectratio]{pic/fermikante}
\caption{\label{fermikante}Aufbau der Meßkammer}
\end{center}
\end{figure}

1,5 mGauß nachgewiesen(vgl. Erdmagnetfeld: max 500 mGauß). Wichtig für die Experimente ist die Unterteilung der Kammer in mehrere Ebenen: Transferebene, Zusatzebene, Meßebene und Pumpebene (von oben nach unten, siehe auch Abb. 3.2 und 3.1) In der Transferebene können die Proben eingeschleust, manipuliert und orientiert werden. Der Probentransfer zwischen Probenschleuse und Meßkammer wird mit einer sog. Transferstange realisiert, die mit einem speziellen Gabelkopf zur Aufnahme der Probenhalter ausgerüstet wurde (zu den Probenhaltern siehe Kap. 5.5) Der Kryostat, der von oben auf die Kammer aufgesetzt ist, nimmt die Proben auf. In die Transferebene wurde ein Vakuum-Manipulierarm (engl. {\it wobble stick}) eingebaut, um bestimmte Arbeitsgänge im Vakuum, zum Beisiel das Betätigen der Kühlschildklappe (Kap. 3.5) oder das Abreißen der Spalthebel (siehe Kap. 5.5), zu ermöglichen. Mit der Zweitrotation des Kryostaten, d.h. die Rotation um die Probennormale, können die Proben optimal orientiert werden.

Ebenfalls wurde in diese Ebene ein LEED/Auger-System (Kap. 3.4) eingebaut. Dadurch können strukturelle wie chemische Eigenschaften der Proben in situ untersucht werden. Insbesondere kann die Probenorientierung durch LEED überprüft werden. Ist die Probe einmal orientiert, wird sie mit dem Kryostaten in die untere Ebene, die Meßebene, gefahren. In der Meßebene fallen (ein genau einjustiertes Experiment vorausgesetzt) der Fokus der Synchrotronstrahlung aus dem Monochromator und 
\newpage


% ************************************************* Kapitel 5: Das Material HgCd Te und Charakterisierung der Proben ******************************************

\chapter{Das Material Hg\textsubscript{1-X}Cd\textsubscript{X}Te und Charakterisierung der Proben}

\section{Eigenschaften}

All measurements were taken with the SCIENTA SES 100 and the WESPHOA
I at the laboratry in Adlershof, Newtonstr. 14. Very usefull for interpretration
of the mesurement data was all GPL-software and so on.


\section{Herstellungsverfahren}



\section{Übersicht der untersuchten Proben} % 5.3

Grundlage dieser Diplomarbeit, deren Thema 2004 ausgeschrieben wurde, waren drei Kristal-
le von Hg1-XCdXTe mit x-Werten zwischen 0.07 und 0,4. Sie sind in Abbildung 5.4 mit I bis
III gekennzeichnet. Diese Proben wurden vom Institut für Tieftemperaturphysik an der Staat-
lichen Moskauer Lomonossow-Universität zur Verfügung gestellt. Insbesondere besteht eine
enge Zusammenarbeit mit Dr. Nikiforov, er hatte die Beschaffung dieser Proben organisiert. Im
Dezember 2006 wurden uns drei weitere Proben zur Verfügung gestellt. Sie sind mit IV bis VI
bezeichnet.

\section{Laue-Aufnahmen zur Kristallqualität} % 5.4

Ausgehend von den Besonderheiten des Herstellungsverfahrens der HgCdTe-Proben ist es not-
wendig, die Kristallqualität zu prüfen. Insbesondere kann nicht davon ausgegangen werden,
dass es sich um Einkristalle handelt. Vielmehr sollten polykristalline Proben mit Korngrößen
von einigen Mikrometern erwartet werden [47].

Die Röntgenbeugung nach Laue ist gut geeignet, um die Kristallinität und Orientierung
eines ganzen Volumens zu bestimmen. Dazu wird ein Lauegramm aufgenommen, also eine
Transmissionsaufnahme erstellt. Damit jedoch auf dem Film hinter der Probe eine ausreichen-
de Intensität der gebeugten Strahlung detektiert werden kann, müssen die Proben hinreichend
dünn sein. Mit der verwendeten Anlage (Kapitel 4.5) konnte noch eine Probendicke von 600
µm untersucht werden.

Abbildung 5.5: Transmissionsaufnahme der Röntgenbeugung nach Laue von Probe I (x=0,07). Die
scharfen Beugungsmaxima in diesem Lauegramm deuten auf eine homogene Orientierung der
Kristalle hin. Die sichtbaren Ringe hingegen erinnern an die Pulvermethode von Debye-Scherrer
und zeigen deutlich, dass kein Einkristall vorliegt. Diese Ringe werden an den Gitterebenen der
Korngrenzen in dieser polykristallinen Probe erzeugt.

Das Ergebnis der Transmissionmessung zeigt Abbildung 5.5. Wie zu erwarten enthält die
Aufnahme ringförmige Schwärzungen von den polykristallinen Anteilen der Probe. Es han-
delt sich demnach bei den zur Verfügung gestellten Kristallen nicht um Einkristalle, wie es
im Thema der Diplomarbeit angegeben wurde. Allerdings besitzen die Körner eine einheitli-
che Orientierung, wie aus den punktförmigen und relativ scharfen Reflexen hervorgeht. Die
weniger scharfen Punkte der intensiven Reflexe ergeben sich aus der Dimension und Intensi-
tätsverteilung des kollimierten Röntgenstrahls.

Die Reflexionsaufnahmen hingegen zeigen eine sehr klare Struktur und weder Ringe von
polykristallinen Anteilen noch Punkte anderer Kristallorientierungen. Bei der Herstellung der
Proben wurden diese aus den Blockkristallen herausgesägt und die Oberfläche poliert. So
konnte der Röntgenstrahl der Röntgenapparatur mit einem speziellen Mikroskop (Abbildung
4.5) exakt senkrecht zur Oberfläche ausgerichtet werden. Die Orientierung der Kristallstruktur
ist nicht immer parallel zur Probenoberfläche, sondern auch gedreht (Abbildung 5.6) oder um
einen gewissen Winkel (Abbildung 5.7) verkippt.

Abbildung 5.6: Epigramm von Probe II (x=0,4).
Deutlich ist die gedrehte, doch gut zentrierte
{110}-Orientierung zu erkennen.

Abbildung 5.7: Die Kristallstuktur von Probe V
(x=0,1855) ist um 4 Grad zur Oberflächennorma-
len verkippt (Orientierung {110}).

Mit den Laueaufnahmen wurde die Orientierung aller sechs Proben überprüft. Auch die-
se Messungen wurden mit der Anlage in Moskau durchgeführt. Die Aufnahmen wurden in
40mm Abstand mit 32 kV bei 30mA in 10 Minuten Belichtungszeit aufgenommen. Exempla-
risch dienen die Aufnahmen 5.6 und 5.7.

Aus den entwickelten Photos konnte abgelesen werden, dass die urspünglichen Blockkris-
talle in Längsrichtung [111]-orientiert waren. Senkrecht dazu wurden sie in Scheiben geschnit-
ten. Die Oberflächen dieser Scheiben sind demnach (111)-Flächen. Sie wurden poliert und bil-
den jeweils die Flächen unserer Proben mit den größten Abmaßen. Bei unseren Proben handelt
es sich somit um Bruchstücke oder Restmaterial dieser Scheiben. Die Seitenflächen schließen
einen Winkel von 90◦ zur Oberfläche ein und sind (110)-orientiert. Da dies auch die natürliche
Spaltfläche von ZnS-Kristallen ist, lassen sich die Proben parallel zu den Seitenflächen sehr gut
spalten.

\section{Bestimmung der Zusammensetzung mittels energiedispersiver Röntgenspektroskopie} % 5.5 

Jeweils nach den Photoemissionsmessungen im März und August 2007 wurde die Zusammen-
setzung der gemessenen Proben mittels energiedispersiver Röntgenspektroskopie oder EDX1
untersucht. Diese Messungen wurden von Dr. Schäfer in unserem Institut durchgeführt.

Das Ergebnis von 15 untersuchten Proben, die an jeweils drei Punkten der Probenoberflä-
che spektroskopiert wurden, finden sich im Anhang A in den Tabellen A.1 und A.2. Aus der
Vielzahl der verfügbaren Daten wurde das statistische Mittel gebildet und in Tabelle 5.2 zu-
sammengefasst. Probe III wurde weder per Photoemission noch per EDX untersucht, daher
liegen hierzu keine Daten vor.

Probe Referenzwert x

Tabelle 5.2: Ergebnisse der EDX-Messungen an fünf der verfügbaren Proben. Die gemessenen
relativen Intensitäten wurden auf 2 Atome pro Einheitszelle renormiert.

Zunächst fällt der Tellur-Anteil in der Einheitszelle ins Auge. Ausgehend von der Sum-
menformel Hg1-XCdXTe sollte dieser stets 1 betragen. Jedoch wird er in EDX wiederholt und
reproduzierbar zu gering gemessen. Der statistische Mittelwert des Telluranteils aus 45 Mes-
sungen beträgt 0,9176 ± 0,0005. Die Genauigkeit jeder Einzelmessung beträgt 8 % relativen
Fehlers für Tellur (siehe Anhang A) und genügt nicht, um die Abweichung zu erklären.

Offensichtlich liegt ein systematischer Fehler vor. Mangels Vergleichsmessung kann nicht
gesagt werden, ob dieser Fehler durch die EDX-Anlage hervorgerufen wird oder die Proben tat-
sächlich einen geringen Tellur-Anteil aufweisen. Allerdings geben eine Vielzahl anderer Mes-
sungen mit der EDX-Anlage keinen Anlass zur Beanstandung. Mit großer Wahrscheinlichkeit
weisen daher die Kristalle einen Mangel an Tellur auf.

Die anderen Messwerte bestätigen recht gut die angegebenen Daten der Zusammenset-
zung. Aufgrund der Normierung kann der relative Anteil Cadmium direkt mit dem Referenz-
wert für x verglichen werden. Doch wie bereits erwähnt wurde, ist die Methode, mit der die
Referenzwerte bestimmt wurden, unbekannt. Ein weiterer Vergleich mit den Messwerten ist
daher nicht sinnvoll. Zudem liegen keine Angaben zur Genauigkeit der Referenzwerte vor.


\section{Überprüfung der Kristalle mit dem Atomkraftmikroskop} % 5.6

Im Moskauer Institut ABMR (Advanced BioMedical Research) stand uns ein Atomkraftmikro-
skop oder AFM2 zur Verfügung. So wurde die Oberfläche aller sechs Proben auch mit dieser

1EDX - Energy Dispersive X-ray spectroscopy
2AFM - Atomic Force Microscope

Methode untersucht. Die Messungen erfolgten an den (111)-Flächen, die beim Zertrennen der
Blockkristalle durch Sägen und Polieren präpariert worden waren (siehe Kap. 5.4).

In Abbildung 5.8 sieht man die parallenen Strukturen, die beim Trennen der Kristalle mit-
tels Elektroerosion entstehen. Man erkennt in der linken Bildhälfte einen „Kanal“ von 3 µm
Breite und 50 nm Tiefe. Die verbliebenen Spuren erscheinen nur aufgrund der starken Ver-
größerung bzw. des kleinen Ausschnittes parallel, relativ zum Blockkristall sind sie natürlich
rund.

Wenn wir nun einen Ausschnitt von 3x3 µm näher betrachten, so erkennt man noch weitere
feine parallele Strukturen innerhalb des Kanals, die eindeutig mit der Säge erzeugt wurden
(Abbildung 5.9). Sie besitzen eine Höhe von 5-10 nm. Jedoch sind zwischen diesen parallelen
Strukturen deutlich kleine Erhebungen oder „bumps“ zu sehen. Ihre Höhe beträgt 10-20 nm
und ihr Durchmesser ca. 150 nm.

Abbildung 5.8: AFM-Bild der (111)-Oberfläche
von HgCdTe. Sie wurde durch Elektroerosion
gesägt (parallele Strukturen).

Abbildung 5.9: Ausschnitt von 3 x 3 µm aus der
linken Aufnahme. Deutlich sind die Fehlstellen
im „Graben“ zu erkennen.

Diese kleinen Erhebungen weisen auf Fehlstellen des Gitters hin. Anhand der Fragmen-
te der Oberflächenpräparation (parallele Strukturen von der Säge) wird deutlich, dass diese
Fehlstellen erst nach dem Zertrennen der Blockkristalle an diese Stelle gelangt sind. Die Be-
weglichkeit von Fehlstellen wurde von Spicer et. al.[48] berichtet. Er erwähnt, dass sich diese
Gitterfehler bereits bei Raumtemperatur im Kristall bewegen können. Sie gelangen so aus dem
Volumen bis an die Oberfläche. Die Ursache dieser Beweglichkeit liegt in der schwachen Hg-
Bindung innerhalb des Volumenkristalls.

Ebenso schreibt er die hohe Oberflächenempfindlichkeit gegenüber mechanischen Einflüs-
sen dieser schwachen Bindung zu. Auch dieser Aspekt wird mit einem Atomkraftmikroskop
deutlich. Bei vielen Messungen konnten wir kleine Kratzer auf der Oberfläche entdecken. Als
Beispiel diene hier Abbildung 5.8.



% ********************************************* Kapitel 6: Präparation der (110) Oberfläche ******************************************

\chapter{Präparation der (110) Oberfläche von HgCdTe}

In Kapitel 3 wurde gezeigt, dass die Photoelektronspektroskopie eine oberflächensensitive Un-
tersuchungsmethode ist. Ursache hierfür ist die geringe mittlere freie Weglänge der Elektronen
im Kristall (siehe Abbildung 3.3). In den meisten Fällen soll die elektronische Struktur des Vo-
lumenkristalls bestimmt werden. Dies setzt eine Oberfläche voraus, die sich möglichst wenig
vom Volumen unterscheidet. Eine unvermeidliche Veränderung im Falle der II/VI-Halbleiter
ist die Relaxation der (110)-Oberfläche (siehe Kapitel 2.3).

Darüber hinaus ist die Oberfläche unter normalen Umgebungsbedingungen der Wechsel-
wirkung mit den jeweiligen Gasen der Luft ausgesetzt. Diese können die Oberfläche bedecken,
physikalisch verändern oder eine chemische Bindung eingehen. Ebenso können Flüssigkeiten
(z. B. Wasser) an der Oberfläche kondensieren. Für die Untersuchung mittels ARPES ist es also
notwendig, die Veränderungen der Oberfläche zu entfernen. Weiterhin ist sicherzustellen, dass
es nicht zu erneuten Bedeckungen und Reaktionen kommt. Daher werden Photoemissionmes-
sungen unter UHV (Ultra High Vacuum, $p < 10^{-9}mbar$) durchgeführt. Anhand eines einfachen
Modells lässt sich mithilfe der kinetischen Gastheorie zeigen, dass unter UHV-Bedingungen
Messungen an der ungestörten Oberfläche von mehreren Stunden möglich sind.

Zu Beginn der Messungen ist deshalb eine saubere Oberfläche sicherzustellen. In einigen
Experimenten wird die zu untersuchende Oberfläche direkt im Vakuum erzeugt (z. B. mittels
MBE) und dann ohne Unterbrechung des Vakuums zur UPS Messung transferiert. Andere Her-
stellungsverfahren für die zu untersuchenden Materialien ermöglichen diese Vorgehensweise
leider nicht. In diesem Fall muss die Oberfläche im Vakuum vor der eigentlichen Messung
gereinigt oder präpariert werden. Zwei häufig verwendete Verfahren werden im folgenden
vorgestellt und auf ihre Anwendbarkeit bei HgCdTe geprüft.

\section{Sputtern und Annealen} 

Bei dem Verfahren des Sputterns (aus dem englischen to sputter = zerstäuben) werden Atome
aus dem Festkörper (Target) durch Beschuss mit energiereichen Ionen herausgelöst. Für den
Beschuss eignen sich besonders Edelgase wie Argon, da sie keine chemische Reaktion mit dem
Target eingehen. Die chemische Zusammensetzung des Targets wird daher durch das Sputtern
nicht verändert. Die in einem Niederdruckplasma generierten positiven Edelgasionen werden
durch eine Gleichspannung zum Target beschleunigt. Die herausgeschlagenen Atome des Tar-
gets gehen in die Gasphase über und kondensieren auf den Wänden der Vakuumkammer oder
werden durch das Pumpsystem aus dem System entfernt.

Durch den Sputterprozess werden die obersten Schichten des Targets abgetragen. Dabei
entsteht eine sehr raue Oberfläche. Für die Messungen muss diese daher wieder geglättet wer-
den. Dazu wird die Probe auf eine Temperatur nahe unter der Schmelztemperatur erwärmt.
Durch die Oberflächenspannung des Probenmaterials glättet sich die Oberfläche wieder. Dieser
Prozess wird Annealen (engl. für tempern, anlassen) genannt. Währenddessen erhöht sich der
Druck in der Vakuumkammer signifikant, da der partielle Dampfdruck aller Elemente mit der
Temperatur ansteigt. Das Restgas setzt sich allerdings fast ausschließlich aus den Elementen
der Probe zusammen, die mit der erwärmten Probenoberfläche in einem dynamischen Gleich-
gewicht steht. Durch die Oberflächenspannung des Probenmaterials glättet sich die Oberfläche
wieder. Die Kombination aus Sputtern und Annealen wird häufig bei der Untersuchung von
Metalloberflächen angewendet.

Die Kombination dieser beiden Methoden kann oftmals nicht angewendet werden, wenn
sich die Schmelztemperaturen der einzelnen Komponenten des zu untersuchenden Materials
stark voneinander unterscheiden. Im Falle von HgCdTe betragen die Unterschiede fast 500 Kel-
vin. Cadmium besitzt eine Schmelztemperatur von 594 K (321 ◦C) und Tellur von 723 K (450
◦C). Quecksilber hingegen geht bereits bei 234 K (-39 ◦C) in die flüssige Phase über. Die Siede-
temperatur von Quecksilber liegt bei 630 K (357 ◦C), noch unterhalb der Schmelztemperatur
von Tellur. Man sollte erwarten, dass das Quecksilber aufgrund seines hohen Dampfdruckes
aus einer erwärmten Probe herausdiffundiert [49]. Bei Raumtemperatur hingegen ist HgCdTe
stabil und behält seine Komposition bei. Die Schmelztemperatur für MCT1 liegt je nach Zu-
sammensetzung zwischen der von HgTe (670 ◦C) und CdTe (1090 ◦C, nach [46], Seite 56).

Abbildung 6.1: Mikrosondenanalyse einer erwärmten HgTe Oberfläche (rechte Bildhälfte) [50]. In der
jeweils linken Bildhälfte ist der Sputterkrater zu sehen, dort wird der Sollwert des Hg-Anteils gemessen.
In der linken Abbildung wurde die Oberfläche auf Tellur untersucht, rechts auf Quecksilber.

Eine experimentelle Untersuchung in unserer Arbeitsgruppe [50] zeigt, dass die Methode
des Annealens bei HgCdTe tatsächlich nicht angewendet werden kann. Es wurde die Oberflä-
che von HgTe untersucht. Die in dieser Diplomarbeit untersuchten Proben haben einen relativ
hohen Anteil Quecksilber, so dass sich die Ergebnisse von HgTe direkt übertragen lassen. Zu-
nächst wurde eine Probe von HgTe erhitzt. Dann wurde ein Teil der Oberfläche durch Sputtern
mit Argonionen gereinigt. Dadurch wurde eine Schicht von einigen Nanometern abgetragen
und somit eine durch das Annealen veränderte Oberfläche entfernt.

Danach wurde die Oberfläche mit der Mikrosondenanalyse untersucht. Das Ergebnis ist
in Abbildung 6.1 zu sehen. Bei dieser Untersuchung erfolgte die Anregung mit 10 keV Rönt-
genstrahlen. Die Auflösung der Elektronenstrahlmikrosonde beträgt 2 µm, die Bilder haben
eine Kantenlänge von 180 µm. Die rechte Bildhälfte zeigt die erwärmte HgTe-Oberfläche. Es
ist deutlich zu sehen, dass die Probe fast kein Quecksilber mehr an der Oberfläche enthält. In
der linken Bildhälfte erkennt man den Sputterkrater. Dort messen wir wieder das erwartete
Verhältnis von Quecksilber zu Tellur.

Damit eignet sich die Methode des Sputterns und Annealens nicht, um die Oberfläche der
zu untersuchenden Proben von Hg1-XCdXTe für eine ARPES-Messung vorzubereiten. Das ist
in Übereinstimmung mit den Ergebnissen von Spicer et. al. [48], [51].


\section{Spalten der Proben im Vakuum} % 6.2

Eine weitere Möglichkeit, eine saubere Oberfläche im Hochvakuum zu erzeugen, ist das Spal-
ten der Kristalle. Besonders Schichtkristalle sind für diese Methode der Oberflächenpräparati-
on geeignet. Da ihre Ebenen nur mit einer schwachen Van-der-Waals-Bindung zusammenge-
halten werden, ist die zur Spaltung benötigte Kraft sehr gering. Auf diese Weise werden zum
Beispiel die Oberflächen von Hochtemperatursupraleitern für Messungen im Vakuum vorbe-
reitet. Dazu genügt es, Tesafilm auf die Probe aufzukleben und diesen im Vakuum abzuziehen.
Die kovalenten Bindungen innerhalb der Ebenen sorgen für die nötige mechanische Stabilität
des Restkristalles. Man erhält sehr ebene Oberflächen von wenigen Angström Rauhigkeit (sie-
he [52], Seite 43). Neben der Photoelektronenspektroskopie benötigen auch Untersuchungen
mit dem Rastertunnelmikroskop oder STM2 saubere Oberflächen.

Bei Kristallen, die durch kovalente oder ionische Bindungen zusammengehalten werden,
gelingt das Spalten jedoch nicht so einfach. Es ist notwenig, eine spezielle Spaltvorrichtung zu
konstruieren (siehe [53], [54] und [55]). Dr. C. Jannowitz hat eine solche Spaltkammer zur Un-
tersuchung der elektronischen Struktur von CdTe konstruiert. Im Rahmen seiner Doktorarbeit
konstruierte auch N. Orlowski eine Spaltkammer, um die Oberflächen von HgTe für die Un-
tersuchung mittels PES vorzubereiten. Daher wurde erneut ein Spaltmechanismus konstruiert,
um die Proben von HgCdTe für die Photoemissionsmessung zu präparieren.

\subsection{Konstruktion einer Spaltkammer} % 6.2.1 

In Abbildung 6.2 sind die Spaltkammer von N. Orlowski und die neu aufgebaute Spaltkammer
zu sehen. Aus früheren Spaltversuchen war bekannt, dass nur gekühlte Spaltungen erfolgreich
sind. Eine Anforderung an die Spaltkammer war daher die Möglichkeit, die Probe vor der ei-
gentlichen Spaltung hinreichend zu kühlen. Der Ansatz ist in Bild (c) zu sehen. Im Zentrum ist
eine Adapterplatte zu erkennen, auf die ein Probenhalter geschraubt werden kann. Für einen
optimalen thermischen Kontakt ist die Oberfläche poliert. Diese Platte bildet den Abschluss ei-
nes Edelstahlrohres, das mit einem CF63-Flansch in einen Manipulator geschraubt ist und oben
eine Öffnung zum Einfüllen einer Kühlflüssigkeit bietet. Das Rohr wurde von der Werkstatt für
UHV-Umgebungen passend verschweißt.

Abbildung 6.2: Links die Spaltkammer von N. Orlowski [7], in der Mitte die im Rahmen dieser Diplom-
arbeit gebaute Spaltkammer. Rechts ein Blick in die Spaltkammer: Links ist der Amboss zu erkennen,
rechts der Spaltkeil und oben die Aufnahmeplatte für den Probenhalter.

Von links ragt ein Amboss in die Vakuumkammer hinein. Er ist auf einer schraubbaren Li-
neardurchführung befestigt. Ihm gegenüber ist der eigentliche Spaltkeil auf einer frei beweg-
lichen Lineardurchführung befestigt. Der Manipulator erlaubt eine exakte Positionierung der
Probe in x-, y- und z-Richtung. Am Probenhalter kann ein PT100 zur Temperaturmessung be-
festigt werden. Mit einer vergleichbaren Apparatur von N. Orlowski wurde bei Kühlung mit
flüssigem Stickstoff innerhalb weniger Minuten eine Temperatur von 80K erreicht. Der direkte
und großflächige Kontakt zwischen Kühlbehälter und Probenhalter bietet demnach deutliche
Vorteile gegenüber einer Kühlung in einem gewöhnlichen Manipulator. Bei letzterem erfolgt
der Wärmetransport vornehmlich über Kupferlitze. Der Kühlmittelbedarf ist ebenfalls bedeu-
tend höher [7].

\subsection{Spaltmechanismus im Probenhalter}

Es stellte sich heraus, dass die von Moskau zur Verfügung gestellten Proben nur sehr klein
waren. Insbesondere wiesen sie eine Dicke von weniger als einem Millimeter auf. Dieser Um-
stand stellt besondere Anforderungen an die genaue Positionierung der Proben im Spaltme-
chanismus. Während der Präparation zeigte sich, dass sie aufgrund ihrer geringen Größe me-
chanisch nicht sehr stabil sind. Das bedeutet, dass die zur Spaltung notwendige Kraft trotz der
kovalenten Bindungen recht gering ist.

Es wurde daher der Versuch unternommen, die Kristalle mit einer bei Schichtkristallen üb-
lichen Methode zu spalten. Dazu werden die Kristalle mit leitfähigem Silberkleber in einem
speziell vorbereiteten Stempel des Probenhalters geklebt. Dieser härtet bei 120 ◦C Wärmebe-
handlung innerhalb einer Stunde aus. Danach wird ein passender Spalthebel auf die Oberseite
der Probe geklebt. Zuvor wird eine Nut von 1 Millimeter Breite in den Aluminiumspalthebel
gesägt, um ihn während des Aushärtens besser auf der Probe zu positionieren. Darüber hin-
aus steht so eine größere Fläche für die Kraftübertragung beim eigentlichen Spaltprozess zur
Verfügung (siehe Abbildung 6.3).

Die so präparierten Proben wurden über das Probenkarussell in das Vakuumsystem trans-
feriert. Mit dem Wobbelstick konnten die Proben sehr einfach gespalten werden, während sie
sich im Kryostatmanipulator (Kapitel 4.1) befinden. Die benötigten Kräfte sind so gering, dass
einige Proben bereits beim Einbau gespalten wurden (siehe Anhang A). Daher wurden die
Spalthebel verkleinert, um die Kräfte auf die Proben durch die Massenträgheit des Hebels zu
minimieren.

Abbildung 6.3: Schematische Darstellung der Spaltmechanik direkt im Probenhalter.
Rechts ist ein Probenhalter mit Probe und Spaltkeil photographiert.

Allerdings spalteten die Kristalle an unvorhersagbaren Stellen und neigten zur Splitter-
bildung. Daher wurden die Kristalle an Hr. Sölle gegeben, der sie in kleinere Einzelkristalle
zersägte. Außerdem sägte er mit der Fadensäge einen 100 µm tiefen Graben in eine Seite der
Kristalle. Dies sollte als Sollbruchstelle dienen. Schematisch ist der finale Aufbau zum Spalten
der Probe in Abbildung 6.3 gezeigt. In der Praxis ließen sich die Kristalle erfolgreich entlang
dieser Sollbruchstelle teilen. Eine Aufnahme ist in Abbildung 6.4 links zu sehen.

Abbildung 6.4: Die linke Aufnahme mit einem Rasterelektronenmikroskop zeigt eine Spaltung entlang
der Sollbruchstelle. Diese ist als 100 µm tiefer Graben im Vordergrund zu erkennen. Das mittlere Bild
zeigt die fragmentierte Oberfläche nach einer Spaltung im Vakuum. Rechts ist die Photographie eines
bei der Spaltung zersplitterten Kristalls zu erkennen.

\section{Aufnahmen der Spaltungen mit dem Rasterelektronenmikroskop}

Die weiteren Messungen mit dem Rasterelektronenmikroskop oder SEM3 an den gespaltenen
Proben legen deutlich offen, dass die Proben keinesfalls einkristallinen Charakter besitzen. Dies
wurde schon durch die Laue-Transmissionsaufnahme (Abbildung 5.5) gezeigt. Makroskopisch
weisen die Proben eine einheitliche Orientierung auf, sind aber in jedem Fall polykristallin.
Darauf weisen auch die beiden Aufnahmen des Elektronenmikroskopes in Abbildung 6.5 hin.
Die linke Aufnahme zeigt das gespaltete Kristall im Probenhalter. Für das rechte Bild wurde
ein Ausschnitt um den Faktor 10 vergrößert.

Abbildung 6.5: Elektronenmikroskopaufnahme einer gekühlten Spaltung einer HgCdTe-Probe. Deutlich
sind Stufen von einigen µm Höhe zu erkennen. Das Kristall zeigt innere Spannungen. Der Maßstab der
Aufnahme beträgt links 1 mm, rechts 100 µm.

Schon bei der Auflösung des Elektronenmikroskopes wird eine uneinheitliche Oberflächen-
struktur sichtbar. Man kann nicht von einer homogenen Spaltung entlang einer (110)-Ebene
sprechen. Vielmehr sind Stufen von einigen Mikrometern Höhe erkennbar sowie innere Span-
nungen des Kristalls. Diese Spannungen haben den Verlauf der Spaltung entscheidend be-
stimmt. Bei der Interpretation winkelaufgelöster Messungen von Photoelektronenspektren be-
nötigt man natürlich eine Referenzebene. Offensichtlich ist diese nur bedingt gegeben, man
sollte in den Spektren Anteile unterschiedlicher Oberflächenorientierungen erwarten. Neben
Effekten der Oberflächen-Relaxation werden zusätzliche Anregungen von weiteren Oberflä-
chenzuständen zu sehen sein.

Die Anforderungen an die Oberflächenqualität für eine Photoemissionsmessung gehen über
den Mikrometerbereich weit hinaus. Es sind gute Eigenschaften auf atomarer Ebene erforder-
lich. Diese werden mit der nächsten Methode überprüft.

\section{Prüfung der Oberflächenqualität durch die Beugung langsamer Elektronen}

Eine übliche Methode, periodische Strukturen (z. B. Kristalloberflächen) auf atomarem Niveau
zu untersuchen, ist die sogenannte Beugung langsamer Elektronen oder LEED4. Die wenig
verheißungsvollen Ergebnisse des Elektronenmikroskopes bestätigten sich auch bei der Unter-
suchung mittels LEED. Mehrere Oberflächen wurden im Vakuum präpariert, indem sie unter
normalen Bedingungen gespalten wurden. Mit diesen Probenoberflächen war es nicht möglich,
ein Beugungsbild auf dem Phosphorschirm zu erhalten.

Aus der Untersuchung ähnlicher Kristalle [7] war bekannt, dass eine gekühlte Spaltung
weitaus bessere Ergebnisse liefern kann. So wurde eine Probe im Kryostat-Manipulator der
AR65 auf 100 K (-170 ◦C) gekühlt und mit dem Wobbelstick gespalten. Danach war es erstmals
möglich, das typische Beugungsmuster einer LEED-Aufnahme zu erkennen. Bei einer Anre-
gungsenergie von 156 eV entstand das Bild, welches in Abbildung 6.6 gezeigt ist.

Abbildung 6.6: Erstes erfolgreiches LEED-Beugungsbild
nach einer gekühlten Spaltung bei 100 Kelvin. Die Elek-
tronenenergie beträgt 156 eV.

Aus dem Beugungsbild kann man die Periodizität des Oberflächengitters berechnen. Für
die Berechnung benötigt man außerdem die Geometrie der Apparatur sowie die Elektronen-
energie. Es sind zwei unterschiedliche Gitterkonstanten zu erkennen, obwohl für HgCdTe stets
nur eine Gitterkonstante angegeben wird. Das liegt daran, dass eine (110)-Oberfläche unter-
sucht wurde. Da diese Ebene nicht parallel zu Flächen des Basisgitters verläuft, sondern dia-
2 verkürzt. Die Relationen der
gonal durch diese, ist eine Gitterkonstante um den Faktor
Oberflächenperiodizität zur Gitterkonstante betragen:

Mit der bekannten Anregungsenergie von 156 eV und der Geometrie der LEED-Apparatur

können die zugehörigen Längenwerte für b1 und b2 aus Abbildung 6.6 berechnet werden:

b1 = (3, 5 ± 0, 2)Å

b2 = (2, 4 ± 0, 2)Å

Aus den Werten b1 und b2 ergibt sich eine Gitterkonstante von 7,0 Å bzw. 6,8 Å. Der Sollwert
für diese Probe (x=0,2) beträgt a = 6, 464 Å. Die Ergebnisse der Beugung langsamer Elektronen
weichen somit um 10 \% von den realen Werten ab. Bei der Berechnung hat die Geometrie bzw.
deren exakte Einhaltung einen großen Einfluss auf das Endergebnis. In unserem Experiment
genügt bereits eine Abweichung der Probenposition um 7 mm vom Zentrum des Phosphor-
schirmes für einen relativen Fehler von 10 \% [23].

In Abbildung 6.6 ist zu erkennen, dass die Beugungsreflexe nicht exakt auf einer Linie lie-
gen. Offensichtlich befindet sich die Probe also nicht an der Referenzposition. Mit dem Ma-
nipulator der AR65 kann die Probe sehr genau relativ positioniert werden, über die absolute
Positionierung kann jedoch keine Aussage gemacht werden. Eine weitere Ungenauigkeit liegt
in der Höhe des Probenstempels sowie der Position des Kristalls in diesem.





% ********************************************* Kapitel 7: Ergebnisse der Photoemission ******************************************

\chapter{Ergebnisse der Photoemission}

\section{Energie der gemessenen Zustände}

In typischen Photoemissionsexperimenten bestimmt ein Analysator die Intensitäten der emit-
tierten Elektronen bei verschiedenen Energien. Ein angeregter Zustand des Festkörpers er-
scheint in diesen Spektren bei einer definierten kinetischen Energie. Der absolute Wert der
gemessenen kinetischen Energie für diesen Zustand ist allerdings von der Anregungsenergie
der Photonen abhängig (siehe Abbildung 3.2). In der Praxis werden daher diese Energien re-
lativ zu einem gemeinsamen Bezugspunkt angegeben, um Spektren und Energiewerte leichter
miteinander vergleichen zu können. Damit befinden sich gleiche Zustände in unterschiedli-
chen Messungen bei unterschiedlichen Photonenenergien bei der gleichen relativen Energie.

In den üblichen Arbeiten zur Photoemission finden insbesondere zwei Referenzpunkte Ver-
wendung. In vielen Fällen werden die kinetischen Energien auf die Fermi-Energie bezogen. Bei
Halbleitern ist es auch üblich, sich auf das Valenzbandmaximum (VBM) zu beziehen.

Es ist mit einigen Besonderheiten verbunden diese Bezugspunkte zu bestimmen. Der Ener-
gieanalysator ist innen mit Graphit beschichtet (siehe Seite 13). Normalerweise würde man er-
warten, die Fermi-Kante stets bei exakt derselben Energie im Abstand von der Photoenenergie
zu messen. Ausgehend von der einfachen Beziehung (3.1) erwartet man daher:

$$
E_f = h\nu - \Phi_{Graphit} = h\nu - 4,14 eV
$$

% EF = hν − ΦGraphit = hν − 4, 14 eV

In der Realität stimmt diese Gleichung jedoch nicht mit den Messergebnissen überein. Die
Spannungsversorgungen der Elektronenlinsen und des Halbkugelanalysators besitzen eine
Spannungsstabilität von wenigen µV. Dennoch sind die gemessenen Energiewerte von den
Umgebungsbedingungen abhängig. Diese driften im Verlauf von Tagen um einige Millivolt.
Daher wird bei hochauflösenden Messungen, wie sie bei Hochtemperatursupraleitern durch-
geführt werden, jedes Mal vor oder nach der Messung die exakte Position der Fermi-Energie
bestimmt.

Diese Kalibrierung sollte auch bei weniger genauen Messungen regelmäßig durchgeführt
werden. Unbedingt erforderlich ist eine solche Messung, wenn die Anlage abgebaut und an
anderer Stelle (z. B. am Synchrotron) wieder aufgebaut wird. So wurde im Frühjahr 2007 mit
der Heliumlampe ($h\nu = 21.22 eV$) eine Fermi-Energie von EF = (17, 085±0, 004) eV bestimmt.
Daraus ergibt sich eine gemessene Austrittsarbeit $\Phi$ = (4, 135 ± 0, 005) eV. Die Messungen im
August 2007 bei BESSY ergaben jedoch eine Austrittsarbeit von $\Phi$ = (4, 180 ± 0, 005) eV. Diese
Angaben beziehen sich auf eine Passenergie von 10 eV. Für eine Passenergie von 5 eV verschiebt
sich die Fermi-Kante um weitere 130 meV (Messung Januar 2007).


Ein großer Vorteil der Messung an der Beamline eines Synchrotrons ist die einstellbare Pho-
tonenenergie. Aus dem kontinuierlichen Spektrum der Synchrotronstrahlung wird mittels ei-
nes Gittermonochromators die gewünschte Wellenlänge ausgewählt. Die relativen Energiepo-
sitionen des Monochromators werden reproduzierbar mit einer Genauigkeit von 5 meV an-
gefahren [56]. Für die BUS-XUV-Beamline wurde diese Angabe im Jahre 2007 nochmals über-
prüft. Allerdings bezieht sich diese Genauigkeit nicht auf die absoluten Energiewerte. So wur-
de im Januar 2007 mit der AR65 eine Austrittsarbeit von $\Phi = (3,92 \pm 0,03) eV$ = (3, 92 ± 0, 03) eV bestimmt. Dieser
Wert unterscheidet sich um 260 meV von dem Ergebnis im August desselben Jahres.

Abbildung 7.1: Gemessene Austrittsarbeit der AR65 in Abhängigkeit
von der Anregungsenergie. Sie wurde durch die Lage der Fermi-Kante
von polykristallinem Gold bei verschiedenen Anregungsenergien im
Januar 2007 bei BESSY gemessen. Mittelwert: $\Phi$ = (3, 92 ± 0, 03) eV.

Die gemessene Fermi-Energie ist außerdem sehr stark von der Photonenenergie abhängig
(siehe Abbildung 7.1), einzelne Werte weichen um 100 meV ab. Für die Auswertung der Spek-
tren haben wir daher den Wert verwendet, der bei der jeweiligen Photonenenergie empirisch
bestimmt worden war.

Eine Vielzahl von Arbeiten bezieht ihre Energieangaben auf das Valenzbandmaximum. Bei
Halbleitern tritt dieses Maximum jedoch nur am $\Gamma$-Punkt auf. Um diesen Energiewert bestim-
men zu können, muss die exakte Position des $\Gamma$-Punktes bekannt sein. Außerdem muss es mög-
lich sein, diesen Punkt mit der Messung zu erreichen. Diese Bedingung schließt sowohl den
richtigen Winkel relativ zur Oberfläche als auch die passende Anregungsenergie ein. Durch die
Dispersion der Energiebänder liegen die Maxima von anderen Punkten der Brillouin-Zone bei
niedrigeren Energien. (siehe Abbildung 7.14). Dieser Abstand kann mehr als ein Elektronen-
volt betragen. Bei Materialien mit einer geringen effektiven Masse der Löcher ist die Position
des $\Gamma$-Punktes sehr scharf begrenzt.

Bei den Messungen an HgCdTe ist es nur sehr selten gelungen, eine klar definierte Kan-
te als Maximum des Valenzbandes zu messen. Als Beispiel diene die Messung vom 16. 7.
2007 in Abbildung 7.2 an Probe I (x=0,07). Es ist die erste Messung mit der neuen Stickstoff-
Nachfüllanlage. Daher war das Spektrum mit unzähligen „Spikes“ von den Schaltvorgängen
des Magnetventils übersät. Der typische inelastische Untergrund hingegen trat vergleichswei-
se schwach in Erscheinung. Die sichtbare Kante liegt bei einer Energie von -0,23 eV relativ zur
Fermi-Energie. Mit der Anpassung an eine Gaussfunktion lässt sich dem Valenzbandmaxium
ein Zustand bei -0,44 eV zuordnen.

Abbildung 7.2: Aus diesem Spektrum wurden nur die Spikes entfernt.
Deutlich ist ein Valenzbandmaximum erkennbar. Die Anregung erfolg-
te mit der He-Lampe. Es wurde Probe I (x=0,07) untersucht.

Bei den meisten anderen Messungen war es nicht möglich, einen Energiewert für das Va-
lenzbandmaximum zu bestimmen. Zwar war es möglich, eine gut sichtbare Dispersion der Vo-
lumenzustände zu detektieren. Das trifft zum Beispiel auf die $k_\perp$-Messung an Probe I (x=0,07)
zu, die in Abbildung 7.14 gezeigt ist. Aufgrund der Dispersion lässt sich der $\Gamma$-Punkt der Ener-
gie von 115 eV zuordnen, der X-Punkt liegt bei 65 eV. Doch im Gegensatz zu Abbildung 7.2
kann am $\Gamma$-Punkt kein klares Maximum ausgemacht werden. Einen Ausschnitt des Spektrums
an diesem Messpunkt zeigt Abbildung 7.3.

Abbildung 7.3: Valenzband HgCdTe: Intensität der Photo-
emission beim $\Gamma$-Punkt (Anregungsenergie 115 eV).

Wie deutlich erkennbar ist, nimmt die Zustandsdichte oberhalb von 0,9 eV Bindungsenergie
stetig ab, um bei der Fermi-Energie (hier als 0 definiert) ihr Minimum zu erreichen. Im weiteren
ist daher bei allen Auswertungen von Spektren die zuvor bestimmte Energie der Fermi-Kante
als Referenzwert („Null“) angesetzt.

Aus anderen Arbeiten geht hervor, dass auch dort die Relativierung der kinetischen Energi-
en über die Fermi-Energie erfolgt. Zuvor wurde die Energie des Valenzbandmaximums (VBM)
relativ zu dieser Energie bestimmt. Dieser Abstand entspricht einer konstanten Energie, die
der Fermi-Energie abgezogen werden kann. In der Auswertung der Spektren wird dann das
VBM als Nullpunkt gesetzt. Der Nullpunkt kann damit durchaus ein Elektronenvolt unterhalb
der Fermi-Energie liegen. Somit könnten auch Zustände mit positiver Bindungsenergie auf-
tauchen ([7], Seite 46). Das ist aber kein Widerspruch, die Energie liegt noch deutlich unterhalb
der Fermi-Energie. Es handelt sich um Zustände in der Energielücke, die zum Beispiel von
Oberflächenzuständen (siehe Seite 6) hervorgerufen werden.

\section{Allgemeine Charakteristika}

Bei unseren Messungen an der BUS-XUV-Beamline bei BESSY standen uns Photonenenergi-
en bis zu 125 eV zur Verfügung (Gitter 1 - 500 Linien). Mit dieser Anregungsenergie haben
wir Probe IV (x=0,183) bestrahlt. Die emittierten Elektronen besitzen ein für HgCdTe typisches
Spektrum, wie es in Abbildung 7.4 gezeigt wird. Auf die einzelnen Bestandteile möchten wir
zunächst kurz eingehen.

In diesem Übersichtsspektrum erkennt man zunächst die Spin-Bahn-aufgespaltenen Kern-
niveaus des Tellur bei ca. 40 eV Bindungsenergie. Das Te4d5/2-Level liegt bei 40,1 eV, während
das Te4d3/2-Level um 1,5 eV abgespalten bei 41,6 eV liegt. Der Unterschied in den Intensitäten
entspricht der Besetzung der jeweiligen Orbitale mit 10 bzw. 6 Elektronen. Dieser Energiebe-
reich ist noch einmal extra in Abbildung 7.5 gezeigt.

Sehr nahe am Valenzband befinden sich weitere Kernniveaus des Quecksilber und Cad-
mium. In den untersuchten Proben dominierte der Quecksilberanteil. Die 5d-Level sind eben-
falls Spin-Bahn aufgespalten und befinden sich bei den Energien E(Hg5d5/2) = 8,27 eV bzw.
E(Hg5d3/2) = 10,08 eV. Auch hier ergeben sich die relativen Intensitäten aus der Besetzung der
Orbitale. Mit steigendem Cadmium-Anteil sind auch die Cd4d-Level im Spektrum sichtbar.
Ihre Energien betragen E(Cd4d5/2) = 10,92 eV bzw. E(Cd4d3/2) = 11,49 eV.

Das Photoemissionsspektrum bis zu 13 eV Bindungsenergie ist in Abbildung 7.6 genauer
dargestellt. Diese Messung erfolgte an Probe V (x=0,1955). Im Energiebereich von 4 bis 6,5 eV
überlagern sich s-artige [48] Valenzelektronen vom Quecksilber und Cadmium. Der Zustand
des Cadmium besitzt eine Energie von 5,17 eV und das Quecksilber-Level 5,95 eV. Die Lage so-
wie der Abstand von 780 meV stimmen mit den Ergebnissen von Silberman et. al. [53] überein.
Anzumerken ist, dass sich diese Arbeit auf das VBM bezog und daher Energiewerte erhält, die
0,5 eV geringer sind.

Im Bereich zwischen 4 eV Bindungsenergie und der Fermi-Energie liegen p-artige Zustände
[48], [57], die eine starke Dispersion aufweisen (siehe Abbildung 7.14). Die Messung in norma-
ler Emission wird sich hauptsächlich auf diesen Bereich konzentrieren.

Abbildung 7.4: Übersichtsspektrum einer Probe HgCdTe mit 125 eV Anregungsenergie bei BESSY.

Abbildung 7.5: Durch Spin-Bahn-
Kopplung
4d-Kern-
aufgespaltene
niveaus des Tellur.

Abbildung 7.6: Valenzbandspektrum von Probe V (x=0,1955)
mit den Kernniveaus von Quecksilber und Cadmium.

\section{Energie der Kernniveaus} % 7.3 

\subsection{Tellur 4d} % 7.3.1 

Kernniveaus zeichnen sich durch eine hohe Zustandsdichte aus, daher sind die gemessenen
Elektronenintensitäten bei diesen Energien besonders hoch. Das Tellur-4d-Niveau bei 40 eV
Bindungsenergie ist ein solcher charakteristischer Zustand. Während der Messungen am Syn-
chrotron war dieses Niveau gut geeignet, eine optimale Position der zu untersuchenden Probe
in der Messapparatur zu finden.

Abbildung 7.7: Doppelstruktur des Te4d-
Kernniveaus bei einer Messung von Probe I
(x=0,07). Offensichtlich liegt eine veränderte
chemische Umgebung vor.

Abbildung 7.8: Spin-Bahn-aufgespaltenes Kern-
niveau Te4d von Probe IV (x=0,183) in ungestör-
ter Umgebung. Die Energie der Aufspaltung be-
trägt (1,46±0,01) eV.

Neben der Positionierung eignen sich die Te4d-Kernniveaus auch, um die chemische Um-
gebung zu überprüfen. Bei einer Messung (Abbildung 7.7) zeigten sich eine Überlagerung von
zwei Spin-Bahn-aufgespaltenen Kernniveaus mit einer Energiedifferenz von $\Delta  = (0,53 \pm 0,02)$ eV.
Dieser sogenannte „chemical shift“ deutet auf eine veränderte chemische Umgebung hin [58].
Die untersuchte Probe weist also keine idealen Eigenschaften auf. In manchen Fällen ist eine
solche Verdopplung ein Hinweis auf eine Gitteränderung (z. B. $\beta$-MoTe\textsubscript{2}).

\subsection{Quecksilber 5d und Cadmium 4d} % 7.3.2

Bereits in früheren Photoemissionsmessungen von Shih et. al. wird von einer Verschiebung der
Kernniveaus Hg5d und Cd4d in Abhängigkeit von der Zusammensetzungs beobachtet [54].
Auch unsere Messungen zeigen unterschiedliche energetische Lagen der Kernniveaus in den
verschiedenen Proben. Einige Messwerte sind in Tabelle 7.1 zusammengetragen.

X Hg5d5/2 Hg5d3/2 Cd4d5/2 Cd4d3/2

0,40
0,18
0,07
0,06

8,35
8,30
8,28
8,05

10,15
10,09
10,04
9,87

10,68
10,69
10,66

11,37
11,38

Tabelle 7.1: Lage der Kernniveaus von Cadmium (Cd4d) und Quecksilber (Hg5d) in Abhängigkeit von
der Komposition. Der absolute Fehler beträgt 10 meV.

In unseren Messungen beobachten wir mit steigendem Cd-Anteil eine Verschiebung des
Hg5d-Niveaus um 300 meV zu größeren Bindungsenergien . Allerdings bezieht sich die Mes-
sung von Shih auf das VBM (siehe linke Ecke in Abbildung 7.9), während wir die Fermi-Energie
als Referenz verwenden. Der Unterschied könnte darin begründet liegen.

Abbildung 7.9: Die Hg5d und Cd4d Kernnive-
aus für HgTe, Hg0.7Cd0.3Te und CdTe, bezogen
auf das VBM. Das Hg 5d Level verschiebt sich
um 0.1 eV zu kleineren Bindungsenergien und
das Cd 4d Level um 0.25 eV zu höheren Bin-
dungsenergien in ternären Mischungen [54].

Abbildung 7.10: Die Hg5d und Cd4d Kernnive-
aus für Hg1-xCdxTe (Werte für x zwischen 0,05
und 0,4) bezogen auf die Fermie-Energie. Das
Hg 5d Level verschiebt sich um 0.3 eV zu größe-
ren Bindungsenergien, Cd 4d Level um 0.12 eV
ebenfalls zu höheren Bindungsenergien.

Spin-Bahn-Aufspaltung

Aus mehreren Messungen haben wir die Spin-Bahn-Aufspaltung der erwähnten Kernniveaus
bestimmt. Sie entsprechen den Ergebnissen von Silberman et. al. [53].

Tellur 4d

Cadmium 4d Quecksilber 5d
(1,47±0,02) eV (0,65±0,05) eV (1,79±0,08) eV


\section{Das Valenzband - Messung in normaler Emission} % 7.4

Die Photoelektronenspektroskopie in senkrechter Richtung zur Kristalloberfläche wird als Mes-
sung in normaler Emission oder $k_\perp$-Messung bezeichnet. Die Parallelkomponente des Wellen-
vektors ist daher null (siehe Formel (3.8)). Im Rahmen dieser Diplomarbeit wurden vier solcher
Messungen an unterschiedlichen Proben durchgeführt.

Abbildung 7.11: Erste Messung in normaler
Emission. Es wurde Probe II (x=0,4) untersucht.
Die Spaltung erfolgte bei Raumtemperatur.

Abbildung 7.12: Valenzbandspektrum von Probe
V (x=0,1955) in normaler Emission nach gekühl-
ter Spaltung.

Unsere erste Messung erfolgte mit Probe II (x=0,4). Das Ergebnis der ungekühlten Spal-
tung ist in Abbildung 7.11 zu sehen. Bei einer Anregungsenergie von 42 und 43 eV erschei-
nen Intensitäten oberhalb der Fermi-Energie. Dabei handelt es sich um Anregungen der Te4d-
Kernniveaus durch die zweite Beugungsordnung des Gittermonochromators. Die kinetische
Energie dieser Kernanregungen entspricht den kinetischen Energien der Valenzelektronen, die
durch die erste Beugungsordnung des Gitters angeregt werden. Störende zusätzliche Intensi-
täten treten auch noch bei niedrigeren Energien auf.

Auf die Schwierigkeiten der Oberflächenpräparation geht bereits Kapitel 6 ein. Da die Pro-
be bei Raumtemperatur gespalten wurde, ist mit einer uneinheitlichen Oberfläche zu rechnen.
Die Photoemissionsmessung bestätigt dies, denn es sind keine scharfen Intensitätsmaxima zu
erkennen. Vielmehr scheint es sich um eine Überlagerung vieler Zustände zu handeln. Das
Spektrum verliert damit seine Richtungsinformation und summiert unterschiedliche Wellen-
vektorwerte aufgrund der Oberflächenstruktur auf. Das Valenzband enthält im Bereich von 0
bis 4 eV keine auswertbare Dispersion.

Im Spektrum ist ein Band zwischen 4 und 6 eV Bindungsenergie sichtbar. Wir können es
s-artigen Zuständen der Kationen zuordnen [48], [11]. Dieses Band zeigt keine erkennbare Di-
spersion, nur eine geringe Intensitätsmodulation ist zu sehen. In theoretischen Bandstruktur-
rechnungen (siehe Abbildung 7.18 und 7.19) finden sich in diesem Energiebereich die Spin-
Bahn-abgespaltenen Lochbänder für die X, W und L-Punkte der Brillouinzone. Zum $\Gamma$-Punkt
hin weisen sie eine sehr starke Dispersion auf. Aufgrund der daraus folgenden geringen Zu-
standsdichte sind diese nur sehr schwer aufzulösen.

Da die Zustände im Energiebereich von 4
bis 6 eV offensichtlich keine Dispersion zeigen
(oder diese noch nicht aufgelöst werden kann),
haben wir uns im Folgenden entschlossen, nur
den Energiebereich von der Fermi-Energie bis
zu 4 eV Bindungsenergie zu untersuchen.

Eine zweite Messung in normaler Emissi-
on erfolgte nach nunmehr gekühlter Spaltung
an Probe V (x=0,1955). Das Ergebnis zeigt Ab-
bildung 7.12. In diesem Spektrum ist deutlich
ein Zustand bei 2 eV Bindungsenergie zu er-
kennen. Aufgrund seiner fehlenden Dispersion
können wir davon ausgehen, dass es sich um
einen Oberflächenzustand handelt (siehe Kapi-
tel 2.3 auf Seite 6). Ein weiterer dispersionsloser
Zustand findet sich bei ca. 3,4 eV in allen Spek-
tren in normaler Emission. Bei seiner Untersu-
chung von HgTe stellte N. Orlowski [7] einen
solchen Zustand bei 3,8 eV fest.

Abbildung 7.13: Bild der gelungenen Spaltung
von Probe I. Die $k_\perp$-Messung an dieser Proben-
oberfläche findet sich in Abbildung 7.14.

Da der Oberflächenzustand die Messung überdeckt, kann das Spektrum nicht weiter aus-
gewertet werden. Die beiden nachfolgenden Spaltungen und Messungen waren hingegen er-
folgreich. Zunächst wurde Probe IV (x=0,183) gekühlt gespalten. Die Dispersion der Zustände
ist in Abbildung 7.15 zu sehen. Man erkennt zum Beispiel ein Maximum des Valenzbandes bei
einer Energie von 115 eV. Auch die Dispersion weist auf einen hochsymetrischen Punkt hin.
In Verbindung mit der folgenden Auswertung kann dieser Energie dem $\Gamma$-Punkt zugeordnet
werden.

Die letzte Messung erfolgte an Probe I (x=0,07). Die gekühlte Spaltung war erfolgreich, wie
die Aufnahme mit dem Rasterelektronenmikroskop in Abbildung 7.13 erkennen lässt. In die-
sem Fall wählten wir Anregungsenergien bis zu 125 eV, obwohl die Intensität an dieser Stelle
durch die Beamline bereits stark eingeschränkt ist. Aufgrund von Symmetriebedingungen ist
es klar, dass sich der $\Gamma$-Punkt bei 115 - 120 eV befindet. Ein weiteres lokales Maximum findet
sich bei 65 eV. Diese Energie entspricht dem X-Punkt.

Insgesamt konnten die Messungen den hohen Erwartungen nicht gerecht werden. Vergli-
chen mit den Ergebnissen von N. Orlowski [34] an HgTe [59] und C. Janowitz an CdTe [63] fällt
sofort die geringe erreichte Auflösung an der Valenzbandkante auf. Bedingt durch die geringe
Anzahl Proben und die eingeschränkten Möglichkeiten zur Messung kann auch die Auswer-
tung (Abschnitt 7.6) nur begrenzt stattfinden.

Abbildung 7.14: Messung in $k_\perp$ an Probe I (x=0.07). Deutlich ist die Dispersion des VBM zu erkennen
mit symmetrischen Punkten bei 120 und 65 eV Anregungsenergie. Durch Anpassung können diese
dem $\Gamma$-Punkt bzw. X-Punkt zugeordnet werden.

Abbildung 7.15: Messung in $k_\perp$ an Probe I (x=0.16). Das Valenzbandmaximum bei 115 eV Anre-
gungsenergie kann dem $\Gamma$-Punkt zugeordnet werden. Die Dispersion bestätigt diese Zuordnung.

\section{Winkelaufgelöste Photoemissionsmessungen} % 7.5

Die Photoemissionsanlage AR65 wurde gebaut, um winkelaufgelöste Messungen in hoher Auf-
lösung zu ermöglichen. Daher wurde auch mit HgCdTe versucht, winkelabhängige Spektren
der Photoelektronen zu messen. Die ersten Ergebnisse zeigen die folgenden Abbildungen.

Abbildung 7.16: Winkelabhängige Messung des
Valenzbandes und von Kernniveaus. Es zeigt
sich keine Dispersion.

Abbildung 7.17: Messung der Winkelabhängig-
keit in größerer Genauigkeit und über einen wei-
teren Winkelbereich.

Der Energiebereich von der Fermi-Energie bis zu 12 eV Bindungsenergie deckt die Kernni-
veaus von Quecksilber und Cadmium ab. Das in Abbildung 7.16 gezeigte Ergebnis der winkel-
aufgelösten Messung zeigt erwartungsgemäß keine Dispersion für die Kerniveaus. Auch das
s-artige Band von 4 bis 6,5 eV zeigt keine Abhängigkeit vom Wellenvektor. Für das Valenz-
band ist eine leichte Dispersion zu erkennen. Allerdings reicht die Auflösung nicht aus, um
eine Auswertung durchzuführen.

Eine weitere Messung wurde über einen Winkelbereich von $-15^\circ < \phi < 30^\circ$ durchgeführt.
Dabei wurden nur Energien bis 9 eV untersucht, um insbesondere das Verhalten der Valenz-
elektronen zu untersuchen. Jedoch ist keine Dispersion erkennbar. Wahrscheinlich sind durch
eine schlechte Oberflächenqualität sämtliche Richtungsinformationen verlorengegangen und
ergeben somit eine winkelintegrierte Intensität.

Auch weitere winkelaufgelöste Messungen waren nicht erfolgreich. Selbst nach gekühlter
Spaltung konnten keine besseren Ergebnisse als die gezeigten erzielt werden. Eine Messung
an der Ausbildungsbeamline BESSY mit dem neuen Scienta der Arbeitsgruppe EES würde
weitaus schellere winkelabhängige Ergebnisse liefern und könnte so auch schneller über er-
folglose Spaltungen Auskunft geben.

\section{Bandstruktur und Theorie} % 7.6

Die Bandstrukturrechnungen zu CdTe und HgTe weisen große Ähnlichkeiten auf [61]. Zwar
unterscheiden sich diese beiden Halbleiter im Betrag ihrer fundamentalen Lücke. Mit der Pho-
toelektronspektroskopie sind jedoch nur besetzte Zustände einer Probe sichtbar. Diese liegen
unterhalb der Fermi-Energie und bilden das Valenzband. In den Abbildungen 7.18 und 7.19
sind es die Zustände negativer Energie. In diesem Bereich gibt es auch die meisten Gemein-
samkeiten im wellenvektorabhängigen Verhalten der Zustände.

Abbildung 7.18: Bandstruktur von CdTe [61]

Abbildung 7.19: Bandstruktur von HgTe [61]

Die drei Materialien CdTe, HgTe und HgCdTe sind sich in ihrem Tellur-Anteil von 50 %
gleich. Das Valenzband wird an der Fermi-Energie von p-artigen Elektronen gebildet [57]. Die-
se Zustände liefert das Tellur, daher ist die Ähnlichkeit nicht verwunderlich. Erst ab einem
Kompositionsverhältnis unter 0,07 für Hg1-xCdxTe erreicht das s-artige Leitungsbandniveau
des Kations die Fermi-Energie und invertiert sein Verhalten. Es klappt unter die Fermi-Energie.
Mit der Photoelektronspektroskopie war es möglich, diese invertierte Bandstruktur direkt zu
beobachten [59].

Insofern sollte man bei Untersuchungen von HgCdTe mittels Photoemission keine großen
Veränderungen innerhalb des Valenzbandes erwarten, sondern das Auftauchen eines zusätzli-
chen Bandes zwischen den $\Gamma$8- und $\Gamma$7-Bändern (siehe Skizze 7.20). Eine Besonderheit der so-
genannten „small-gap“-Halbleiter ist das veränderte Dispersionsverhalten direkt am $\Gamma$-Punkt.
Das Kane-Modell [62] beschreibt dieses nicht-parabolische Verhalten bei Halbleitern, deren
Bandlücke kleiner als 0,35 eV ist.

Abbildung 7.20: Schematische Darstellung der Bandstruktur von
CdTe und HgTe am $\Gamma$-Punkt. - on page 54

Neben den schon beschriebenen Problemen, überhaupt ein Valenzbandmaximum zu be-
stimmen, war auch das rechnerische Anpassen der gemessenen Spektren an eine Verteilung
von Zuständen keineswegs trivial. Da die Spektren nicht aus einzelnen, separaten Peaks be-
standen, konnten die automatischen Funktionen („Fit Multi-peaks“) von Origin [25] nicht ver-
wendet werden. Jeder Versuch, mit dieser Automatik dennoch die Messergebnisse anzupassen,
führte zu wertlosen und nicht reproduzierbaren Ergebnissen.

Daher musste die anzupassende Funktion selbst geschrieben („Non-linear Curve Fit - Ad-
vanced Fitting Tool“) und manuell mit Werten versehen werden. Die automatische Anpassung
der Fitparameter mit dem Levenberg-Marquardt-Algorithmus führte wieder zu unbrauchba-
ren Ergebnissen. Die Anpassung musste zum großen Teil „von Hand“ erfolgen. Erst in der
Feinabstimmung konnten Automatiken eingesetzt werden. Allerdings war es nicht möglich,
mit den drei erwarteten Bändern eine befriedigende Anpassung der gemessenen Spektren zu
erreichen.

Abbildung 7.21: Dispersion des Punktes maximaler Intensität aus der $k_\perp$-Messung
an Probe I, eingetragen in eine Bandstrukturrechnung.

Zum Vergleich mit der Bandstrukturrechnung wurden daher nur die Ergebnisse der $k_\perp$-
Messungen von Probe I herangezogen. Man beobachtet in Abbildung 7.14 für Photonenener-
gien von 120 bis 85 eV eine Dispersion zu höheren Bindungsenergien. Im Bereich von 80 bis
65 eV sieht man hingegen eine Dispersion zu niedrigeren Bindungsenergien. Dieses Verhalten
kann mithilfe unterschiedlicher reziproker Gittervektoren in Formel (3.8) erklärt werden. Im
ersten Fall wird (cid:126)G1 = $2\pi$
a (3, 3, 1). Damit ist
die Energie von 115 eV dem $\Gamma$-Punkt zugeordnet und die Energie von 65 eV dem X-Punkt.

a (4, 4, 0) angewendet und im zweiten Fall (cid:126)G2 = $2\pi$

In den Einzelspektren der Messung in normaler Emission findet sich jeweils ein Punkt ma-
ximaler Intensität, der sehr genau bestimmt werden kann. Dieser Punkt weist außerdem eine
starke Dispersion auf. Er repräsentiert daher mit großer Wahrscheinlichkeit ein Volumenband.
Unabhängig vom der weiteren Intensitätsverteilung im gemessenen Spektrum wurde die Ener-
gie dieses Punktes bestimmt. Mit den gefundenen passenden reziproken Gittervektoren wurde
jeder dieser Punkte einem Punkt im reziproken Raum zugeordnet. In Abbildung 7.21 sind die-
se Punkte mit einer Bandstrukturrechnung überlagert. Das Intensitätsmaxima kann demnach
nicht nur einem einzigen Band zugeordnet werden.



% ********************************************* Kapitel 8: Zusammenfassung ******************************************

\chapter{Zusammenfassung}

Im Rahmen dieser Diplomarbeit bestand die Aufgabe, die elektronische Struktur einiger Pro-
ben (Abbildung 5.4) von Hg1-xCdxTe zu bestimmen. Die Zusammensetzung der Kristalle vari-
iert zwischen x=0,07 und x=0,4.

In einem ersten Schritt mussten die Proben charakterisiert werden. Die Orientierung konnte
mit der Röntgenbeugung nach Laue (Epigramm) sehr genau bestimmt werden. Eine Transmis-
sionsmessung mit dieser Methode legte deutlich den polykristallinen Charakter der Proben
dar. Jedoch ist die makroskopische Orientierung der Kristalle einheitlich. Die Untersuchungen
mit energiedispersiver Röntgenspektroskopie ergab einen noch ungeklärten Mangel an Tellur.
Doch die Daten der Zusammensetzung konnten bestätigt werden.

Die nächste Herausforderung ist die Präparation einer Oberfläche, die für Photoemission-
messungen geeignet ist. Das Verfahren des Sputterns und Annealens kann nicht verwendet
werden. Entlang der (110)-Richtung lassen sich die Kristalle hingegen sehr gut spalten. Die-
se Spaltrichtung ist allgemein von Kristallen in Zinkblendestruktur bevorzugt. Noch bessere
Ergebnisse konnten erzielt werden, wenn die Proben vor dem Spalten gekühlt wurden. Die
Qualität einer durchgeführten Spaltung wurde mit der Beugung langsamer Elektronen sowie
mit einem Rasterelektronenmikroskop überprüft und bestätigt. Weiterhin zeigte die Untersu-
chung der Probenoberflächen mit einem Atomkraftmikroskop, dass sich Fehlstellen in HgCdTe
schon bei Raumtemperatur aus dem Volumen an die Oberfläche bewegen können.

Auch die eigentliche Photoemissionsmessung war erfolgreich. Bei der Untersuchung der
energetischen Lage der Kernniveaus wurden Einflüsse der chemischen Umgebung sichtbar. So
konnte zum Beispiel eine Verdopplung der Spin-Bahn-aufgespaltenen Te4d-Niveaus gemessen
werden (Abbildung 7.7). Des weiteren ließ sich eine Verschiebung der Kernniveaus nahe dem
Valenzband in Abhängigkeit von der Zusammensetzung messen.

Die Messungen an gut präparierten (110)-Oberflächen in normaler Emission zeigten eine
gut sichtbare Dispersion, die sich eindeutig symmetrischen Punkten der Brillouin-Zone zuord-
nen lassen. In einigen Punkten stimmte der Verlauf dieser Dispersion mit den theoretischen
Rechnungen überein. Die Kernniveaus zeigten erwartungsgemäß keinerlei Dispersion. Ihre
energetische Lage ist winkelunabhängig.




% ********************************************* Anhang A ******************************************

\appendix
\chapter[Anhang]{}

\section{Messdaten EDX April 2007}

Nr. Cd
0.07
1
0.06
0.05
0.13
0.11
0.15
0.39
0.40
0.38
0.20
0.18
0.17
0.16
0.16
0.16

5

4

rel.Fehler Hg
1.00
1.02
1.05
0.99
0.99
0.93
0.67
0.65
0.69
0.89
0.90
0.90
0.90
0.89
0.91

29.6\%
40.6\%
45.5\%
17.9\%
20.5\%
16.1\%
10.4\%
9.9\%
10.1\%
15.2\%
16.3\%
19.0\%
19.7\%
19.0\%
18.4\%

rel. Fehler
18.9\%
19.6\%
19.1\%
19.3\%
19.3\%
18.5\%
21.9\%
22.2\%
21.2\%
20.3\%
20.3\%
21.7\%
20.8\%
23.1\%
22.8\%

Te
0.93
0.92
0.90
0.89
0.90
0.92
0.94
0.95
0.93
0.91
0.92
0.93
0.93
0.95
0.93

rel.Fehler Probe

I

V

II

V

V

8.2\%
8.5\%
8.8\%
8.8\%
8.7\%
7.8\%
8.4\%
7.8\%
7.9\%
8.6\%
8.6\%
8.8\%
8.8\%
8.8\%
8.9\%

x (soll)
0.07

0.1955

0.4

0.1955

0.1955

Tabelle A.1: Zusammensetzung ausgewählter fünf Proben, Ergebnisse von EDX. Es wurden je drei un-
terschiedliche Punkte der Proben untersucht. An den Proben wurde zuvor mittels Photoemission ge-
messen.

Weitere Messungen in Photoemission wurden im Juli und August 2007 vorgenommen. Bei
der Präparation ist eine Probe bereits vor der Messung gespalten. Sie ist auf der folgenden Seite
mit Probe 10 bezeichnet. Unter der Tabelle finden sich zwei SEM-Aufnahmen dieser Probe.


\section{Messdaten EDX August 2007}

Nach Beendigung der Messungen bei BESSY im August 2007 wurden auch diese Proben zur
Untersuchung mittels EDX gegeben:

rel.Fehler Probe

5

4

2

3

Nr. Cd
0.16
1
0.09
0.11
0.39
0.38
0.40
0.07
0.04
0.05
0.11
0.11
0.10
0.12
0.11
0.08
0.07
0.07
0.09
0.17
0.16
0.17
0.17
0.15
0.17
0.06
0.05
0.06
0.06
0.06
0.07

10

7

6

8

9

rel.Fehler Hg
0.95
1.00
0.96
0.64
0.65
0.66
0.98
1.08
1.07
1.00
0.97
0.99
0.96
0.94
1.03
1.05
1.03
1.00
0.93
0.91
0.90
0.90
0.94
0.91
1.02
1.05
1.04
1.05
1.02
1.00

20.7\%
20.5\%
21.2\%
8.9\%
9.6\%
9.1\%
40.9\%
54.4\%
46.0\%
23.5\%
18.8\%
25.5\%
22.1\%
22.4\%
32.5\%
38.2\%
39.5\%
27.0\%
17.9\%
19.2\%
18.8\%
17.3\%
16.7\%
16.4\%
31.4\%
46.2\%
26.3\%
35.8\%
37.4\%
39.2\%

rel. Fehler
21.9\%
17.4\%
20.8\%
19.1\%
19.9\%
20.0\%
22.0\%
20.3\%
19.0\%
20.2\%
18.5\%
20.1\%
21.0\%
21.3\%
20.6\%
19.4\%
20.0\%
22.1\%
20.5\%
21.8\%
22.5\%
19.9\%
18.4\%
17.6\%
18.3\%
19.9\%
16.8\%
19.8\%
19.6\%
20.9\%

Te
0.89
0.91
0.92
0.97
0.96
0.94
0.95
0.88
0.88
0.89
0.92
0.91
0.92
0.94
0.89
0.88
0.90
0.91
0.90
0.93
0.93
0.93
0.91
0.92
0.92
0.90
0.90
0.89
0.92
0.93

9.0\%
7.7\%
9.0\%
7.1\%
7.5\%
7.2\%
9.0\%
8.9\%
8.7\%
8.9\%
7.8\%
9.0\%
9.0\%
8.4\%
9.0\%
8.7\%
8.8\%
9.0\%
8.8\%
8.9\%
8.8\%
8.4\%
8.1\%
7.6\%
7.6\%
8.7\%
7.2\%
8.7\%
8.5\%
8.9\%

x (soll)
0.105

0.4

0.07

VI

II

I

VI

0.105

VI

0.105

I

0.07

V

0.1955

IV

0.183

I

I

0.07

0.07

Tabelle A.2: EDX-Ergebnisse an den Proben, die im Juli und August 2007 per ARPES untersucht wurden.


Two images at the bottom!!!









% \chapter{Elektronische Eigenschaften von CdTe, HgTe und Cd$_{1-X}$Hg$_{X}$Te}
% \markright{\upshape Elektronische Eigenschaften von CdTe, HgTe und Cd$_{1-X}$Hg$_{X}$Te}
% \section{Halbleiter mit schmaler Bandlücke}



\begin{thebibliography}{99}
\bibitem{lawson} W. D. Lawson, S. Nielson, E. H. Putley, and A. S. Young. Preparation and properties of HgTe and mixed crystals of HgTe-CdTe. {\itshape J. Phys. Chem. Solids} {\bfseries 9}, 325-329 (1959).
\bibitem{long} D. Long and J. L. Schmit. Mercury-cadmium telluride and closely related alloys. {\itshape Semiconductors and Semimetals,} Vol. 5, pp. 175-255, edited by R. K. Willardson and A. C. Beer, Academic Press, New York (1970).
\bibitem{finger} Gert Finger, J. Garnett, N. Bezawada, R. Dorn, L. Mehrgan, M. Meyer, A. Moorwood, J. Stegmeier, G. Woodhouse. Performance evaluation and calibration issues of large format infrared hybrid active pixel sensors used for ground- and space-based astronomy. {\itshape Nuclear Instruments and Methods in Physics Research} {\bfseries A} 565 (2006) 241-250.
\bibitem{ives} Derek Ives, Nagajara Bezawada. Large area near infra-red detectors for astronomy. {\itshape Nuclear Instruments and Methods in Physics Research} {\bfseries A} 573 (2007) 107-110.
\bibitem{gillessen} S. Gillessen, et. al. {\itshape The Messenger} 120 (2005) 26-32.
\bibitem{groves} S. H. Groves, R. N. Brown. C. R. Pidgeon. Interband Magnetoreflection and Band Structure of HgTe. {\itshape Phys. Rev.} {\bfseries 161} (1967), 779.
\bibitem{orlowski} N. Orlowski. Untersuchung der elektronischen Struktur von HgSe und HgTe mittels winkelaufgelöster Photoemission. {\itshape Diplomarbeit,} AG EES, (2000).
\bibitem{higgins} W. M. Higgins, G. N. Pultz, R. G. Roy, R. A. Lancaster, J. L. Schmit. Standard relationships in the properties of Hg\textsubscript{1-x}Cd\textsubscript{x}Te. {\itshape  Vakuum Science and Technology} {\bfseries A} 7 (1989), p 271-275.
\bibitem{preuss} E. Preuss, B. Krahn-Urban, R. Butz. {\itshape Laue Atlas. Plotted Laue Back-Reflection Patterns of the Elements, the Compounds $R X$ and $R X_{2}$.} Bertelsmann Universitätsverlag, Düsseldorf, 1974.
\bibitem{tanaka} A. Tanaka, Y. Masa, S. Seto, T. Kawasaki. Zinc and selenium co-doped CdTe substrates lattice matched to HgCdTe. {\itshape J. Cryst. Growth} vol. 94 (1989) p. 166-70.
\bibitem{gawlik} K.-U. Gawlik. Untersuchung der elektronischen Struktur von II-VI-Verbindungshalb-leitern mit direkter und inverser Photoemission. {\itshape Dissertation,} (1996).
\bibitem{ford} W. K. Ford, T. Guo, D. L. Lessor, C. B. Duke. Dynamical low-energy electron-diffraction analysis of bismuth and antimony epitaxy on GaAs(110). {\itshape Phys. Rev.} {\bfseries B} 42, 14 (1990), 8952.
\bibitem{duke} C. B. Duke. Structure and bonding of tetrahedrally coordinated compound semiconductor cleavage faces. {\itshape  J. Vac. Sci. Technol.} A 10(4), 2032 (1992).
\bibitem{hertz} H. Hertz. Über den Einfluss des ultravioletten Lichtes auf die elektrische Entladung. {\itshape Ann. Phys.} {\bfseries 31}, 983 (1887).
\bibitem{hallwachs} W. Hallwachs. Über den Einfluss des Lichtes auf elektrostatisch geladene Körper. {\itshape Ann. Phys.} {\bfseries 33}, 303 (1888).
\bibitem{einstein} A. Einstein. Über einen die Erzeugung und Verwandlung des Lichtes betreffenden heuristischen Gesichtspunkt. {\itshape Annalen der Physik} {\bfseries 17} (1905), 132-148.
\bibitem{spicer1} W. E. Spicer. Photoemissive, Photoconductive, and Optical Absorption Studies of AlkaliAntimony Compounds {\itshape Phys. Rev.} {\bfseries 112}, 117 (1958).
\bibitem{huefner} Stefan Hüfner. {\itshape Photoelectron Spectroscopy, Principles and Applications, Third Edition.} Springer-Verlag Berlin Heidelberg New York 2003.
\bibitem{heimburger} R. Heimburger. Elektronische Eigenschaften und Phasentransformation von $\beta$ - $\mathrm{MoTe}_{2}$. {\itshape Diplomarbeit,} AG EES, (2007).
\bibitem{seah} M. P. Seah, W. A. Dench. Quantitative Electron Spectroscopy of Surface: A Standard Data Base for Electron Inelastic Mean Free Paths in Solids. {\itshape Surface and Interface Analysis,} Vol. 1, Issue 1, p 2-11 (1979).
\bibitem{shirley} D. A. Shirley. High-Resolution X-Ray Photoemission Spectrum of the Valence Bands of Gold. {\itshape Phys. Rev.} {\bfseries B} 5, 4709-4714 (1972).
\bibitem{tougaard} S. Tougaard. Deconvolution of loss features from electron spectra. {\itshape Surface Science} 139 (1984) pp. 208-218.
\bibitem{kauert} M. Kauert. Vanadiumoxide. Herstellung, Charakterisierung und elektronische Struktur. {\itshape Diplomarbeit,} AG EES, (2007).
\bibitem{savitzky} A. Savitzky, Marcel J.E. Golay. Smoothing and Differentiation of Data by Simplified Least Squares Procedures. {\itshape Analytical Chemistry,} 36: 1627-1639 (1964). doi:10.1021/ac60214a047
\bibitem{origin} Softwarepaket: Origin 7.5 SR5, \href{http://www.OriginLab.com}{http://www.OriginLab.com} (2004)
\bibitem{plake} T. Plake. Aufbau und Inbetriebnahme des Photoemissionsexperimentes HIRE-PES: Charakterisierung und erste Untersuchungen an $\mathrm{Bi}_{2} \mathrm{Sr}_{2} \mathrm{CuO}_{6}$-Hochtemperatursupraleitern. {\itshape Diplomarbeit,} AG EES, (1998).
\bibitem{janowitz} C. Janowitz, R. Müller, T. Plake, Th. Böger, R. Manzke. New high-resolution photoemission station for synchrotron radiation at BESSY. {\itshape Journal of Electron Spectroscopy and Related Phenomena} 105, 43-49 (1999).
\bibitem{purcell} E. Purcell. {\itshape Phys. Rev.} 54 (1938) 818.
\bibitem{mante} G. Mante. {\itshape Doktorarbeit,} Universität Kiel (1992).
\bibitem{omicron}  {\itshape Instruction Manual VUV Discharge Lamp HIS 13.} Omicron Vakuumphysik GmbH (\href{http://www.omicron.de}{www.omicron.de}).
\bibitem{pt100} Platin-Widerstandsthermometer Pt100 nach IEC 751 / DIN EN 60751.
\bibitem{ar65view}  {\itshape AR65view.} Java-Software zur Analyse und Manipulation der Messdaten von AR65 und WESPHOA. \href{http://sourceforge.net/projects/ar65view/}{sourceforge.net/projects/ar65view/}.
\bibitem{martins} M. Martins, G. Kaindl, N. Schwentner. Design of the high-resolution BUS XUV-beamline for BESSY II. {\itshape Journal of Electron Spectroscopy and Related Phenomena} 101-103, 965-969 (1999).
\bibitem{orlowski2} N. Orlowski, J. Augustin, Z. Golacki, C. Janowitz, R. Manzke. Direct evidence for the inverted band structure of HgTe. {\itshape Phys. Rev. B, Rapid Communications,} 61, R5058-R5061, (2000).
\bibitem{brice} J. C. Brice. Properties of Mercury Cadmium Telluride. {\itshape EMIS Datareviews Series} No 3. INSPEC, IEE, p. 3 (1994).
\bibitem{hansen} G. L. Hansen, J. L. Schmit, and T. N. Casselman. Energy gap versus alloy composition and temperature in Hg\textsubscript{1-X}Cd\textsubscript{X}Te. {\itshape J. Appl. Phys.}  {\bfseries 53}, 7099-7101 (1982).
\bibitem{norton} P. Norton. HgCdTe infrared detectors. {\itshape Opto-Electronics Review}  {\bfseries 10}(3), 159-174 (2002).
\bibitem{vere} A. W. Vere, B. W. Straughan, D. J. Williams, N. Shaw, A. Royle, J. S. Gough, J. B. \mbox{Mullin}. Growth of  Cd\textsubscript{X}Hg\textsubscript{1-X}Te by a pressurised cast-recrystallise-anneal technique, {\itshape J. Cryst. Growth} vol. 59 (1982) p. 121-129.
\bibitem{colombo} L. Colombo, A. J. Syllaios, r. W. Perlaky, M. J. Brau. Growth of large diameter (Hg,Cd)Te crystals by incremental quenching. {\itshape J. Vac. Sci. Technol.} A 3(1), 100-104 (1985).
\bibitem{wang} Yue Wang, Quanbao Li, Qinglin Han, Qinghua Ma, Bingwen Song, Wanqi Jie, Yaohe Zhou, Yuko Inatomi. A two-stage technique for single crystal growth of HgCdTe using a pressurized Bridgman method. {\itshape Journal of Crystal Growth} 263 (2004) 273-282.
\bibitem{colombo2} L. Colombo, R. R. Chang, C. J. Chang, B. A. Baird. Growth of Hg-based alloys by the traveling heater method. {\itshape J. Vac. Sci. Technol.} A 6(4), 2795-2799 (1988).
\bibitem{harman} T. C. Harman. Optically pumped LPE-grown Hg\textsubscript{1-X}Cd\textsubscript{X}Te lasers. {\itshape  J. Electron. Mater.} vol. 8 (1979) p. 191-200.
\bibitem{koo} B. H. Koo, Y. Ishikawa, J.F. Wang, M. Isshiki. Crowth of Hg\textsubscript{1-x}(Cd\textsubscript{1-y}Zn\textsubscript{y})\textsubscript{x}Te epilayers on (100) Cd\textsubscript{1-y}Zn\textsubscript{y}Te/GaAs substrates by ISOVPE. {\itshape Materials Science and Engineering,} B66, 70-74, (1999).
\bibitem{wang2} Yue Wang, Quanbao Li, Qinglin Han, Qinghua Ma, Rongbin Ji, Bingwen Song, Wanqi Jie, Yaohe Zhou, Yuko Inatomi. Growth and properties of 40 mm diameter Hg\textsubscript{1-X}Cd\textsubscript{X}Te using the two-stage Pressurized Bridgman method. {\itshape Journal of Crystal Growth} 263 (2004) 54-62.
\bibitem{nikiforov} Vladimir Nikiforov (\ru{Никифоров Владимир Николаевиш}), MSU, Private communication.
\bibitem{capper} P. Capper. {\itshape Properties of narrow gap Cadmium-based compounds.} INSPEC, the Institution of Electrical Engineers (1994).
\bibitem{williams} D. J. Williams, A. W. Vere. Sub-grain boundaries in $\mathrm{Cd}_{\mathrm{X}} \mathrm{Hg}_{1-\mathrm{X}} \mathrm{Te}$ and CdTe. {\itshape J. Cryst. Growth} vol. 83 (1987) p. 341-352.
\bibitem{spicer2} W. E. Spicer, J. A. Silberman, and I. Lindau. Band gap variation and lattice, surface, and interface ``instabilities'' in Hg\textsubscript{1-X}Cd\textsubscript{X}Te and related compounds. {\itshape J. Vac. Sci. Technol. A} 1(3) pp. 1735-1743 (1983).
\bibitem{morgen} P. Morgen, J. Silberman, I. Landau, W. E. Spicer, J. A. Wilson. Stability of an atomically clean Hg\textsubscript{1-X}Cd\textsubscript{X}Te surface in vacuum and under $\mathrm{O}_{2}$ exposure. {\itshape Journal of Crystal Growth} 56, 493-497 (1982).
\bibitem{mitdank} R. Mitdank. Mikrosondenanalyse. Beispiele für Untersuchungen an der Elektronenstrahlmikrosonde. \href{http://htc.physik.hu-berlin.de/~mitdank/semq.htm}{http://htc.physik.hu-berlin.de/\textasciitilde mitdank/semq.htm}.
\bibitem{berding} M. A. Berding, A. Sher, A-B Chen and R. Patrick. Vacancies and surface segregation in HgCdTe and HgZnTe. {\itshape Semicond. Sci. Technol.} vol.  {\bfseries 5} pp. S86-S89 (1990).
\bibitem{ribbat} C. Ribbat. Aufbau einer MBE-Anlage zur Herstellung von MolybdändichalkogenidSchichtkristallen für photovoltaische Anwendungen. {\itshape Diplomarbeit,} AG EES, (1998).
\bibitem{silberman} J. A. Silberman, P. Morgen, I. Lindau, and W. E. Spicer. UPS study of the electronic structure of Hg\textsubscript{1-X}Cd\textsubscript{X}Te: Breakdown of the virtual crystal approximation. {\itshape J. Vac. Sci. Technol.} 21(1) pp. 142-145 (1982).
\bibitem{shih} C. K. Shih and W. E. Spicer. Photoemission studies of core level shifts in HgCdTe, CdMnTe, and HgZnTe. {\itshape J. Vac. Sci. Technol. A,} 5(5) pp. 3031-3034 (1987).
\bibitem{shih2} C. K. Shih, J. A. Silberman, A. K. Wahi, G. P. Carey, I. Lindau, and W. E. Spicer. Angle resolved photoemission study of the alloy scattering effect in Hg\textsubscript{1-X}Cd\textsubscript{X}Te. {\itshape  J. Vac. Sci. Technol. A,} 5(5) pp. 3026-3030 (1987).
\bibitem{puettner} Dr. Ralph Püttner, FU Berlin. Beamlinebetreuer BUS. Private communication. 
\bibitem{moon} Chang-Youn Moon, Su-Huai Wei. Band gap of Hg chalcogenides: Symmetry-reduction-induced band-gap opening of materials with inverted band structures. {\itshape Phys. Rev.} {\bfseries B} 74:045205 (2006).
\bibitem{moulder} J. F. Moulder, W. F. Stickle, P. E. Sobol, K. D. Bomben. {\itshape Handbook of X-ray Photoelectron Spectroscopy.} Perkin-Elmer Corporation, 1992.
\bibitem{janowitz2} C. Janowitz, N. Orlowski, R. Manzke, Z. Golacki. On the band structure of HgTe and HgSe - view from photoemission. {\itshape Journal of Alloys and Compounds,} 328, 84-89, (2001).
\bibitem{fleszar} A. Fleszar, W. Hanke. Electronic structure of $\mathrm{II}^{\mathrm{B}}$-VI semiconductors in the GW approximation. {\itshape Phys. Rev.} {\bfseries B} 71:045207 (2005).
\bibitem{chen} A-B Chen, Y-M Lai-Hsu, S. Krishnamurthy, M. A. Berding, and A. Sher. Band structures of HgCdTe and HgZnTe alloys and superlattices. {\itshape Semicond. Sci. Technol.} vol. {\bfseries 5} pp. S100-S102 (1990). doi:\href{http://dx.doi.org/10.1088/0268-1242/5/3S/021}{10.1088/0268-1242/5/3S/021}
\bibitem{kane} E. O. Kane. Band structure of indium antimonide. {\itshape J. Phys. Chem. Solids} 1 (4), 249-261 (1957)
\bibitem{janowitz3} C. Janowitz, L. Kipp, R. Manzke. Experimental surface band structure of CdTe(110). {\itshape Surface Science} 231 (1990) 25-31.
\end{thebibliography}

\clearpage
\pagestyle{empty}

\chapter*{Danksagung}
\thispagestyle{empty}

Die Mitarbeit in einem wissenschaftlichen Team stellt den krönenden Abschluss meines Studiums dar. Daher möchte ich mich bei der Arbeitsgruppe EES für die freundliche Aufnahme und gute Zusammenarbeit bedanken. Neben ARPES konnte ich so viele andere Geheimnisse näher kennen lernen, die sich hinter Abkürzungen wie STM, XUV, LEED, EDX, SEM, AFM, EPR und BESSY verbergen. Mein Dank gilt insbesondere:

\begin{itemize}
  \item Prof. Manzke für die Bereitstellung des interessanten Themas und die Möglichkeit, in seiner Arbeitsgruppe ein so umfassendes Forschungsthema eigenständig bearbeiten zu können.
  \item Dr. Krapf für die Unterstützung, den dreimonatigen Forschungsaufenthalt in Moskau zu organisieren.
  \item Hr. Sölle für die Präparation und Zerteilung der Proben.
  \item Der Werkstatt (insbesondere Hr. Rausche, Hr. Fahnauer und Frau Rosinska) für die gute Zusammenarbeit, die Hilfe beim Herstellen zahlloser Kleinteile und die Unterstützung bei der Reparatur einiger Gerätschaften.
  \item \ru{Телегина Инна Васильевна} für die umfassende Laue-Untersuchung meiner Kristalle.
  \item \ru{Никифоров Владимир Николаевич} für die Betreuung während des Moskauer Forschungsaufenthaltes.
  \item den Korrekturlesern Gabi, Rahela, Susan und Horst für die große Mühe, all die kleinen Fehler zu entdecken, die sich eingeschlichen hatten. Danke auch für die guten Vorschläge zu besseren Formulierungen.
  \item auch meiner Familie, die mit Geduld mein Studium begleitet hat und ohne deren Unterstützung das alles hier nicht möglich gewesen wäre.
\end{itemize}


\chapter*{Erklärung}
\thispagestyle{empty}

Hiermit versichere ich, die vorliegende Arbeit ohne unerlaubte fremde Hilfe angefertigt zu haben. Es sind keine anderen als die angegebenen Quellen und Hilfsmittel benutzt worden. Ich gestatte Einsichtnahme in diese Diplomarbeit in der Fachbereichsbibliothek.
\vspace{3cm}

\begin{adjustwidth}{1cm}{1cm}
Berlin, Januar 2008 \hfill Matthias Kreier
\end{adjustwidth}

\end{document}
