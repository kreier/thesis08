Elektronische Eigenschaften von CdXHg1-XTe mit
0.07\textless x\textless0.4

Diplomarbeit

H UMBOLDT-U NIVERSITÄT ZU B ERLIN I NSTITUT FÜR P HYSIK AG E
LEKTRONISCHE E IGENSCHAFTEN UND S UPRALEITUNG

eingereicht von

Matthias Kreier geb. am 3.10.1975 in Nauen

Berlin, 24. 1. 2008

Inhaltsverzeichnis 1

Einleitung

1

2

Eigenschaften von II-VI Halbleitern 2.1 Kristallstruktur . . . . . . . .
. . . . . . . . . . . . . . . . . . . . . . . . . . . . . . 2.2
Volumen-Brillouin-Zone . . . . . . . . . . . . . . . . . . . . . . . . .
. . . . . . . . 2.3 Oberflächeneigenschaften . . . . . . . . . . . . . .
. . . . . . . . . . . . . . . . . .

3 3 4 6

3

Theoretische Grundlagen der Photoemission 3.1 Messprinzip der
Photoelektronspektroskopie . . . . . . . . . . . . . . . . . . . . . 3.2
Das Drei-Stufen-Modell . . . . . . . . . . . . . . . . . . . . . . . . .
. . . . . . . . 3.3 Auswertung der Messdaten und Spektren . . . . . . .
. . . . . . . . . . . . . . . . 3.3.1 Konstanter Untergrund . . . . . .
. . . . . . . . . . . . . . . . . . . . . . . 3.3.2 Subtraktion des
inelastischen Untergrundes . . . . . . . . . . . . . . . . . 3.3.3
Glättung der Spektren nach Savitzky-Golay . . . . . . . . . . . . . . .
. . 3.4 Trennung von Oberflächen und Volumenbandstruktur . . . . . . . .
. . . . . . .

7 7 8 13 13 14 15 15

4

Experimentelles 4.1 Die Photoemissionsanlage AR65 . . . . . . . . . . .
. . . . . . . . . . . . . . . . . 4.2 Heliumlampe Focus HIS 13 . . . . .
. . . . . . . . . . . . . . . . . . . . . . . . . . 4.3 Automatische
Stickstoff-Nachfüllanlage . . . . . . . . . . . . . . . . . . . . . . .
. 4.4 BUS-Beamline bei BESSY . . . . . . . . . . . . . . . . . . . . . .
. . . . . . . . . . . 4.5 Röntgenuntersuchung mittels Laue . . . . . . .
. . . . . . . . . . . . . . . . . . . .

17 17 19 20 22 23

5

Das Material HgCdTe und Charakterisierung der Proben 5.1 Eigenschaften .
. . . . . . . . . . . . . . . . . . . . . . . . . . . . . . . . . . . .
. . 5.2 Herstellungsverfahren . . . . . . . . . . . . . . . . . . . . .
. . . . . . . . . . . . . 5.3 Übersicht der untersuchten Proben . . . .
. . . . . . . . . . . . . . . . . . . . . . . 5.4 Laue-Aufnahmen zur
Kristallqualität . . . . . . . . . . . . . . . . . . . . . . . . . 5.5
Bestimmung der Zusammensetzung mittels energiedispersiver
Röntgenspektroskopie . . . . . . . . . . . . . . . . . . . . . . . . . .
. . . . . . . . . . . . . . . . . . 5.6 Überprüfung der Kristalle mit
dem Atomkraftmikroskop . . . . . . . . . . . . . .

25 25 27 29 30

Präparation der (110) Oberfläche 6.1 Sputtern und Annealen . . . . . . .
. . . . . . . . . . . . . . . . . . . . . . . . . . . 6.2 Spalten der
Proben im Vakuum . . . . . . . . . . . . . . . . . . . . . . . . . . . .
. 6.2.1 Konstruktion einer Spaltkammer . . . . . . . . . . . . . . . . .
. . . . . . . 6.2.2 Spaltmechanismus im Probenhalter . . . . . . . . . .
. . . . . . . . . . . . 6.3 Aufnahmen der Spaltungen mit dem
Rasterelektronenmikroskop . . . . . . . . . 6.4 Prüfung der
Oberflächenqualität durch die Beugung langsamer Elektronen . . .

34 34 36 36 37 38 39

6

I

32 32

II

INHALTSVERZEICHNIS

7

Ergebnisse der Photoemission 7.1 Energie der gemessenen Zustände . . . .
. . . . . . . . . . . . . . . . . . . . . . . 7.2 Allgemeine
Charakteristika . . . . . . . . . . . . . . . . . . . . . . . . . . . .
. . . 7.3 Energie der Kernniveaus . . . . . . . . . . . . . . . . . . .
. . . . . . . . . . . . . . 7.3.1 Tellur 4d . . . . . . . . . . . . . .
. . . . . . . . . . . . . . . . . . . . . . . . 7.3.2 Quecksilber 5d und
Cadmium 4d . . . . . . . . . . . . . . . . . . . . . . . 7.4 Das
Valenzband - Messung in normaler Emission . . . . . . . . . . . . . . .
. . . 7.5 Winkelaufgelöste Photoemissionsmessungen . . . . . . . . . . .
. . . . . . . . . . 7.6 Bandstruktur und Theorie . . . . . . . . . . . .
. . . . . . . . . . . . . . . . . . . .

41 41 44 46 46 46 48 52 53

8

Zusammenfassung

56

A Anhang A.1 Messdaten EDX April 2007 . . . . . . . . . . . . . . . . .
. . . . . . . . . . . . . . . A.2 Messdaten EDX August 2007 . . . . . .
. . . . . . . . . . . . . . . . . . . . . . . .

57 57 58

Kapitel 1

Einleitung Aufgrund militärischer Anforderungen war man in den 40er und
50er Jahren des 20. Jahrhunderts intensiv auf der Suche nach einem
direkten intrinsischen Halbleiter für den langwelligen
Infrarot-Wellenlängenbereich (LWIR, 8-14 µm). Ein Ergebnis dieser
Bemühungen war die Synthese der ternären Legierung HgCdTe im Jahre 1958
von der Lawson Forschungsgruppe {[}1{]} am Royal Radar Establishment in
England. Die Bedeutung dieser Arbeit wurde schon früh erkannt. Das
führte zu einer intensiven Entwicklung in zahlreichen Ländern wie
England, Frankreich, Polen, Deutschland, der Sowjetunion sowie den USA
{[}2{]}. Doch wurde über die Entwicklung in den ersten Jahren wenig
veröffentlicht. Die Arbeiten in den USA zum Beispiel unterlagen der
Geheimhaltung bis in die späten 60er Jahre. Die ersten Photodetektoren
wurden in den USA bereits 1964 von Texas Instruments hergestellt.

Abbildung 1.1: HgCdTe in Infrarotkameras: Das Rockwell 2x2 2Kx2K
Infrarot-Array Hawaii-2RG {[}3{]}.

Verschiedene Eigenschaften machen HgCdTe zum idealen Material als IR
Detektor. Einige von diesen sind eine einstellbare Bandlücke von 0,7 to
25 µm, die direkte Bandlücke mit hohem Absorptionskoeffizienten,
moderate Dielektrizitätskonstante und Brechungsindex sowie einen
geringen thermischen Ausdehnungskoeffizienten. Weiterhin gibt es eine
Auswahl passender Substrate für epitaktisches Wachstum über einen großen
Wellenlängenbereich (z. B. Cd0.96 Zn0.04 Te). Aufgrund seiner
Eigenschaften ist HgCdTe im Bereich der IR-Detektoren das Material der
Wahl {[}3{]}, {[}4{]}. In Abbildung 1.1 ist das 2x2 2Kx2K Infrarotarray
Hawaii-2RG (16 Megapixel) zu sehen. Es wurde für das 6,5 m James Webb
Space Telescope (JWST), den Nachfolger des Hubble-Teleskopes,
entwickelt. Es wird unter anderem im Very Large Telescope (VLT) der ESO

1

2

Einleitung

im Experiment SINFONI verwendet {[}5{]}. Das Material wird schon seit
langem bei Weltraummissionen für die Infrarotastronomie eingesetzt. Zu
erwähnen wäre hier das Experiment NICMOS (Near Infrared Camera and
Multi-Object Spectrometer), welches aus Kameras und Spektrometern für
das nahe Infrarot bis 2,5 µm Wellenlänge am Hubble-Weltraumteleskop
besteht und seit 1997 im Einsatz ist. Seine besonderen Eigenschaften
erhält die ternäre Verbindung HgCdTe durch eine Mischung der
Eigenschaften seiner beiden Bestandteile CdTe und HgTe. Beide
Materialien lassen sich in beliebigen Verhältnissen mischen. Das
Verhältnis von Cd zu Hg wird mit 0 \textless{} x \textless{} 1 in Hg1-X
CdX Te angegeben. Bei CdTe handelt es sich um einen klassischen
Halbleiter mit einer direkten Bandlücke von 1,6 eV. Die Legierung HgTe
hingegen ist nicht einfach einzuordnen. Sie besitzt metallische
Eigenschaften, weshalb einige sie als Semimetall bezeichnen.
Andererseits zeigt sie Halbleitereigenschaften, die allerdings nur
richtig erklärt werden können, wenn man von einer negativen Bandlücke
ausgeht. Gemäß der Definition der Bandlücke als Energiedifferenz
zwischen Γ8 und Γ6 beträgt die Bandlücke daher -0.283 eV. Dies ist in
Übereinstimmung mit magnetooptischen Transportmessungen {[}6{]}. Ziel
dieser Diplomarbeit ist die Untersuchung der elektronischen Struktur von
Cdx Hg1-x Te mit x-Werten von 0,07 bis 0,4. Uns interessiert, wie sich
die elektronische Bandstruktur beim Übergang vom Halbleiter zum
'zero-gap'-Halbleiter zum Halbleiter mit negativer Bandlücke verhält.
Als verheißungsvoll erwies sich die Arbeit von N. Orlowski {[}7{]} an
HgTe, in der das unter das VBM geklappte Γ6 -Band spektroskopisch
aufgelöst werden konnte. Seine Ergebnisse von 2000 motivierte zweifellos
zur Ausschreibung der folgenden Diplomarbeit im Jahre 2004:
Elektronische Eigenschaften der II-VI-Halbleiterverbindungen Cd1-x Hgx
Te und Pb1-x Snx Te Die in der Arbeit zu untersuchenden ternären
Halbleiterverbindungen erlauben in Abhängigkeit von der Zusammensetzung
eine weite Variation der fundamentalen Bandlücke, die nicht nur zu Null
sondern sogar auch negativ gemacht werden kann (wird bei Nachfrage näher
erläutert). Dabei wird das prinzipielle Auftreten einer negativen
Bandlücke bei verschiedenen Halbleitern heute noch kontrovers diskutiert
und eine direkte Untersuchungsmethode wie die winkelaufgelöste
Photoemission könnte hier die Klärung bringen. Die Aufgabe der/des
Diplomandin/en besteht u.a. darin, die in einem kooperierenden Moskauer
Institut hergestellten Einkristalle zu charakterisieren (z.B.
SEM/Röntgenemission, LaueBeugung, LEED) und anschließend mit
winkelaufgelöster Photoemission und evtl. auch inverser Photoemission
die experimentelle Bandstruktur zu bestimmen und die Ergebnisse mit
Rechnungen verschiedener Modelle zu vergleichen und zu diskutieren. Die
Messungen sollen sowohl im Berliner Institut als auch teilweise mit
Synchrotronstrahlung (BESSY in Berlin oder HASYLAB in Hamburg)
durchgeführt werden.

Kapitel 2

Eigenschaften von II-VI Halbleitern 2.1

Kristallstruktur

Die Struktur von kristallinen Festkörpern wird durch das Gitter und die
Basis beschrieben. Das Gitter ist eine dreidimensionale Anordnung von
Punkten, deren kleinste Einheit die Elementarzelle ist. Es wird durch
die entsprechenden Gitterkonstanten sowie primitive Vektoren
beschrieben. Die Basis definiert die Anordnung der Atome in der
Elementarzelle. Zu jeder Struktur gibt es ein entsprechendes reziprokes
Gitter im reziproken Raum. In diesem wird die elektronische Struktur
beschrieben. Die Wigner-Seitz-Zelle des reziproken Gitters heißt erste
Brillouin-Zone.

Abbildung 2.1: Zinkblendestruktur von HgCdTe. Die gelben Punkte
repräsentieren das Tellur. An den grauen Plätzen befindet sich Cadmium
mit einem Anteil von „x`` oder Quecksilber mit einem Anteil „1-x``.

Die räumliche Struktur von HgCdTe wird als „Zinkblende-Struktur``
bezeichnet (Abbildung 2.1). Diese Struktur wird durch zwei
kubisch-flächenzentrierte (fcc) Elementarzellen gebildet, die um ein
Viertel ihrer Raumdiagonalen gegeneinander verschoben sind. Hier bildet
ein Tellur-Atom die Basis des ersten Gitters. Die Basis des zweiten
Gitters wird zu x aus Cadmiumatomen gebildet und zu 1 − x aus Atomen von
Quecksilber.

3

4

Eigenschaften von II-VI Halbleitern

Tabelle 2.1: Entfernungen der hochsymmetrischen Punkte vom Zentrum der
Volumen-Brillouin-Zone des fcc-Gitters. Die formalen Werte für df sind
angegeben, um aus einer beliebigen Gitterkonstante a die Entfernungen
berechnen zu können.

d \textbar ΓΛL\textbar{} \textbar Γ∆X\textbar{} \textbar ΓΣK\textbar{}
\textbar ΓW\textbar{} \textbar ΓU\textbar{} \textbar ΓK X \textbar{}

d(HgTe) Å−1 0.844 0.974 1.034 1.090 1.034 1.379

df π/a √ 3 2√ 3/ √ 2 5 √ 3/√ 2 2 2

d(Hg0.8 Cd0.2 Te) Å−1 0.843 0.973 1.033 1.088 1.033 1.377

d(CdTe) Å−1 0.839 0.968 1.027 1.083 1.027 1.370

In dieser Kristallstruktur ist jedes Atom tetraedrisch von vier nächsten
Nachbarn der anderen Atomsorte umgeben. Die Ursache ist die
Hybridisierung der s- und p-Orbitale der Valenzelektronen zu sp3
-Hybridorbitalen. Sie schließen einen Bindungswinkel von 109,5◦ ein. Die
Gitterkonstanten von HgTe und CdTe unterscheiden sich um weniger als
0.7\% voneinander. Für die ternären Verbindung Hg1-x Cdx Te folgt sie
entsprechend dem Kompositionsparameter x der empirischen Formel a = 6,
4614 + 0, 00084x + 0, 0168x2 − 0, 0057x3 {[}8{]} . Substanz HgTe Hg0.8
Cd0.2 Te CdTe

a/Å 6,445 {[}9{]} 6,464 {[}10{]} 6,488 {[}9{]}

Das Kristallgitter wird durch folgende primitive Vektoren beschrieben:

\textasciitilde a1 =

a (1, 1, 0) , 2

\textasciitilde a2 =

a (0, 1, 1) , 2

\textasciitilde a3 =

a (1, 0, 1). 2

(2.1)

Gemäß Definition ergeben sich daraus die zugehörigen reziproken
Gittervektoren:

2π b\textasciitilde1 = (1, 1, −1) , a

2.2

2π b\textasciitilde2 = (−1, 1, 1) , a

2π b\textasciitilde3 = (1, −1, 1). a

(2.2)

Volumen-Brillouin-Zone

Aus diesen reziproken Gittervektoren resultiert ein reziprokes Gitter,
das kubisch raumzentriert (bcc, body centered cubic) ist. Die erste
Brillouin-Zone ist das in Abbildung 2.2 dargestellte abgestumpfte
Oktaeder. Die Abbildung zeigt einen Schnitt durch die
Volumen-BrillouinZone sowie die Oberflächen-Brillouin-Zonen, die den
Flächen (110) und (001) entspechen. Ebenfalls eingezeichnet sind die
hochsymmetrischen Punkte Γ, X, L, K, W und U. Es finden sich auch die
drei hochsymmetrischen Richtungen ∆, Σ und Λ. Sie entsprechen den
Richtungen {[}001{]},{[}110{]} und {[}111{]}. Die jeweiligen Abstände
können in Tabelle 2.1 abgelesen werden.

Volumen-Brillouin-Zone

Abbildung 2.2: Volumen-Brillouin-Zone des fcc-Gitters sowie
Oberflächen-BrillouinZonen der idealen (001) und (110) Oberflächen.
Einige hochsymmetrische Punkte sind eingezeichnet. Die Richtungen ∆, Σ
und Λ entsprechen jeweils der Richtung {[}001{]}, {[}110{]} und
{[}111{]}.

5

6

2.3

Eigenschaften von II-VI Halbleitern

Oberflächeneigenschaften

An der Oberfläche kommt es zu einem Bruch der Translationssymmetrie des
Volumenkristalls. Man kann daher die Oberfläche als eine Störung
auffassen; das Kristall nimmt einen Zustand minimaler Energie an. In der
idealen Oberfläche besetzen die Oberflächenatome eine wohldefinierte
Gitterebene des Volumenkristalls, ihre Periodizität wäre somit
festgelegt. Dieser Zustand ist jedoch äußerst selten der Fall. Die
einfachste Abweichung von der idealen Oberfläche wird als Relaxation
bezeichnet. Hierbei treten einheitliche Verschiebung der obersten oder
der oberen Lagen gegenüber dem Volumen auf. Von einer Rekonstruktion
spricht man, wenn die Atome der obersten Lage periodisch gegeneinander
verschoben sind und eine Überstruktur bilden {[}11{]}. Die natürliche
Spaltfläche der II-VI-Halbleiter (wie auch HgCdTe) ist die (110)-Fläche.
Durch ein Spalten der Probe im Vakuum lässt sich diese Oberfläche am
einfachsten generieren. Alle Messungen dieser Arbeit wurden daher an
dieser Oberfläche durchgeführt. Andere Oberflächen sind weitaus
schwieriger herzustellen, wie zum Beispiel die (001)-Oberfläche {[}7{]}.

Abbildung 2.3: Geometrie der (110)-Oberfläche der Zinkblende-Struktur
{[}12{]} im Querschnitt und in √ der Draufsicht. Die Parameter der
Oberfläche sind ax = a/ 2 und ay = a.

Die (110)-Oberfläche enthält ebenso viele Anionen wie Kationen und ist
daher energetisch begünstigt. Es handelt sich um eine unpolare
Oberfläche. Bei der Spaltung relaxiert die (110)Oberfläche. Die Anionen
entfernen sich von der Oberfläche, während sich die Kationen auf das
Kristall zu bewegen. Für diese Oberfläche wurde ein universales Modell
entwickelt, das unabhängig von der Substanz gültig ist {[}13{]}. Hierbei
bleiben die Bindungslängen nahezu erhalten, die Symmetrie parallel zur
Oberfläche wird nicht geändert. An der Oberfläche wird die
Anion-Kation-Bindung aufgebrochen. Bei der Relaxation kommt es zu einem
Ladungstransfer vom Kation-'dangling bond' zum Anion-'dangling-bond'.
Das 'dangling-bond' am Kation wird vollständig geleert und am Anion
entsprechend vollständig besetzt. Jedes Oberflächenatom besitzt nur noch
drei der ursprünglich vier Nachbarn. So hat jedes Anion (Kation) der
Oberfläche zwei Bindungen zu einem Oberflächen-Kation (Anion), eine zu
einem Kation (Anion) im Volumen gerichtete Bindung sowie eine ins Vakuum
gerichtete freie Valenz ('dangling bond'). Es kommt daher zu einer
Dehybridisierung der sp3 -Orbitale. Die Anionen nehmen eine s2 p3
Koordination an, die Kationen erhalten eine planare sp2 Koordination.
Das vom Anion-'dangling-bond'-abgeleitete Band senkt sich energetisch
ab. Es verschiebt sich zum oberen Rand des Volumen-Valenzbandes.

Kapitel 3

Theoretische Grundlagen der Photoemission Um geeignete Materialien in
Bauelementen wie Solarzellen, Detektoren oder integrierten Schaltkreisen
verwenden zu können, muss ihre elektronische und strukturelle
Charakteristik bekannt sein. Dazu gehören ebenfalls die
Transporteigenschaften des Materials. Die Bestimmung der elektronischen
Bandstruktur ist hierzu eines der wichtigsten Hilfsmittel. Eine
leistungsfähige Methode, die direkte Zustandsdichte und die
impulsaufgelöste Energiebandstruktur zu bestimmen, ist die
Photoelektronspektroskopie. Im Rahmen dieser Diplomarbeit findet
insbesondere die winkelaufgelöste Photoelektronspektroskopie oder ARPES1
Anwendung. Im Folgenden werden ihre physikalischen Grundlagen erläutert.

3.1

Messprinzip der Photoelektronspektroskopie

Die Photoelektronspektroskopie basiert auf dem so genannten äußeren
Photoeffekt. Dieses physikalische Phänomen wurde bereits 1887 von H.
Hertz {[}14{]} und 1888 von W. Hallwachs {[}15{]} entdeckt und
untersucht. Die richtige Deutung erfolgte durch A. Einstein im Jahre
1905 ({[}16{]}, Nobelpreis 1921). Der Effekt wird durch die folgende
Formel beschrieben: max = hν − Φ. Ekin

(3.1)

Sie gibt die maximale kinetische Energie Emax kin an, mit der Elektronen
bei Anregung mit Strahlung der Energie hν aus einem Metall austreten.
Hier ist h das Plancksche Wirkungsquantum, ν die Frequenz des
ionisierenden Photons und Φ die Austrittsarbeit des angeregten
Materials. Abhängig von der Energie der anregenden Photonen spricht man
in der Photoelektronenspektroskopie von UPS2 oder XPS3 , wobei UPS
Photonenenergien im UV-Bereich (10 bis 100 eV) bezeichnet und XPS
Photonen im Röntgenbereich (\textgreater{} 1000 eV). Auf Grund ihrer
höheren Energie werden durch XPS auch Rumpfelektronen angeregt. In
Abhängigkeit der chemischen Umgebung zeigen XPS-Spektren Unterschiede in
den Bindungsenergien eines Rumpf1

ARPES - Angle Resoved Photo Electron Spectroscopy UPS - Ultraviolet
Photoemission Spectroscopy, UV-Photoemission 3 XPS - X-Ray Photoemission
Spectroscopy, Röntgenphotoemission 2

7

8

Theoretische Grundlagen der Photoemission

Abbildung 3.1: Messprinzip der winkelaufgelösten Photoemission. Als
Photonenquelle kann ein Synchrotron dienen. Als Energieanalysator ist
der Schnitt durch einen aktuellen Scienta dargestellt.

elektrons. In vielen Fällen kann aus der Form der Spektren Aufschluss
über den Valenzzustand eines Elementes gegeben werden. Diese chemische
Analyse ist unter der Bezeichnung ECSA (Electron Spectroscopy for
Chemical Analysis) bekannt. Demgegenüber werden in UPS nur
Valenzelektronen und Elektronen aus hochgelegten Rumpfniveaus angeregt.
UPS eignet sich daher zur Untersuchung der Valenzbandstruktur von
Halbleitern. In ARPES werden die emittierten Elektronen winkel- und
energieaufgelöst detektiert. Das Prinzip wird in Abbildung 3.1
verdeutlicht.

3.2

Das Drei-Stufen-Modell

Es gibt eine Vielzahl erfolgreicher theoretischer Beschreibungen des
Photoemissionsprozesses. In der Praxis hat sich das anschauliche
Drei-Stufen-Modell {[}17{]} durchgesetzt. Im Einteilchenbild gliedert
sich der Prozess damit in die folgenden drei unabhängigen Schritte
{[}18{]}: • Absorption des Photons und Anregung eines Elektrons aus
einem Anfangszustand im Valenzband in einen Endzustand im Leitungsband •
Transport des angeregten Elektrons zur Oberfläche des Festkörpers •
Austritt des Elektrons durch die Oberfläche ins Vakuum Die einzelnen
Schritte dieses Modells werden im Folgenden einzeln beschrieben.

Schritt 1: Anregung des Elektrons Der erste Schritt beschreibt die
Photoionisation. Lokal wird ein Photon absorbiert und ein Elektron
angeregt. Dieser Prozess lässt sich mit der zeitabhängigen
Störungstheorie erklären {[}19{]}.

Das Drei-Stufen-Modell

9

Die Übergangsrate Tf →i für ein Elektron von einem Anfangszustand
\textbar Φi i mit einer Anfangsenergie Ei in einen Endzustand
\textbar Φf i mit der Energie Ef ist durch Fermis Goldene Regel gegeben:
Tf →i =

2π \textbar hΦf \textbar HW W \textbar Φi i\textbar2 δ(Ef − Ei −
\textasciitilde ω) \textasciitilde{}

(3.2)

Hierbei bezeichnet \textasciitilde ω die Energie des Photons. Die
Wechselwirkung zwischen Elektron und Photon wird durch den
Hamiltonoperator HW W beschrieben. In Coulomb-Eichung und linearer
Näherung lautet dieser HW W =

e \textasciitilde{} A · p\textasciitilde{} . 2mc

\begin{description}
\item[(3.3)]
gegeben Das Vektorpotential der einfallenden elektromagnetischen
Strahlung ist durch A und beinhaltet Eigenschaften wie Frequenz, Phase
und Polarisation. p\textasciitilde{} ist der quantenmechanische
Impulsoperator p\textasciitilde{} = −i\textasciitilde∇. Die
Energieerhaltung fordert, dass nur Übergänge vorkommen, die der Relation
Ef = Ei + \textasciitilde ω genügen. Diese Bedingung wird in Formel 3.2
durch die δ-Funktion berücksichtigt. Es seien die Wellenvektoren des
Anfangs- und Endzustandes mit \textasciitilde ki und \textasciitilde kf
gegeben. Der Photonenimpuls kann bei den geringen Anregungsenergien von
ARPES gegenüber dem Elektronenimpuls vernachlässigt werden. Wir erhalten
daher wellenvektorerhaltende direkte Übergänge. Die Impulserhaltung
lautet somit: k\textasciitilde i = k\textasciitilde f =
\textasciitilde k
\end{description}

(3.4)

Die Übergangswahrscheinlichkeit wird im wesentlichen durch das
Betragsquadrat des Übergangsmatrixelementes in Gleichung 3.2 bestimmt.
Dieses hängt sowohl vom Anfangszustand \textbar Φi i als auch vom
Endzustand \textbar Φf i ab. In der Photoemission wird daher eine
Kombination aus beiden Zustandsdichten gemessen.

Schritt 2: Transport des angeregten Elektrons zur Oberfläche Die
Photonen dringen mehrere 100 Å in den Festkörper ein und regen
Elektronen an. Beim Transport zur Oberfläche verlieren einige Elektronen
durch inelastische Stöße kinetische Energie. Dabei geht die Information
über den Anfangszustand verloren. Die mittlere freie Weglänge der
Elektronen in Abhängigkeit von ihrer kinetischen Energie ist in
Abbildung 3.3 gezeigt. Diese „universelle Kurve`` ergibt sich aus
Messungen, die an verschiedenen Materialien durchgeführt wurden. Die
eingetragenen Messpunkte verdeutlichen, dass die mittlere freie Weglänge
weitgehend unabhängig vom Material ist. Aus der Abbildung ist
ersichtlich, dass die mittlere freie Weglänge bei den in UPS üblichen
Anregungsenergien nur einige Angström beträgt. Damit ist der Transport
zur Oberfläche der limitierende Schritt der PES und begründet seine
Oberflächensensitivität. Messungen der Photoemission repräsentieren nur
die obersten Atomschichten. Sie erfordern ein gutes Vakuum, um eine
Bedeckung der Oberfläche mit Fremdatomen zu verhindern. Weiterhin werden
besondere Anforderungen gestellt, um zuvor eine geeignete Oberfläche zu
erhalten.

10

Theoretische Grundlagen der Photoemission

Abbildung 3.2: Photoemissionsprozess und zugehörige Energieverhältnisse
für Probe und Analysator. Die einzelnen Energien werden im Text
erläutert.

Das Drei-Stufen-Modell

11

Abbildung 3.3: Energieabhängige mittlere freie Weglänge von Elektronen
im Festkörper {[}20{]}.

Abbildung 3.4: Brechung der Elektro\textasciitilde{} Wellennen an der
Kristalloberfläche. K \textasciitilde{} vektor im Vakuum, kf
Wellenvektor des Endzustandes im Kristall

Der dominierende Stoßprozess ist die Elektron-Elektron Streuung. Durch
diesen Prozess wird ein Spektrum von niederenergetischen
Sekundärelektronen generiert, die später im Photoemissionsspektrum
sichtbar sind (siehe Abbildung 3.2). Die Elektron-Phonon Wechselwirkung
hat nur bei sehr geringen Energien eine Bedeutung {[}18{]}.

Schritt 3: Durchtritt des Elektrons vom Festkörper ins Vakuum Der dritte
Schritt ist mit einer Brechung verbunden. Wir können das angeregte
Elektron innerhalb des Kristalls als quasi-frei betrachten. Die
Energie-Impuls-Beziehung für den Endzustand \textasciitilde kf vor dem
Durchtritt durch die Oberfläche lautet daher

Ef =

\textasciitilde2 \textasciitilde{} 2 k . 2me f

(3.5)

Der Wellenvektor \textasciitilde kf zerlegt sich in seine Anteile
\textasciitilde kf k parallel zur Probenoberfläche und
\textasciitilde kf ⊥ senkrecht zur Oberfläche (siehe Abbildung 3.4).
Aufgrund der Translationsinvarianz bleibt die parallele Komponente beim
Durchtritt bis auf einen reziproken Gittervektor oder
Oberflächengittervektor \textasciitilde gk erhalten: \textasciitilde{} k
= \textasciitilde kf k + \textasciitilde gk K Die Addition eines solchen
reziproken Gittervektors bezeichnet man als Umklappprozess. Betrachten
wir die Emission unter einem Winkel ϑ zur Probennormalen und
vernachlässigen Umklappprozesse (d.~h. \textasciitilde gk = 0), so
erhalten wir für die Parallelkomponente des Wellenvektors: r
\textasciitilde kf = K \textasciitilde k = k

2m Ekin sin ϑ \textasciitilde{}

(3.6)

12

Theoretische Grundlagen der Photoemission

Die senkrechte Komponente bleibt nicht erhalten, da das
Kristallpotential V0 außerhalb des Festkörpers nicht vorhanden ist. Dies
führt zu einer Verkürzung der senkrechten Komponente. Dennoch können wir
über sie eine Aussage machen, wenn wir als Endzustände freie
Elektronenparabeln annehmen. Die kinetische Energie ist dann wie folgt
gegeben:

Ekin =

\begin{description}
\item[\textasciitilde2 \textasciitilde{}]
2 − \textbar V0 \textbar{} (kf + G) 2m
\item[(3.7)]
repräsentieren die jeweiligen Elektronenparabeln. Die reziproken
Volumengittervektoren G \textasciitilde{} Der Wellenvektor kf lässt sich
in seine Komponenten \textasciitilde kf = \textasciitilde kf k +
\textasciitilde kf ⊥ zerlegen. Das Ergebnis von Gleichung (3.6) kann in
diesen Ansatz eingesetzt werden. In senkrechter Emission (ϑ = 0)
erhalten wir: r k f⊥ =
\item[2m]
2 − G⊥ (Ekin + \textbar V0 \textbar) − G k \textasciitilde2
\item[(3.8)]
in seine Komponenten zerlegt. Jedoch kann G

k auch Hier wurde der Volumengittervektor G ein reziproker
Oberflächengittervektor sein. Aufgrund der Impulserhaltung (3.4) kann so
aus (3.6) und (3.8) die gemessene Zustandsdichte einem bestimmten Punkt
in der Brillouin-Zone zugeordnet werden. Durch ein systematisches
Abrastern des k-Raumes kann auf diese Weise die Bandstruktur
E(\textasciitilde k⊥ , \textasciitilde kk ) = E(\textasciitilde k)
bestimmt werden. In der Praxis erfolgt dies entlang hochsymmetrischer
Richtungen. Ei Energie des Anfangszustandes Ef Energie des Endzustandes
EV BM Valenzbandmaximum Evac,S Vakuumenergie der Probe Evac,A
Vakuumenergie des Analysators \textasciitilde ω Energie des Photons Ub
Bindungsenergie Ethr Photoemissionsschwellwert Ekin,S kinetische Energie
der Elektrons bezüglich der Probe (Sample) Ekin,A kinetische Energie des
Elektrons bezüglich des Analysators UK Kontaktpotential ΦS
Austrittsarbeit der Probe ΦA Austrittsarbeit des Analysators EF
Fermi-Energie
\end{description}

Abbildung 3.5: Energieniveauschema für die Photoemission an Probe und
Analysator

Aus der kinetischen Energie der emittierten Elektronen lassen sich
Informationen über die Bindungsenergien im Kristall erhalten. Ist die
Energie Uthr des Photoemissionsschwellwertes bekannt, so kann die
Bindungsenergie bezüglich des Valenzbandmaximums durch folgende Formel
bestimmt werden:

Auswertung der Messdaten und Spektren

Eb = Ekin + \textbar Uthr \textbar{} − hν

13

(3.9)

Hier muss zwischen der kinetischen Energie bezüglich der Probe Ekin,S
und bezüglich des Analysators Ekin,A unterschieden werden. Die relativen
Energieverhältnisse sind in Abbildung 3.5 noch einmal gesondert
dargestellt. Der Unterschied zwischen den beiden kinetischen Energien
entspricht dem Kontaktpotential UK . Es ergibt sich aus der Differenz
der Austrittsarbeiten von Probe und Analysator Ekin,S = Ekin,A − UK

(3.10)

In den Gleichungen (3.6), (3.8) und (3.9) bezeichnet Ekin die kinetische
Energie bezüglich der Probe. Letzten Endes wird die kinetische Energie
jedoch im Analysator bestimmt. Die maximal gemessene kinetische Energie
hängt damit nur von der Energie der anregenden Photonen und der
Austrittsarbeit des Analysators ab - siehe auch die eingangs erwähnte
Formel (3.1). Im Falle eines untersuchten Metalls entspricht diese der
Fermi-Energie. Somit ist die gemessene energetische Lage der Fermi-Kante
unabhängig von der Austrittsarbeit der Probe. Dieser spezielle Umstand
des elektrostatischen Elektronenanalysators wird z. B. in der
Untersuchung von HTSCs4 genutzt. Es werden die Spektren einer Probe
aufgenommen, die von besonderem Interesse sind. Anschließend wird die
Probe an einer speziellen Stelle der Vakuumkammer mit amorphen Gold
bedampft. Danach wird die Fermi-Kante der so präparierten Oberfläche
gemessen. Dies dient als Referenzpunkt für Bindungsenergien. Im Rahmen
dieser Diplomarbeit wurden die gemessenen Spektren ebenfalls auf die
Fermi-Energie bezogen, deren Lage mit einer amorphen Goldprobe bestimmt
wurde. Außerdem ist es möglich, auf diese Weise die Austrittsarbeit der
untersuchten Probe selbst zu bestimmen. Die bestimmte Fermi-Energie ist
bei jeder Messung an derselben Stelle, weil das gesamte Spektrum beim
Eintritt in den Analysator verschoben wird. Der Betrag der Verschiebung
ist durch die Kontaktspannung gegeben (Abbildung 3.5). Die meisten
Analysatoren sind innen mit Graphit beschichtet, welches eine geringe
Austrittsarbeit von 4,14 eV besitzt. Die minimale kinetische Energie der
Sekundärelektronen beim Austritt aus der Probe beträgt 0 eV. Diese Kante
wird ebenfalls um die Kontaktspannung UK verschoben. Aus der mit dem
Analysator bestimmten niederenergetischen Grenze der Sekundärelektronen
kann damit die Austrittsarbeit der Probe bestimmt werden.

3.3

Auswertung der Messdaten und Spektren

3.3.1

Konstanter Untergrund

Die zur Detektion der Elektronen eingesetzten elektronischen Bauelemente
(u. a. Channeltron, Operationsverstärker) besitzen ein
charakteristisches Eigenrauschen. Dieses Rauschen verursacht einen
energieunabhängigen konstanten Untergrund IB . Betrachten wir die
Intensität Ii an einem Messpunkt oder Kanal i. Dieser Wert Ii kann der
Intensität bei einer bestimmten Energie I(E) zugeordnet werden. Diese
Intensität sollte bei kinetischen Energien oberhalb der Fermi-Energie
Null betragen. Damit lässt sich der konstante Untergrund IB bestimmen
und von den Messdaten abziehen {[}21{]}: 4

HTSC - High Temperature SupraConductor, Hochtemperatursupraleiter oder
High-TC -Supraleiter

14

Theoretische Grundlagen der Photoemission

Ii0 = Ii − IB

(3.11)

In den letzten Jahrzehnten sind große Fortschritte in der Technologie
der Detektoren sowie ihrer Elektronik gemacht worden. Oftmals besitzen
sie eine Einstellmöglichkeit für die Rauschunterdrückung {[}19{]}. So
kann dieser Anpassungsschritt heute meist entfallen.

3.3.2

Subtraktion des inelastischen Untergrundes

Insbesondere bei Anregung durch die He-Gasentladungslampe ist das
gemessene Spektrum von einem deutlichen inelastischen Untergrund
überlagert (siehe Abbildung 3.6 im Vergleich zu Abbildung 7.4). Ein
wichtiger Unterschied ist die Größe des Fokus, in dem die Anregung
erfolgt. An der BUS5 -Beamline liegt der Durchmesser des bestrahlten
Fleckes im Bereich von 100 µm. Das ermöglicht bei empfindlichen Proben
ein „Abrastern`` der Oberfläche {[}23{]}. Demgegenüber regen die
Photonen der He-Lampe HIS 13 trotz Fokussierspiegel einen Bereich von
ca. 1mm Durchmesser an {[}19{]}.

Abbildung 3.6: Valenzbandspektrum von HgCdTe mit inelastischem
Untergrund, berechnetem Untergrundsignal und dem Spektrum ohne
Untergrund.

Wir betrachten wieder die Intensität Ii0 an einem Messpunkt i. Zur
Korrektur der inelastischen Streuung wird von der Intensität Ii0 ein
Betrag abgezogen, der proportional zur integrierten Intensität des
Valenzbandspektrums bei höheren Energien E \textgreater{} E(i) ist. Die
korrigierte Intensität Ii00 ergibt sich daher wie folgt:

Ii00 = Ii0 − I000

X k\textgreater i

Ik0

X k\textgreater0

Ik0



(3.12)

Die Intensität I00 in Kanal 0 steht für eine Energie unterhalb des
Valenzbandes und wurde mit I000 = Ii0max festgelegt. 5

BUS - Berliner Universitäts Strahlrohr, XUV-Beamline bei BESSY, siehe
Abschnitt 4.4

Trennung von Oberflächen und Volumenbandstruktur

3.3.3

15

Glättung der Spektren nach Savitzky-Golay

Wie jedes andere analoge Signal sind die gemessenen Intensitäten mit
einem statistischen Fehler behaftet, der sich in einem Rauschen des
Messsignals äußert. Dieses lässt sich selbstverständlich durch eine
Erhöhung der Statistik verringern. Damit ist aber auch ein Anstieg der
nötigen Messzeit verbunden; der möglichen Zählrate sind Grenzen gesetzt.
In der Praxis muss ein Kompromiss zwischen akzeptabler Messzeit und
hoher Zählrate gefunden werden. Die Spektren der Photoemission werden
meist mit sehr hoher Auflösung aufgenommen, so dass die einzelnen Werte
von Datenpunkt zu Datenpunkt nur wenig variieren. Diese sind mit einem
Rauschen überlagert. Für diese Art Daten eignet sich der
Glättungsalgorithmus nach Savitzky-Golay {[}24{]} sehr gut. Hierbei
ergibt sich der geglättete Wert Ii0 eines Messpunktes aus der
gewichteten Mittelung über seine Nachbarwerte: Ii0 =

nR X

cn Ii+n

(3.13)

n=−nL

Die Mittelung im Intervall k ∈ {[}i − nL , i + nR {]} soll gerade dem
Wert des Least-SquarePolynomfits durch die Punkte Ik am Punkt Ii
entsprechen, die Koeffizienten cn werden entsprechend gewählt. Zur
Auswertung der Spektren wurde die Software Origin 7.5 verwendet
{[}25{]}. Sie bietet einen eingebauten Algorithmus zur Datenglättung
nach Savitzky-Golay. Er fand bei den ausgewerteten Spektren Anwendung.

3.4

Trennung von Oberflächen und Volumenbandstruktur

Im Allgemeinen interessiert bei Photoemissionsmessungen die
elektronische Struktur des Volumens. Bedingt durch die starke
Oberflächensensitivität dieser Messmethode spielen energetische Zustände
der Oberfläche in den gemessenen Spektren eine große Rolle. Die
Wellenfunktion dieser Zustände fällt auf beiden Seiten der Oberfläche
exponentiell ab. Im Fall einer Entartung mit Volumenzuständen fällt sie
im Kristall nur auf einen endlichen Wert ab, man spricht von einer
Oberflächenresonanz. Der Ursprung dieser abweichenden Zustände liegt in
der abweichenden Bindungsstruktur der Oberfläche. Dazu zählen sogenannte
'dangling bonds' - freie Valenzen - sowie Brückenbindungen ('bridge
bonds') und Rückbindungen ('back bonds'). Oberflächenabgeleitete
Zustände besitzen einige Eigenschaften, die sie von Volumenzuständen
unterscheiden. In der folgenden Liste sind einige aufgeführt {[}7{]}.
Dabei ermöglicht ein einzelner Punkt allein keine eindeutige Zuordnung
{[}11{]}. Vielmehr sollten immer mehrere dieser Kriterien erfüllt sein:
• Die energetische Lage von Oberflächenzuständen ist unabhängig von der
Photonenenergie. Bei einer Messung in normaler Emission (auch: k⊥
-Messung, kk = 0) zeigen sie keine Dispersion. • Ihre Periodizität ist
an die Oberflächen-Brillouin-Zone gekoppelt. Sie zeigen daher ein
anderes Dispersionsverhalten bezüglch kk . • Oberflächenzustände liegen
teilweise in einer Energielücke des Volumens. Oberflächenresonanzen
fallen dagegen mit Volumenzuständen zusammen.

16

Theoretische Grundlagen der Photoemission

• Oberflächenzustände reagieren empfindlich auf Adsorbate. Sie
verschwinden vielfach bei Adsorption von weniger als einer Monolage. •
Volumenzustände können häufig durch den Vergleich mit
Volumenbandstrukturrechnungen identifiziert werden. • Wenn ebenso
Bandstrukturrechnungen für die Oberfläche vorliegen, kann ein Vergleich
weitere Hinweise auf Oberflächenzustände liefern. • Teilweise besitzen
Oberflächenzustände Orbitalcharakter. Durch gezielte Variation des
Übergangsmatrixelementes oder der Symmetrieeigenschaften in Abhängigkeit
vom Azimutalwinkel kann dieser Charakter untersucht werden.

Kapitel 4

Experimentelles Alle Photoemissionsmessungen wurden mit der
hochauflösenden AR65-Anlage durchgeführt. Sie ist transportabel
gestaltet, daher konnten Messungen sowohl im Labor mit einer He-Lampe
als auch bei BESSY an einem Synchrotronmessplatz durchgeführt werden.

4.1

Die Photoemissionsanlage AR65

Die UHV-Anlage AR65 wurde in unserer Arbeitsgruppe bis 1999 aufgebaut
{[}26{]}, um hochauflösende Messungen am neu gebauten Synchrotron BESSY
II zu ermöglichen {[}27{]}. Die 750kg schwere Anlage wurde kompakt und
fahrbar aufgebaut, damit an verschiedenen Beamlines oder auch im Labor
gemessen werden kann. Sie besteht aus einer Hauptkammer, in der die
eigentlichen Messungen durchgeführt werden, und einer Einschleuskammer.
In Kapitel 2 wurde die hohe Oberflächensensitivität der Photoemission
erläutert. Weitere Aspekte HgCdTe betreffend werden in Kapitel 6
erörtert. Um diesen hohen Anforderungen gerecht zu werden, ist die
Hauptkammer mit einer Vielzahl unterschiedlicher Pumpen verbunden. Diese
laufen an der Pumpkammer zusammen, dort findet auch die Druckmessung
statt. Die Pumpkammer ist über ein 200mm-Pumprohr von 1 Meter Länge mit
der Messkammer verbunden. Auf diese Weise sollen störende Einflüsse der
Pumpsysteme und des Druckmesskopfes auf die Photoemissionsmessung
minimiert werden. Insbesondere magnetische Streufelder der
Ionengetterpumpe könnten das Messergebnis beeinflussen. An die
Pumpkammer sind eine 500l-Turbomolekularpumpe, eine
500l-Ionengetterpumpe, eine 1200l-Titansublimationspumpe sowie eine
1500l-Kryopumpe angeschlossen. Die angegebenen Pumpleistungen pro
Sekunde beziehen sich mit Ausnahme der TSP (hier H2 ) auf Stickstoff.
Mit ihnen kann am Pumpenkreuz ein Druck von p = 1.3 · 10−10 mbar
erreicht werden (2007), ohne Kryopumpe werden noch p = 3 · 10−10 mbar
erreicht. Der entsprechende Druck in der Hauptkammer liegt um den Faktor
2 höher {[}26{]}. Die Druckmessung der IGP unterschreitet im ersteren
Fall ihren Grenzbereich von 6.4 · 10−11 mbar und zeigt low
pressure\ldots{} an. Um störende Magnetfelder (wie auch das
Erdmagnetfeld) vom Messort fernzuhalten, ist die Messkammer innen mit
einer Abschirmung aus µ-Metall versehen. Dadurch können äußere
Magnetfelder auf 0,5 \% reduziert werden. Die winkelaufgelöste
Untersuchung der emittierten Photonen erfolgt mit dem Photoelek-

17

18

Experimentelles

tronenanalysator AR65 der Firma Omicron. Er besitzt einen 180◦
Kugelkondensator oder SDA1 mit einem Sollbahnradius von 65 Millimetern
und ist damit der Namengeber der gesamten Anlage. Dieser ist auf einem
2-Achsen-Goniometer montiert. Damit können Winkel im Bereich von 10◦
\textless{} Φ \textless-90◦ in der Vertikalen und 0◦ \textless{} θ
\textless{} 350◦ in der Horizontalen angefahren werden. Allgemein kann
die spektrale Auflösung ∆EA eines Kugelkondensators mithilfe der
folgenden Formel {[}28{]} bestimmt werden:  ∆EA = Epass

s + α2 2r0

(4.1)

Sie hängt für eine Passenergie Epass der eintretenden Elektronen unter
einem Öffnungswinkel α vom Radius r0 des Kondensators sowie der Breite s
des Eingangs- und Ausgangsspaltes ab. Für die AR65 sind die Spaltbreiten
mit s = 1mm gegeben, die Winkelakzeptanz beträgt α = ±1◦ . Für eine
standardmäßig verwendete Passenergie von Epass = 10eV ergibt sich damit
eine Energieauflösung von ∆EA ≈ 80meV . Im Jahre 2007 wurde die
Winkelansteuerung des AR65-Analysators mit einer Schrittmotorsteuerung
versehen {[}19{]}. Auf diese Weise können bis zu 10 Orte automatisch
angefahren werden, um dort Spektren aufzunehmen. Außerdem ist es
möglich, große Winkelbereiche automatisiert abzufahren, um zum Beispiel
Fermiflächen zu vermessen. Die Auswertesoftware SPECTRA der Firma
Omicron wurde entsprechend angepasst und erweitert. In einer höheren
Ebene ist die Messkammer auf einem DN 150 CF Flansch mit einem
LEED/Auger-System versehen. Auch dieses System ist mit einer
µ-Metallabschirmung versehen. Mit dieser Messapparatur können die Proben
noch im Vakuum vor oder nach der Messung charakterisiert und orientiert
werden. Das LaB6 -Einkristallfilament wurde im Sommer 2007 gegen ein
Wolframfilament getauscht. Damit darf der Filamentstrom im Dauerbetrieb
nun bis zu 2 Ampere betragen. Die Aufnahme der Proben in der Messkammer
erfolgt in einem Manipulatorkryostat. Dieser wurde von der Universität
Kiel in Zusammenarbeit mit der Firma VAb Elmshorn entwickelt {[}29{]}.
Der Manipulator besitzt 5 Freiheitsgrade: Neben den drei Translationen
in den Richtungen x, y und z kann er auch um die z-Achse sowie die Achse
senkrecht zur Probennormalen rotiert werden. Die Kühlung kann sowohl mit
flüssigem Helium als auch mit Stickstoff erfolgen. Mit flüssigem Helium
wurden Temperaturen von 10,5 K innerhalb von 40 Minuten erreicht
{[}26{]}. Die Temperaturmessung erfolgt mittels eines PT100 und einer
Siliziumdiode (Lake Shore DT470-CO), die mittels Vierpunktmethode
angeschlossen wurden. Um einen guten Enddruck in der Messkammer
sicherzustellen, wird auch der Manipulator differentiell abgepumpt. In
der Nähe des Einschleuspunktes ist außerdem ein sogenannter
„Wobblestick`` an der Messkammer befestigt. Mit ihm können mechanische
Manipulationen im UHV durchgeführt werden. Im Allgemeinen wird er
benutzt, um einen auf die Probenoberfläche aufgeklebten Probenstempel
abzureißen. Dadurch wird die Probe im Vakuum gespalten und eine saubere
Oberfläche erzeugt. An die Hauptkammer ist eine spezielle Transferkammer
mit drei Probenplätzen montiert, um Proben ohne Unterbrechung des
Vakuums der Hauptkammer zum Messort zu bringen. Diese Einschleuskammer
ist bewusst klein gehalten, um den Enddruck von p = 7 · 10−10 mbar
schneller zu erreichen. Die Kammer wird mit einer Turbomolekularpumpe
evakuiert. Eine Kombinationsmessröhre (Wärmeleitungsvakuummeter nach
Pirani und Bayart-Alpert Glüh1

Spherical Deflection Analyser

Heliumlampe Focus HIS 13

19

kathoden-Ionisationsvakuummeter) für den Messbereich von 10−10 bis 103
mbar ist ebenfalls angeschlossen. Außerdem besitzt die Kammer einen
Goldofen, um amorphes Au auf die Probenoberfläche aufzudampfen. Nach dem
Transfer in die Messkammer kann damit die FermiKante sehr genau bestimmt
werden.

4.2

Heliumlampe Focus HIS 13

Im Labor diente die Gasentladungslampe HIS 13 der FOCUS GmbH (Vertrieb
über OMICRON) als Photonenquelle. In ihrem Aufbau ist erkennbar, dass
die Hochspannungsanode in die Entladungskapillare integriert wurde
(Abbildung 4.1). Dadurch werden sehr hohe Intensitäten von bis zu 1016
Photonen je Sekunde und Steradiant erreicht {[}30{]}. In der
Entladungsröhre wird ein Plasma gezündet und hauptsächlich der
HeIα-Übergang (21,22 eV) zur Anregung der Proben genutzt.

Abbildung 4.1: Querschnitt durch die Gasentladungslampe HIS 13 {[}30{]}.

Diese Photonenenergie liegt im Bereich des VUV2 . Dieser bezeichnet den
Energiebereich von 6,2 bis 124 Elektronenvolt. In diesem Bereich
existiert kein Fenstermaterial, auch in Luft werden Photonen dieser
Energie stark absorbiert. Daher muss die Anbindung einer solchen
Photonenquelle über ein Vakuum erfolgen, daraus folgt auch die
Bezeichnung VUV für diesen Bereich. Das Heliumplasma benötigt jedoch
einen Druck von ca. 8 · 10−2 mbar für einen kontinuierlichen
Brennvorgang. Um nun das Plasma vom UHV zu trennen, wird die Kapillare
zur Messkammer differentiell gepumpt. Seit 2007 findet eine Kapillare
von 0,8 Millimetern Durchmesser Verwendung. Mit dieser steigt der Druck
in der Hauptkammer bei Betrieb der Heliumlampe auf einen Wert von 5 ·
10−10 mbar an. Dieser Anstieg des Enddrucks in der Messkammer stellt in
der Regel kein Problem dar, da sich Helium als Edelgas inert verhält,
also keine chemische Bindung eingeht oder die Probe anderweitig
verändert. Im Gegensatz zu einem Messplatz am Synchrotron ist man mit
einer Gasentladungslampe auf eine Anregungsenergie festgelegt, die sich
aus dem verwendeten Gas und dem Druck ergeben. Die verwendeten
Bedingungen sind auf eine Emission der HeIα-Linie optimiert. Sie besitzt
die beiden Satelliten HeIβ (23.09 eV) und HeIγ (23.74 eV). Mit relativen
Intensitäten von 1,5 \% bzw. 0,5 \% können diese jedoch vernachlässigt
werden {[}19{]}. 2

Vacuum Ultra Violett

20

4.3

Experimentelles

Automatische N2 Nachfüllanlage

Die Versorgung der Gasentladungslampe mit Helium erfolgt aus einer 200
bar Gasflasche. Ihr nachgeschaltet ist ein Druckminderer, der den
Arbeitsdruck auf konstant 1 bar hält. Hinter dem Druckminderer
durchläuft das Helium zunächst eine Kühlfalle, bevor es durch ein
Dosierventil in die Brennkammer geleitet wird. Die Kühlfalle hat die
Aufgabe, die eventuell vorhandenen Verunreinigungen „auszufrieren``. Im
Langzeitbetrieb könnten diese Verunreinigungen die Leistungsfähigkeit
und Zuverlässigkeit der Lampe einschränken. Zum Beispiel könnte
Sauerstoff in der Brennkammer zu chemischen Reaktionen an der Anode oder
Kathode führen. Die Kühlfalle besteht aus einem Edelstahlbehälter von
ca. 2 Litern Volumen, der UHV-dicht abgeschlossen ist. Diese Falle
befindet sich in einem thermisch isolierten Behälter, der mit flüssigem
Stickstoff gefüllt ist. Auf diese Weise besitzt auch die Innenwand der
Kühlfalle eine Temperatur von 77 Kelvin. Bei dieser Temperatur gehen
alle Gase in den flüssigen oder festen Aggregatzustand über (mit
Ausnahme von Helium) und kondensieren aus bzw. frieren fest. Vor
Inbetriebnahme wird die Kammer zweifach mit Helium gespült, danach kann
der Dauerbetrieb aufgenommen werden.

Abbildung 4.2: Elektrische Schaltung zur Renormierung des PT 1000, damit
dieser an einem PT100-Eingang verwendet werden kann.

Abbildung 4.3: Temperaturcontroller der Nachfüllanlage im
Isolierbehälter für flüssigen Stickstoff.

Der als Kühlflüssigkeit verwendete flüssige Stickstoff geht im Laufe der
Zeit durch Energieabgabe an die Umgebung in die Gasphase über. Deshalb
muss in regelmäßigen Abständen flüssiger Stickstoff in den
Isolierbehälter nachgefüllt werden. Bisher ist dies manuell in einem
Intervall von 30 Minuten erfolgt, die Messungen konnten somit nur unter
Betreuung durchgeführt werden. Nachdem die Winkelverstellung durch die
Installation von Schrittmotoren im Jahre 2006 automatisiert wurde,
stellte dies eine unnötige Einschränkung dar. Auch der Nachfüllprozess
für den flüssigen Stickstoff sollte automatisiert werden. Bestandteil
dieser Diplomarbeit ist daher die Konstruktion einer automatischen
Nachfüllanlage für kryogene Flüssigkeiten. In Folge sollte es möglich
sein, im Labor rund um die Uhr Messungen vornehmen zu können. Der erste
Bestandteil dieser Nachfüllanlage ist ein passendes Magnetventil, das in
die Leitung zwischen Devargefäß und Phasenseparator zwischengeschaltet
wird. Die Mechanik des Ventils muss wasserdicht konstruiert sein, damit
kondensierender und erstarrender Wasserdampf der Umgebungsluft nicht zu
einer Blockierung des Ventilmechanismus führt. Weiterhin

Automatische Stickstoff-Nachfüllanlage

21

muss das verwendete Dichtelement auf tiefe Temperaturen ausgelegt sein.
Diese Anforderungen werden vom Magnetventil ASCO Joucomatic cryogenic
263LT (SCE263B206LT) mit PTFE3 Dichtung erfüllt. Der zweite Teil der
Anlage besteht aus einem Füllstandssensor mit angeschlossener
Auswerteelektronik, um das Magnetventil anzusteuern. Wir haben uns
entschieden, den Füllstand mithilfe eines Temperatursensors zu
überprüfen. Dazu sind eine Reihe von Besonderheiten zu berücksichtigen,
die im Folgenden erörtert werden. Wir verwenden einen
Temperaturcontroller Tempatron PJ32 V6P. Er besitzt zwei Eingänge für
einen PT100 sowie einen Relaisausgang, der bis zu 16 Ampere schaltet.
Der ausgewertete Temperaturbereich beträgt -50 ◦ C bis +150 ◦ C. Für den
Platinsensor PT100 ist der temperaturabhängige Widerstandswert
(Temperatur T hier in Grad Celsius) nach folgender Näherung definiert
{[}31{]}: R(T ) = R0 (1 + aT + bT 2 + c(T − 100 ◦ C)T 3 )

(4.2)

a = 3, 9083 · 10−3 ◦ C−1 b = −5, 775 · 10−7 ◦ C−2 c = −4, 183 · 10−12 ◦
C−4 Der Nennwiderstand eines PT100 ist für die Temperatur von 0 ◦ C
definiert und beträgt R0 = 100Ω. Sein Widerstandswert sinkt bei der
Temperatur von flüssigem Stickstoff auf 20,3 Ω ab. Die
Auswerteelektronik des Temperaturcontrollers hat jedoch seine untere
Messgrenze bei einem Widerstandswert von 80,3 Ω, entsprechend der
Temperatur von -50 ◦ C. Eine Lösung wäre nun die Reihenschaltung mit
einem Widerstand von 79,7 Ω. Dann würde eine Anzeige des Controllers von
„0`` exakt der Temperatur flüssigen Stickstoffs entsprechen. Der
Temperaturkoeffizient des PT100 beträgt bei dieser Temperatur α(77K) =
4, 27 · 10−3 K −1 im Gegensatz zu α(300K) = 3, 85 · 10−3 K −1 bei
Raumtemperatur. Dies ist jedoch nicht weiter von Belang, da keine exakte
Temperatur gemessen werden soll, sondern eine zuverlässige Anzeige der
Füllhöhe gewünscht ist. Es stellte sich jedoch heraus, dass die
Temperatur des Sensors einen Zentimeter oberhalb der Flüssigkeit durch
die gute Isolationswirkung des Behälters nur 83K (-190 ◦ C) beträgt, die
Digitalanzeige folglich nur auf „6`` steigen würde. So suchten wir nach
einer Möglichkeit, die Messgenauigkeit zu erhöhen, um feinere
Einstellungen des Nachfüllsystems vornehmen zu können. Zunächst haben
wir den Temperatursensor durch einen PT1000 ersetzt, da er eine zehnmal
höhere Empfindlichkeit besitzt. Jedoch beträgt sein Widerstandswert bei
77 K noch 202,64 Ω, liegt also wieder außerhalb des Messbereiches
unseres Temperaturcontrollers. Um diesen Wert zu verringern, haben wir
zwei weitere Widerstände parallelgeschaltet (siehe Abbildung 4.2). Der
Gesamtwiderstand dieser elektrischen Schaltung ergibt sich nach der
Formel 1 Rparallel

=

1 1 + R1 + R2 RPT1000

.

(4.3)

In einem zweiten Schritt haben wir die Anzeige am Temperaturcontroller
auf die Einheit Fahrenheit eingestellt. Da der eigentliche Messwert egal
ist, wird dadurch die relative Sensiti3

PolyTetraFluorEthen, umgangssprachlich als Teflon bezeichnet

22

Experimentelles

vität eines Digits zum Widerstandswert um den Faktor 1,8 erhöht. Die
„0`` entspricht nun einer Temperatur von 255,6 K (-17,8 ◦ C) und einem
Widerstand R = 93, 11Ω des Temperatursensors. Für den Sollwert von 93,11
Ω muss die Reihenschaltung von R1 und R2 nach (4.3) einen
Widerstandswert von 172,3 Ω ergeben. Aus der E12-Reihe wurde ein R1 =
150 Ω Festwiderstand sowie ein R2 = 39 Ω Potentiometer ausgewählt. Mit
dem Potentiometer kann der Nullpunkt für die Messelektronik genau
abgeglichen werden. Die Empfindlichkeit dieser Parallelschaltung beträgt
α = 8, 35 · 10−3 K −1 und ist damit doppelt so groß wie die
Reihenschaltung eines PT100 mit einem passenden Festwiderstand.
Abschließend musste die Anlage noch kalibriert werden. Die Anzeige „0``
entspricht nun der Temperatur von flüssigem Stickstoff. Der Sollwert ist
auf „5`` eingestellt, die Hysteresis (A0) auf ihrem Maximalwert. Bei
dieser Einstellung schaltet sich die Nachfüllanlage bei einem Messwert
von „20`` ein. Für den optimalen Zeitpunkt der Abschaltung des
Nachfüllprozesses ist die Anzeigestabilität (/2) auf 7 eingestellt. Zur
Sicherheit beträgt die Mindestausschaltzeit der Regelausgänge (c2) 15
Minuten. Im praktischen Einsatz erfolgt alle 35 Minuten eine
automatische Nachfüllung des Kühlbehälters. Mit einem 100l-Dewar kann so
60 Stunden ununterbrochen gemessen werden. Durch die Schaltvorgänge des
Magnetventils treten in den aufgenommenen Spektren Spikes auf. Da diese
Vorgänge jedoch auf ein Minimum reduziert wurden, halten sich diese in
vertretbaren Grenzen. Ein Auswerteprogramm für die Spektren der AR65
{[}32{]} ist mit einem Algorithmus zur Erkennung und Entfernung von
Spikes ausgestattet.

4.4

BUS-Beamline bei BESSY

Die meisten Photoemissionsmessungen dieser Diplomarbeit wurden an der
BUS-XUV-Beamline {[}33{]} bei BESSY durchgeführt. Die Messung an einem
Synchrotron zeichnet sich durch eine Reihe von Vorteilen gegenüber der
Messung mit einer Gasentladungslampe aus: • Die Synchrotronstrahlung
besitzt einen großen Spektralbereich, auf den mittels eines
Monochromators selektiv zugegriffen werden kann. Die Photonenenergie ist
somit einstellbar. • Die Strahlung ist auf einen kleinen Bereich
fokussiert (\textless{} 100 µm). • Verglichen mit der Gasentladungslampe
ist der Photonenfluss um den Faktor 10 höher (siehe {[}19{]}, zu
beachten: Undulatorbeamline, gilt nicht unbedingt für Dipolbeamline). •
Es herrschen sehr gute UHV-Bedingungen, da kein Edelgasplasma zur
Anregung verwendet wird. • Synchrotronlicht ist in der Hauptebene linear
polarisiert, somit sind polarisationsabhängige Messungen möglich. Zur
Zeit werden alle Undulatoren bei BESSY umgebaut, damit eine beliebige
Polatisation einstellbar ist. Diesen vielen Vorteilen gegenüber steht
der technische Aufwand zum Betrieb eines Synchrotrons. Als
Photonenquelle in Laborgröße stehen sie zur Zeit noch nicht zur
Verfügung, befinden sich aber in der Entwicklungs- und Testphase. Die
Messungen mussten bei BESSY angemeldet werden und innerhalb der
zugeteilten zwei Wochen durchgeführt werden. Während

Röntgenuntersuchung mittels Laue

23

dieser Messzeit steht der Strahl nur 50 \% der Zeit zur Verfügung, da
der Undulator U125/2 noch von zwei weiteren Gruppen genutzt wird. Der
Undulator ist für Energien von 10 bis 600 eV ausgelegt. Aus diesem
Spektrum wird die gewünschte Photonenenergie mit einem sphärischen
Gittermonochromator (SGM) ausgewählt. Zur Zeit stehen zwei Gitter mit
500 (30 - 125 eV) bzw 1100 (64 - 180 eV) Linien pro Millimeter zur
Verfügung. Das Gap des Undulators kann automatisch an die Energie des
Monochromators angepasst werden. Die Intensität der in Photoemission
emittierten Elektronen steht in (weitgehend) linearem Zusammenhang mit
der Intensität der anregenden Photonen. Es ist daher notwendig, die
Intensität der Anregung zu kennen, um gemessene Spektren renormieren
oder miteinander vergleichen zu können. Standardmäßig ist zu diesem
Zweck ein Goldnetz kurz vor dem Experiment in den Strahlgang der
Beamline geschaltet. Es absorbiert weniger als 5 \% der Strahlung und
kann während der Photoemissionsmessung im Strahl belassen werden.
Mithilfe des Goldnetzes kann ebenfalls das Gap des Undulators auf
maximale Intensität optimiert werden.

Abbildung 4.4: Intensität der BUS XUV-Beamline bei verschiedenen
Photonenenergien, gemessen mit der Si-Diode im Strahlstrom. Die
Messwerte wurden auf einen Strom von 200 mA im Synchrotron renormiert.

In der Messzeit im August 2007 stand das Goldnetz leider nicht zur
Verfügung. Die Intensität der Beamline konnte jedoch mit einer
Siliziumdiode bestimmt werden, die in den Strahl gebracht wurde.
Allerdings absorbiert sie alle Photonen, eine simultane Messung des
Photonenstromes zur Photoemissionsmessung ist nicht möglich. Daher wurde
die Energieabhängigkeit der Intensität in eine Tabelle aufgenommen und
auf einen Wert des Ringstromes von 200 mA normiert (siehe Abbildung
4.4). Ausgehend vom tatsächlichen Wert des Ringstromes und der bekannten
Energie des Monochromators kann die Intensität nun berechnet werden. Der
aktuelle Wert kann auf der Webseite von BESSY mit einem Java-Applet
ausgelesen werden. Man kann auch eine Tabelle mit den Daten der letzten
24 Stunden ausgeben. Zudem steht ein Referenzkanal zur Verfügung, der in
die eigenen Auswertesoftware eingebunden und ausgelesen werden kann. Der
Wert lässt sich so in die Messdatei integrieren.

4.5

Röntgenuntersuchung mittels Laue

Eine übliche Methode der Kristallstrukturanalyse ist die Untersuchung
von Proben nach dem Laue-Verfahren. Dabei wird die einkristalline Probe
mit weißem Röntgenlicht bestrahlt und die gebeugten Intensitäten mit
einer Photoplatte aufgenommen. Die Beugungsmaxima ent-

24

Experimentelles

sprechend jeweils einer Gitterebene (hkl), die die Lauesche
Interferenzbedingung \textasciitilde k − k\textasciitilde0 =
\textasciitilde ghkl

\textbar\textasciitilde k\textbar{} = \textbar k\textasciitilde0
\textbar{}

(4.4)

erfüllt. Die Aufnahmen in Transmission werden auch als (gewöhnliche)
Lauegramme bezeichnet. Reflexionaufnahmen tragen die Bezeichnung
Epigramm. Für Transmissionsmessungen müssen die Proben hinreichend dünn
sein. Die Lauemessungen dieser Arbeit wurden vornehmlich an der Moskauer
Lomonosow-Universität (Moskovskiǐ Gosudarstvenyǐ Universitet imeni
Lomonosova) am Lehrstuhl für Festkörperphysik (Kafedra Fiziki tvërdogo
tela) von Frau Inna Telegina (Telegina Inna Vasiljevna) durchgeführt. Die
verwendete Röntgenkamera hat die Bezeichung RKW-86A (Rentgenovska Kamera
Vraweni RKV86A).

Abbildung 4.5: Röntgenkamera RKV-86A in Moskau. Zu erkennen sind die
Röntgenröhre, Mikroskop, Goniometer, Photoplatte, Absorber und
Kapillare.

In Abbildung 4.5 erkennt man den typischen Aufbau für ein Epigramm. Im
Vordergrund ist das Mikroskop mit Lichtquelle zu sehen, zwischen beiden
befindet sich ein Absorber. Hinter dem Goniometer ist das Gehäuse für
die Photoplatte befestigt. In der Mitte enthält es eine Bohrung für die
Kapillare, durch welche die Röntgenstrahlung kollimiert wird. Am rechten
Bildrand erkennt man schließlich die Röntgenröhre. Die Abmessungen der
Epigramme (80 × 100 mm) sind etwas geringer als die der Lauegramme (120
× 100 mm). Letztere besitzen kein charakteristisches Loch in der Mitte.
Das rechte Bild von Abbildung 4.5 zeigt das Goniometer, auf dem die zu
untersuchende Probe montiert wird. Wie man erkennen kann, lässt sich die
relative Positionierung auf ein zehntel Grad genau in beiden Achsen
vornehmen. Auch in den Translationsfreiheitsgraden lässt sich die Probe
sehr genau (ca. 0.1 mm) im Röntgenstrahl positionieren. Damit ist es
auch möglich, eine ortsaufgelöste Laueuntersuchung durchzuführen. Dies
wird in einigen Fällen gewünscht.

Kapitel 5

Das Material Hg1-XCdXTe und Charakterisierung der Proben 5.1

Eigenschaften

Die vollständige Bezeichnung Hg1-X CdX Te deutet bereits an, dass dieses
Material aus den beiden Halbleitern CdTe und HgTe zusammengesetzt ist.
In einigen Eigenschaften sind sich die beiden Ausgangsmaterialien sehr
ähnlich. Zum Beispiel sind die Gitterkonstanten nur etwa 7 Promille
voneinander entfernt (6,488 bzw 6,445 Å, siehe Abbildung 5.2). Eine
andere fundamentale Eigenschaft eines Halbleiters ist die Breite seiner
Energielücke. In diesem charakteristischen Merkmal unterscheiden sich
die beiden Materialien grundlegend.

Abbildung 5.1: Breite der Energielücke von Hg1-X CdX Te in Abhängigkeit
der Komposition x nach Formel (5.3). Der Einfluss der Temperatur auf den
Kurvenverlauf ist für 80K und 300K demonstriert.

Bei CdTe handelt es sich um einen konventionellen II-VI-Halbleiter
{[}11{]}. Er besitzt eine direkte Bandlücke von 1,56 eV (300 K). Im
allgemeinen besitzt das Valenzbandmaximum in Halbleitern der
Zinkblende-Struktur eine Γ8 -Symmetrie, während das Leitungsbandminimum
eine Γ6 -Symmetrie aufweist. Daher

25

26

Das Material HgCdTe und Charakterisierung der Proben

ist die fundamentale Energielücke E0 für II-VI-Halbleiter wie folgt
definiert: E0 = E(Γ6 ) − E(Γ8 ) .

(5.1)

Im Falle von HgTe fällt jedoch das Leitungsbandminimum mit dem
Valenzbandmaximum zusammen, sie weisen eine Γ8 -Symmetrie auf. Das Γ6
ist unter das Valenzbandmaximum geklappt und die Bandlücke daher nach
Definition (5.1) negativ. Man spricht von einer invertierten
Bandstruktur. Bei einer Temperatur von 80 Kelvin beträgt die
Energielücke von HgTe EG = −0.283eV {[}34{]}. Mit der invertierten
Bandstruktur lassen sich die Ergebnisse erklären, die in
magnetooptischen Transportmessungen sowie Experimenten zum Hall-Effekt
gefunden wurden. Die negative Bandlücke sowie das umgeklappte Γ6 -Band
konnten in unserer Arbeitsgruppe experimentell mittels winkelaufgelöster
Photoemission bestätigt werden {[}34{]}. Die beiden Halbleiter CdTe und
HgTe lassen sich nun in beliebigem Verhältnis miteinander mischen. Bei
der Angabe der vollständigen Summenformel hat sich die Notation Hg1-X
CdX Te etabliert. Daher findet man oft nur die Bezeichnung HgCdTe und
den zugehörigen Wert von x. Eine Reihe von Eigenschaften dieses neuen
ternären Kristalls folgt weitgehend einer Interpolation der Daten seiner
Ausgangsmaterialien, ihren Anteilen entsprechend. Dies trifft zum
Beispiel auf die Gitterkonstante (Kapitel 2.1) oder auf die Dichte zu
{[}35{]}: ρ = 8, 076 − 2, 23x(±0, 02)gcm−3

(5.2)

Auch die Bandlücke der Legierung folgt in erster Näherung einem solchen
linearen Zusammenhang. Zudem hat jedoch die Temperatur einen großen
Einfluss auf den Betrag der Bandlücke. In einer Auswertung von 22
Arbeiten erstellte Hansen et al {[}36{]} eine empirische Gleichung, die
heute am häufigsten genutzt wird:

EG = −0.302 + 1.93x − 0.81x2 + 0.832x3 + 5.35(1 − 2x)10−4 T

(5.3)

Zur Verdeutlichung dieses Zusammenhangs und der Temperaturabhängigkeit
dient Abbildung 5.1. So beträgt die Bandlücke für x = 0, 2 bei 80 Kelvin
nur die Hälfte des Wertes bei Raumtemperatur (80 meV statt 155 meV).
Dies ist insbesondere von Bedeutung, da Infrarotdetektoren oftmals
gekühlt werden, um das thermische Rauschen minimal zu halten. In den
letzten Jahren sind jedoch große Fortschritte gemacht worden, um
einerseits das Eigenrauschen zu minimieren und andererseits den
Kühlaufwand zu verringern. Da HgCdTe ein intrinsischer Halbleiter ist,
besitzt er seine hohe Quanteneffizienz nicht erst bei extrem geringen
Temperaturen, wie dies bei Quantenwells der Fall ist. Deshalb wird
dieses Material auch in Zukunft für Infrarotdetektoren unverzichtbar
bleiben. Übliche Zusammensetzungen für den mittleren bis langwelligen
Spektralbereich (3 - 14 µm) besitzen x-Werte zwischen 0,2 und 0,4. Mit
größeren x-Werten kann die Grenzwellenlänge auf bis zu 0,7 µm verringert
werden, entsprechend der Bandlücke von CdTe. Da die Bandlücke von HgTe
negativ ist, können theoretisch beliebig kleine Bandlücken erreicht
werden. In der Praxis wurde HgCdTe nicht nennenswert über 25 µm
Wellenlänge hinaus verwendet {[}37{]}.

Herstellungsverfahren

27

Abbildung 5.2: Abhängigkeit der Bandlücke und Gitterkonstante von der
Zusammensetzung für HgCdTe und CdZnTe Substrate. Zum Vergleich sind
einige III-V-Verbindungen angegeben {[}37{]}.

5.2

Herstellungsverfahren

Die Verfahren zur Herstellung von HgCdTe haben sich wie bei jedem
anderen Halbleiter weiterentwickelt. In den letzten 50 Jahren {[}1{]}
wurden vor allem die folgenden Kristallzüchtungsmethoden verwendet: •
Blockwachstum (bulk growth) -- Festkörper-Rekristallisation (SSR - solid
state recrystallisation, auch CRA - Cast Recrystallise Anneal {[}38{]}
oder QA - quench anneal {[}39{]}) -- Hochdruckverfahren nach Bridgman
{[}40{]} -- Zonenschmelzverfahren (traveling heater method {[}41{]},
traveling solvent) • Flüssigphasenepitaxie (LPE - liquid phase epitaxy
{[}42{]}) • Chemische Gasphasenepitaxie (CVD - chemical vapour
deposition; auch VPE - vapour phase epitaxy) -- Metallorganische
Gasphasenepitaxie (MOVPE - metal-organic VPE) -- Molekularstrahlepitaxie
(MBE - molecular beam epitaxy) -- Isotherme Gasphasenepitaxie (ISOVPE -
isothermal-VPE {[}43{]}) Die zeitliche Entwicklung der einzelnen
Methoden zeigt Abbildung 5.3. Die Kristallzucht mit dem modifizierten
Bridgman-Verfahren gehört keineswegs der Vergangenheit an, sondern ist
immer noch Bestandteil der aktuellen Forschung {[}44{]}. Im Folgenden
möchten wir uns nur mit Blockwachstum beschäftigen, da die untersuchten
Proben auf diese Weise hergestellt wurden {[}45{]}. Eine große
Schwierigkeit bei der Herstellung von Blockkristallen (als Ingot
bezeichnet, englisch auch boule) ist der hohe Dampfdruck des
Quecksilbers bei der Schmelztemperatur von

28

Das Material HgCdTe und Charakterisierung der Proben

Abbildung 5.3: Übersicht der unterschiedlichen Züchtungsmethoden für
HgCdTe und ihre zeitliche Entwicklung von 1958 bis heute. Die
überwiegend angewandte Produktionsmethode ist die Flüssigphasenepitaxie
(LPE, nach {[}37{]}).

HgCdTe (T ≈ 950 ◦ C). Die in der Schmelze stöchiometrisch vorliegenden
Ausgangsmaterialen werden sehr schnell verdichtet, damit es beim
Erstarren nicht zu einer Entmischung oder Abscheidung von HgTe kommt.
Ursache einer solchen Trennung sind die unterschiedlichen
Schmelztemperaturen von CdTe und HgTe. Dennoch ist die Konzentration von
HgTe im Kern dieser Blöcke etwas höher. Zudem sind diese Ingots hoch
polykristallin. Daher werden sie nach dem Erstarren nochmals
wärmebehandelt (anneal), also bis unterhalb des Schmelzpunktes erhitzt.
Auf diese Weise wächst die Korngröße bedeutend an. Bei dem
Bridgman-Verfahren entstehen ebenfalls polykristalline Ingots. Diese
weisen in Erstarrungsrichtung eine Variation der Komposition auf. Doch
wird dies genutzt, um in einem einzigen Wachstumsprozess
Ausgangsmaterialien für beide interessanten Wellenlängenbereiche (3 - 5
und 8 - 12 µm) herzustellen {[}46{]}. Die Korngrößen liegen bei beiden
Verfahren im Bereich von 50 - 500 µm {[}47{]}. Für die weitere
Verwendung werden die Blockkristalle oder Ingots in Scheiben von ca. 500
µm Dicke gesägt. Neben normalen Fadensägen kommen auch
elektropyrolytische Trennungsverfahren zur Anwendung. Danach werden sie
poliert und auf die gewünschte Dicke (oft 10µm) gebracht. Diese Schritte
sind jedoch sehr arbeitsintensiv und waren eine starke Motivation, nach
anderen Verfahren zu suchen. Die stark toxische Wirkung von
Quecksilberdämpfen hat außerdem in einer Vielzahl von Ländern zu
gesetzlichen Einschränkungen geführt, die ebenso die Kristallzüchtung
betreffen. Ein Wachstum auf einem Trägermaterial stellt hohe
Anforderungen an das Substrat, da die Qualität der aufgewachsenen
Schichten stark mit diesem korreliert ist. Die große Bedeutung der
Gitterfehlanpassung auf die Kristallqualität und Oberflächenmorphologie
der epitaktischen Schichten wurde vor 20 Jahren entdeckt. Ein gutes
Substrat für die meisten interessanten HgCdTe-Verbindungen war mit
Cd0.96 Zn0.04 Te gefunden (siehe Abbildung 5.2). Dieses Material wird
noch immer in sehr guter kristalliner Qualität mittels „bulk
growth``-Verfahren hergestellt. Der störende hohe Dampfdruck des
Quecksilbers tritt hier nicht auf.

Übersicht der untersuchten Proben

5.3

29

Übersicht der untersuchten Proben

Grundlage dieser Diplomarbeit, deren Thema 2004 ausgeschrieben wurde,
waren drei Kristalle von Hg1-X CdX Te mit x-Werten zwischen 0.07 und
0,4. Sie sind in Abbildung 5.4 mit I bis III gekennzeichnet. Diese
Proben wurden vom Institut für Tieftemperaturphysik an der Staatlichen
Moskauer Lomonossow-Universität zur Verfügung gestellt. Insbesondere
besteht eine enge Zusammenarbeit mit Dr.~Nikiforov, er hatte die
Beschaffung dieser Proben organisiert. Im Dezember 2006 wurden uns drei
weitere Proben zur Verfügung gestellt. Sie sind mit IV bis VI
bezeichnet.

Abbildung 5.4: Übersicht der sechs untersuchten Proben

Hg1-X CdX Te Wert für X Gewicht Orientierung

I 0.07 186.3 mg {[}111{]}

II 0.4 130.3 mg

III 0.2 158.5 mg {[}110{]}

IV 0.183 129.2 mg {[}110{]}

V 0.1955 248.5 mg {[}110{]}

VI 0.105 108.9 mg {[}110{]}

Tabelle 5.1: Einige Kenndaten der untersuchten Kristalle

Zu den Proben wurden nur wenige Daten mitgeliefert. Eine Übersicht mit
einigen Kenndaten ist in Tabelle 5.1 zu finden. Eine mitgelieferte
Angabe ist der charakteristische Wert für x in Hg1-X CdX Te. Allerdings
schwankt die Genauigkeit dieses Wertes von einer bis vier
Nachkommastellen, ein relativer Fehler ist nicht angegeben. Es lässt
sich leider nicht mehr rekonstruieren, auf welche Weise die Proben
charakterisiert wurden. Es ist weder der Zeitpunkt noch die Methode
bekannt, daher kann keine Aussage über die Genauigkeit dieser Angabe
gemacht werden. Die zweite Angabe betrifft die Orientierung, die zu fünf
Kristallen angegeben war. Jedoch fehlt auch hier ein Hinweis, auf welche
Fläche der Kristalle sich diese Angabe bezieht und ist daher wertlos.
Somit bestand eine der ersten Aufgaben darin, die Kristalle neu zu
charakterisie-

30

Das Material HgCdTe und Charakterisierung der Proben

ren. Die angegebenen Massen wurden mit einer genauen Waage des
Chemielabors bestimmt. In Summe basiert diese Diplomarbeit demnach auf
weniger als einem Gramm Ausgangsmaterial. Die einzelnen Proben besitzen
eine Dicke von 600 bis 900 µm. Ende 2007 wurden noch einmal in Moskau
die Kenndaten der Kristalle gesucht. Doch leider war auch ihr
Herkunftsort nicht mehr zu bestimmen. Mit großer Sicherheit wurden diese
Kristalle nicht in einem Institut gezüchtet, sondern in einer Fabrik,
die sich auf diese Materialien konzentriert hatte. Die Ergebnisse der
eigenen Züchtungsversuche im Moskauer Institut waren nicht
zufriedenstellend. Als Kristallzuchtverfahren wurde ein Blockwachstum
verwendet (siehe 5.2), wahrscheinlich das Bridgman-Verfahren (siehe
{[}44{]}, nach {[}45{]}).

5.4

Laue-Aufnahmen zur Kristallqualität

Ausgehend von den Besonderheiten des Herstellungsverfahrens der
HgCdTe-Proben ist es notwendig, die Kristallqualität zu prüfen.
Insbesondere kann nicht davon ausgegangen werden, dass es sich um
Einkristalle handelt. Vielmehr sollten polykristalline Proben mit
Korngrößen von einigen Mikrometern erwartet werden {[}47{]}. Die
Röntgenbeugung nach Laue ist gut geeignet, um die Kristallinität und
Orientierung eines ganzen Volumens zu bestimmen. Dazu wird ein Lauegramm
aufgenommen, also eine Transmissionsaufnahme erstellt. Damit jedoch auf
dem Film hinter der Probe eine ausreichende Intensität der gebeugten
Strahlung detektiert werden kann, müssen die Proben hinreichend dünn
sein. Mit der verwendeten Anlage (Kapitel 4.5) konnte noch eine
Probendicke von 600 µm untersucht werden.

Abbildung 5.5: Transmissionsaufnahme der Röntgenbeugung nach Laue von
Probe I (x=0,07). Die scharfen Beugungsmaxima in diesem Lauegramm deuten
auf eine homogene Orientierung der Kristalle hin. Die sichtbaren Ringe
hingegen erinnern an die Pulvermethode von Debye-Scherrer und zeigen
deutlich, dass kein Einkristall vorliegt. Diese Ringe werden an den
Gitterebenen der Korngrenzen in dieser polykristallinen Probe erzeugt.

Laue-Aufnahmen zur Kristallqualität

31

Das Ergebnis der Transmissionmessung zeigt Abbildung 5.5. Wie zu
erwarten enthält die Aufnahme ringförmige Schwärzungen von den
polykristallinen Anteilen der Probe. Es handelt sich demnach bei den zur
Verfügung gestellten Kristallen nicht um Einkristalle, wie es im Thema
der Diplomarbeit angegeben wurde. Allerdings besitzen die Körner eine
einheitliche Orientierung, wie aus den punktförmigen und relativ
scharfen Reflexen hervorgeht. Die weniger scharfen Punkte der intensiven
Reflexe ergeben sich aus der Dimension und Intensitätsverteilung des
kollimierten Röntgenstrahls. Die Reflexionsaufnahmen hingegen zeigen
eine sehr klare Struktur und weder Ringe von polykristallinen Anteilen
noch Punkte anderer Kristallorientierungen. Bei der Herstellung der
Proben wurden diese aus den Blockkristallen herausgesägt und die
Oberfläche poliert. So konnte der Röntgenstrahl der Röntgenapparatur mit
einem speziellen Mikroskop (Abbildung 4.5) exakt senkrecht zur
Oberfläche ausgerichtet werden. Die Orientierung der Kristallstruktur
ist nicht immer parallel zur Probenoberfläche, sondern auch gedreht
(Abbildung 5.6) oder um einen gewissen Winkel (Abbildung 5.7) verkippt.

Abbildung 5.6: Epigramm von Probe II (x=0,4). Deutlich ist die gedrehte,
doch gut zentrierte \{110\}-Orientierung zu erkennen.

Abbildung 5.7: Die Kristallstuktur von Probe V (x=0,1855) ist um 4 Grad
zur Oberflächennormalen verkippt (Orientierung \{110\}).

Mit den Laueaufnahmen wurde die Orientierung aller sechs Proben
überprüft. Auch diese Messungen wurden mit der Anlage in Moskau
durchgeführt. Die Aufnahmen wurden in 40mm Abstand mit 32 kV bei 30mA in
10 Minuten Belichtungszeit aufgenommen. Exemplarisch dienen die
Aufnahmen 5.6 und 5.7. Aus den entwickelten Photos konnte abgelesen
werden, dass die urspünglichen Blockkristalle in Längsrichtung
{[}111{]}-orientiert waren. Senkrecht dazu wurden sie in Scheiben
geschnitten. Die Oberflächen dieser Scheiben sind demnach (111)-Flächen.
Sie wurden poliert und bilden jeweils die Flächen unserer Proben mit den
größten Abmaßen. Bei unseren Proben handelt es sich somit um Bruchstücke
oder Restmaterial dieser Scheiben. Die Seitenflächen schließen einen
Winkel von 90◦ zur Oberfläche ein und sind (110)-orientiert. Da dies
auch die natürliche Spaltfläche von ZnS-Kristallen ist, lassen sich die
Proben parallel zu den Seitenflächen sehr gut spalten.

32

Das Material HgCdTe und Charakterisierung der Proben

5.5

Bestimmung der Zusammensetzung mittels energiedispersiver
Röntgenspektroskopie

Jeweils nach den Photoemissionsmessungen im März und August 2007 wurde
die Zusammensetzung der gemessenen Proben mittels energiedispersiver
Röntgenspektroskopie oder EDX1 untersucht. Diese Messungen wurden von
Dr.~Schäfer in unserem Institut durchgeführt. Das Ergebnis von 15
untersuchten Proben, die an jeweils drei Punkten der Probenoberfläche
spektroskopiert wurden, finden sich im Anhang A in den Tabellen A.1 und
A.2. Aus der Vielzahl der verfügbaren Daten wurde das statistische
Mittel gebildet und in Tabelle 5.2 zusammengefasst. Probe III wurde
weder per Photoemission noch per EDX untersucht, daher liegen hierzu
keine Daten vor. Probe I II III IV V VI

Referenzwert x 0.07 0.4 0.2 0.183 0.1955 0.105

Anteil Cd 0,063 ± 0,003 0,390 ± 0,005

Anteil Hg 1,031 ± 0,007 0,650 ± 0,005

Anteil Te 0,907 ± 0,005 0,957 ± 0,007

0,163 ± 0,015 0,160 ± 0,006 0,110 ± 0,007

0,92 ± 0,01 0,92 ± 0,01 0,978 ± 0,009

0,920 ± 0,005 0,920 ± 0,005 0,910 ± 0,006

Messungen 15 6 0 3 12 9

Tabelle 5.2: Ergebnisse der EDX-Messungen an fünf der verfügbaren
Proben. Die gemessenen relativen Intensitäten wurden auf 2 Atome pro
Einheitszelle renormiert.

Zunächst fällt der Tellur-Anteil in der Einheitszelle ins Auge.
Ausgehend von der Summenformel Hg1-X CdX Te sollte dieser stets 1
betragen. Jedoch wird er in EDX wiederholt und reproduzierbar zu gering
gemessen. Der statistische Mittelwert des Telluranteils aus 45 Messungen
beträgt 0,9176 ± 0,0005. Die Genauigkeit jeder Einzelmessung beträgt 8
\% relativen Fehlers für Tellur (siehe Anhang A) und genügt nicht, um
die Abweichung zu erklären. Offensichtlich liegt ein systematischer
Fehler vor. Mangels Vergleichsmessung kann nicht gesagt werden, ob
dieser Fehler durch die EDX-Anlage hervorgerufen wird oder die Proben
tatsächlich einen geringen Tellur-Anteil aufweisen. Allerdings geben
eine Vielzahl anderer Messungen mit der EDX-Anlage keinen Anlass zur
Beanstandung. Mit großer Wahrscheinlichkeit weisen daher die Kristalle
einen Mangel an Tellur auf. Die anderen Messwerte bestätigen recht gut
die angegebenen Daten der Zusammensetzung. Aufgrund der Normierung kann
der relative Anteil Cadmium direkt mit dem Referenzwert für x verglichen
werden. Doch wie bereits erwähnt wurde, ist die Methode, mit der die
Referenzwerte bestimmt wurden, unbekannt. Ein weiterer Vergleich mit den
Messwerten ist daher nicht sinnvoll. Zudem liegen keine Angaben zur
Genauigkeit der Referenzwerte vor.

5.6

Überprüfung der Kristalle mit dem Atomkraftmikroskop

Im Moskauer Institut ABMR (Advanced BioMedical Research) stand uns ein
Atomkraftmikroskop oder AFM2 zur Verfügung. So wurde die Oberfläche
aller sechs Proben auch mit dieser 1 2

EDX - Energy Dispersive X-ray spectroscopy AFM - Atomic Force Microscope

Überprüfung der Kristalle mit dem Atomkraftmikroskop

33

Methode untersucht. Die Messungen erfolgten an den (111)-Flächen, die
beim Zertrennen der Blockkristalle durch Sägen und Polieren präpariert
worden waren (siehe Kap. 5.4). In Abbildung 5.8 sieht man die parallenen
Strukturen, die beim Trennen der Kristalle mittels Elektroerosion
entstehen. Man erkennt in der linken Bildhälfte einen „Kanal`` von 3 µm
Breite und 50 nm Tiefe. Die verbliebenen Spuren erscheinen nur aufgrund
der starken Vergrößerung bzw. des kleinen Ausschnittes parallel, relativ
zum Blockkristall sind sie natürlich rund. Wenn wir nun einen Ausschnitt
von 3×3 µm näher betrachten, so erkennt man noch weitere feine parallele
Strukturen innerhalb des Kanals, die eindeutig mit der Säge erzeugt
wurden (Abbildung 5.9). Sie besitzen eine Höhe von 5-10 nm. Jedoch sind
zwischen diesen parallelen Strukturen deutlich kleine Erhebungen oder
„bumps`` zu sehen. Ihre Höhe beträgt 10-20 nm und ihr Durchmesser ca.
150 nm.

Abbildung 5.8: AFM-Bild der (111)-Oberfläche von HgCdTe. Sie wurde durch
Elektroerosion gesägt (parallele Strukturen).

Abbildung 5.9: Ausschnitt von 3 × 3 µm aus der linken Aufnahme. Deutlich
sind die Fehlstellen im „Graben`` zu erkennen.

Diese kleinen Erhebungen weisen auf Fehlstellen des Gitters hin. Anhand
der Fragmente der Oberflächenpräparation (parallele Strukturen von der
Säge) wird deutlich, dass diese Fehlstellen erst nach dem Zertrennen der
Blockkristalle an diese Stelle gelangt sind. Die Beweglichkeit von
Fehlstellen wurde von Spicer et. al.{[}48{]} berichtet. Er erwähnt, dass
sich diese Gitterfehler bereits bei Raumtemperatur im Kristall bewegen
können. Sie gelangen so aus dem Volumen bis an die Oberfläche. Die
Ursache dieser Beweglichkeit liegt in der schwachen HgBindung innerhalb
des Volumenkristalls. Ebenso schreibt er die hohe
Oberflächenempfindlichkeit gegenüber mechanischen Einflüssen dieser
schwachen Bindung zu. Auch dieser Aspekt wird mit einem
Atomkraftmikroskop deutlich. Bei vielen Messungen konnten wir kleine
Kratzer auf der Oberfläche entdecken. Als Beispiel diene hier Abbildung
5.8.

Kapitel 6

Präparation der (110) Oberfläche In Kapitel 3 wurde gezeigt, dass die
Photoelektronspektroskopie eine oberflächensensitive
Untersuchungsmethode ist. Ursache hierfür ist die geringe mittlere freie
Weglänge der Elektronen im Kristall (siehe Abbildung 3.3). In den
meisten Fällen soll die elektronische Struktur des Volumenkristalls
bestimmt werden. Dies setzt eine Oberfläche voraus, die sich möglichst
wenig vom Volumen unterscheidet. Eine unvermeidliche Veränderung im
Falle der II/VI-Halbleiter ist die Relaxation der (110)-Oberfläche
(siehe Kapitel 2.3). Darüber hinaus ist die Oberfläche unter normalen
Umgebungsbedingungen der Wechselwirkung mit den jeweiligen Gasen der
Luft ausgesetzt. Diese können die Oberfläche bedecken, physikalisch
verändern oder eine chemische Bindung eingehen. Ebenso können
Flüssigkeiten (z. B. Wasser) an der Oberfläche kondensieren. Für die
Untersuchung mittels ARPES ist es also notwendig, die Veränderungen der
Oberfläche zu entfernen. Weiterhin ist sicherzustellen, dass es nicht zu
erneuten Bedeckungen und Reaktionen kommt. Daher werden
Photoemissionmessungen unter UHV (Ultra High Vacuum, p \textless{} 10−9
mbar) durchgeführt. Anhand eines einfachen Modells lässt sich mithilfe
der kinetischen Gastheorie zeigen, dass unter UHV-Bedingungen Messungen
an der ungestörten Oberfläche von mehreren Stunden möglich sind. Zu
Beginn der Messungen ist deshalb eine saubere Oberfläche
sicherzustellen. In einigen Experimenten wird die zu untersuchende
Oberfläche direkt im Vakuum erzeugt (z. B. mittels MBE) und dann ohne
Unterbrechung des Vakuums zur UPS Messung transferiert. Andere
Herstellungsverfahren für die zu untersuchenden Materialien ermöglichen
diese Vorgehensweise leider nicht. In diesem Fall muss die Oberfläche im
Vakuum vor der eigentlichen Messung gereinigt oder präpariert werden.
Zwei häufig verwendete Verfahren werden im folgenden vorgestellt und auf
ihre Anwendbarkeit bei HgCdTe geprüft.

6.1

Sputtern und Annealen

Bei dem Verfahren des Sputterns (aus dem englischen to sputter =
zerstäuben) werden Atome aus dem Festkörper (Target) durch Beschuss mit
energiereichen Ionen herausgelöst. Für den Beschuss eignen sich
besonders Edelgase wie Argon, da sie keine chemische Reaktion mit dem
Target eingehen. Die chemische Zusammensetzung des Targets wird daher
durch das Sputtern nicht verändert. Die in einem Niederdruckplasma
generierten positiven Edelgasionen werden durch eine Gleichspannung zum
Target beschleunigt. Die herausgeschlagenen Atome des Targets gehen in
die Gasphase über und kondensieren auf den Wänden der Vakuumkammer oder

34

Sputtern und Annealen

35

werden durch das Pumpsystem aus dem System entfernt. Durch den
Sputterprozess werden die obersten Schichten des Targets abgetragen.
Dabei entsteht eine sehr raue Oberfläche. Für die Messungen muss diese
daher wieder geglättet werden. Dazu wird die Probe auf eine Temperatur
nahe unter der Schmelztemperatur erwärmt. Durch die Oberflächenspannung
des Probenmaterials glättet sich die Oberfläche wieder. Dieser Prozess
wird Annealen (engl. für tempern, anlassen) genannt. Währenddessen
erhöht sich der Druck in der Vakuumkammer signifikant, da der partielle
Dampfdruck aller Elemente mit der Temperatur ansteigt. Das Restgas setzt
sich allerdings fast ausschließlich aus den Elementen der Probe
zusammen, die mit der erwärmten Probenoberfläche in einem dynamischen
Gleichgewicht steht. Durch die Oberflächenspannung des Probenmaterials
glättet sich die Oberfläche wieder. Die Kombination aus Sputtern und
Annealen wird häufig bei der Untersuchung von Metalloberflächen
angewendet. Die Kombination dieser beiden Methoden kann oftmals nicht
angewendet werden, wenn sich die Schmelztemperaturen der einzelnen
Komponenten des zu untersuchenden Materials stark voneinander
unterscheiden. Im Falle von HgCdTe betragen die Unterschiede fast 500
Kelvin. Cadmium besitzt eine Schmelztemperatur von 594 K (321 ◦ C) und
Tellur von 723 K (450 ◦ C). Quecksilber hingegen geht bereits bei 234 K
(-39 ◦ C) in die flüssige Phase über. Die Siedetemperatur von
Quecksilber liegt bei 630 K (357 ◦ C), noch unterhalb der
Schmelztemperatur von Tellur. Man sollte erwarten, dass das Quecksilber
aufgrund seines hohen Dampfdruckes aus einer erwärmten Probe
herausdiffundiert {[}49{]}. Bei Raumtemperatur hingegen ist HgCdTe
stabil und behält seine Komposition bei. Die Schmelztemperatur für MCT1
liegt je nach Zusammensetzung zwischen der von HgTe (670 ◦ C) und CdTe
(1090 ◦ C, nach {[}46{]}, Seite 56).

Abbildung 6.1: Mikrosondenanalyse einer erwärmten HgTe Oberfläche
(rechte Bildhälfte) {[}50{]}. In der jeweils linken Bildhälfte ist der
Sputterkrater zu sehen, dort wird der Sollwert des Hg-Anteils gemessen.
In der linken Abbildung wurde die Oberfläche auf Tellur untersucht,
rechts auf Quecksilber.

Eine experimentelle Untersuchung in unserer Arbeitsgruppe {[}50{]}
zeigt, dass die Methode des Annealens bei HgCdTe tatsächlich nicht
angewendet werden kann. Es wurde die Oberfläche von HgTe untersucht. Die
in dieser Diplomarbeit untersuchten Proben haben einen relativ hohen
Anteil Quecksilber, so dass sich die Ergebnisse von HgTe direkt
übertragen lassen. Zunächst wurde eine Probe von HgTe erhitzt. Dann
wurde ein Teil der Oberfläche durch Sputtern mit Argonionen gereinigt.
Dadurch wurde eine Schicht von einigen Nanometern abgetragen und somit
eine durch das Annealen veränderte Oberfläche entfernt. 1

Mercury Cadmium Telluride, übliche Abkürzung für HgCdTe in
englischsprachiger Literatur

36

Präparation der (110) Oberfläche

Danach wurde die Oberfläche mit der Mikrosondenanalyse untersucht. Das
Ergebnis ist in Abbildung 6.1 zu sehen. Bei dieser Untersuchung erfolgte
die Anregung mit 10 keV Röntgenstrahlen. Die Auflösung der
Elektronenstrahlmikrosonde beträgt 2 µm, die Bilder haben eine
Kantenlänge von 180 µm. Die rechte Bildhälfte zeigt die erwärmte
HgTe-Oberfläche. Es ist deutlich zu sehen, dass die Probe fast kein
Quecksilber mehr an der Oberfläche enthält. In der linken Bildhälfte
erkennt man den Sputterkrater. Dort messen wir wieder das erwartete
Verhältnis von Quecksilber zu Tellur. Damit eignet sich die Methode des
Sputterns und Annealens nicht, um die Oberfläche der zu untersuchenden
Proben von Hg1-X CdX Te für eine ARPES-Messung vorzubereiten. Das ist in
Übereinstimmung mit den Ergebnissen von Spicer et. al.~{[}48{]},
{[}51{]}.

6.2

Spalten der Proben im Vakuum

Eine weitere Möglichkeit, eine saubere Oberfläche im Hochvakuum zu
erzeugen, ist das Spalten der Kristalle. Besonders Schichtkristalle sind
für diese Methode der Oberflächenpräparation geeignet. Da ihre Ebenen
nur mit einer schwachen Van-der-Waals-Bindung zusammengehalten werden,
ist die zur Spaltung benötigte Kraft sehr gering. Auf diese Weise werden
zum Beispiel die Oberflächen von Hochtemperatursupraleitern für
Messungen im Vakuum vorbereitet. Dazu genügt es, Tesafilm auf die Probe
aufzukleben und diesen im Vakuum abzuziehen. Die kovalenten Bindungen
innerhalb der Ebenen sorgen für die nötige mechanische Stabilität des
Restkristalles. Man erhält sehr ebene Oberflächen von wenigen Angström
Rauhigkeit (siehe {[}52{]}, Seite 43). Neben der
Photoelektronenspektroskopie benötigen auch Untersuchungen mit dem
Rastertunnelmikroskop oder STM2 saubere Oberflächen. Bei Kristallen, die
durch kovalente oder ionische Bindungen zusammengehalten werden, gelingt
das Spalten jedoch nicht so einfach. Es ist notwenig, eine spezielle
Spaltvorrichtung zu konstruieren (siehe {[}53{]}, {[}54{]} und
{[}55{]}). Dr.~C. Jannowitz hat eine solche Spaltkammer zur Untersuchung
der elektronischen Struktur von CdTe konstruiert. Im Rahmen seiner
Doktorarbeit konstruierte auch N. Orlowski eine Spaltkammer, um die
Oberflächen von HgTe für die Untersuchung mittels PES vorzubereiten.
Daher wurde erneut ein Spaltmechanismus konstruiert, um die Proben von
HgCdTe für die Photoemissionsmessung zu präparieren.

6.2.1

Konstruktion einer Spaltkammer

In Abbildung 6.2 sind die Spaltkammer von N. Orlowski und die neu
aufgebaute Spaltkammer zu sehen. Aus früheren Spaltversuchen war
bekannt, dass nur gekühlte Spaltungen erfolgreich sind. Eine Anforderung
an die Spaltkammer war daher die Möglichkeit, die Probe vor der
eigentlichen Spaltung hinreichend zu kühlen. Der Ansatz ist in Bild (c)
zu sehen. Im Zentrum ist eine Adapterplatte zu erkennen, auf die ein
Probenhalter geschraubt werden kann. Für einen optimalen thermischen
Kontakt ist die Oberfläche poliert. Diese Platte bildet den Abschluss
eines Edelstahlrohres, das mit einem CF63-Flansch in einen Manipulator
geschraubt ist und oben eine Öffnung zum Einfüllen einer Kühlflüssigkeit
bietet. Das Rohr wurde von der Werkstatt für UHV-Umgebungen passend
verschweißt. Von links ragt ein Amboss in die Vakuumkammer hinein. Er
ist auf einer schraubbaren Lineardurchführung befestigt. Ihm gegenüber
ist der eigentliche Spaltkeil auf einer frei beweg2

Scanning Tunnel Microscope

Spalten der Proben im Vakuum

37

Abbildung 6.2: Links die Spaltkammer von N. Orlowski {[}7{]}, in der
Mitte die im Rahmen dieser Diplomarbeit gebaute Spaltkammer. Rechts ein
Blick in die Spaltkammer: Links ist der Amboss zu erkennen, rechts der
Spaltkeil und oben die Aufnahmeplatte für den Probenhalter.

lichen Lineardurchführung befestigt. Der Manipulator erlaubt eine exakte
Positionierung der Probe in x−, y− und z-Richtung. Am Probenhalter kann
ein PT100 zur Temperaturmessung befestigt werden. Mit einer
vergleichbaren Apparatur von N. Orlowski wurde bei Kühlung mit flüssigem
Stickstoff innerhalb weniger Minuten eine Temperatur von 80K erreicht.
Der direkte und großflächige Kontakt zwischen Kühlbehälter und
Probenhalter bietet demnach deutliche Vorteile gegenüber einer Kühlung
in einem gewöhnlichen Manipulator. Bei letzterem erfolgt der
Wärmetransport vornehmlich über Kupferlitze. Der Kühlmittelbedarf ist
ebenfalls bedeutend höher {[}7{]}.

6.2.2

Spaltmechanismus im Probenhalter

Es stellte sich heraus, dass die von Moskau zur Verfügung gestellten
Proben nur sehr klein waren. Insbesondere wiesen sie eine Dicke von
weniger als einem Millimeter auf. Dieser Umstand stellt besondere
Anforderungen an die genaue Positionierung der Proben im
Spaltmechanismus. Während der Präparation zeigte sich, dass sie aufgrund
ihrer geringen Größe mechanisch nicht sehr stabil sind. Das bedeutet,
dass die zur Spaltung notwendige Kraft trotz der kovalenten Bindungen
recht gering ist. Es wurde daher der Versuch unternommen, die Kristalle
mit einer bei Schichtkristallen üblichen Methode zu spalten. Dazu werden
die Kristalle mit leitfähigem Silberkleber in einem speziell
vorbereiteten Stempel des Probenhalters geklebt. Dieser härtet bei 120 ◦
C Wärmebehandlung innerhalb einer Stunde aus. Danach wird ein passender
Spalthebel auf die Oberseite der Probe geklebt. Zuvor wird eine Nut von
1 Millimeter Breite in den Aluminiumspalthebel gesägt, um ihn während
des Aushärtens besser auf der Probe zu positionieren. Darüber hinaus
steht so eine größere Fläche für die Kraftübertragung beim eigentlichen
Spaltprozess zur Verfügung (siehe Abbildung 6.3). Die so präparierten
Proben wurden über das Probenkarussell in das Vakuumsystem transferiert.
Mit dem Wobbelstick konnten die Proben sehr einfach gespalten werden,
während sie sich im Kryostatmanipulator (Kapitel 4.1) befinden. Die
benötigten Kräfte sind so gering, dass

38

Präparation der (110) Oberfläche

einige Proben bereits beim Einbau gespalten wurden (siehe Anhang A).
Daher wurden die Spalthebel verkleinert, um die Kräfte auf die Proben
durch die Massenträgheit des Hebels zu minimieren.

Abbildung 6.3: Schematische Darstellung der Spaltmechanik direkt im
Probenhalter. Rechts ist ein Probenhalter mit Probe und Spaltkeil
photographiert.

Allerdings spalteten die Kristalle an unvorhersagbaren Stellen und
neigten zur Splitterbildung. Daher wurden die Kristalle an Hr. Sölle
gegeben, der sie in kleinere Einzelkristalle zersägte. Außerdem sägte er
mit der Fadensäge einen 100 µm tiefen Graben in eine Seite der
Kristalle. Dies sollte als Sollbruchstelle dienen. Schematisch ist der
finale Aufbau zum Spalten der Probe in Abbildung 6.3 gezeigt. In der
Praxis ließen sich die Kristalle erfolgreich entlang dieser
Sollbruchstelle teilen. Eine Aufnahme ist in Abbildung 6.4 links zu
sehen.

Abbildung 6.4: Die linke Aufnahme mit einem Rasterelektronenmikroskop
zeigt eine Spaltung entlang der Sollbruchstelle. Diese ist als 100 µm
tiefer Graben im Vordergrund zu erkennen. Das mittlere Bild zeigt die
fragmentierte Oberfläche nach einer Spaltung im Vakuum. Rechts ist die
Photographie eines bei der Spaltung zersplitterten Kristalls zu
erkennen.

6.3

Aufnahmen der Spaltungen mit dem Rasterelektronenmikroskop

Die weiteren Messungen mit dem Rasterelektronenmikroskop oder SEM3 an
den gespaltenen Proben legen deutlich offen, dass die Proben keinesfalls
einkristallinen Charakter besitzen. Dies wurde schon durch die
Laue-Transmissionsaufnahme (Abbildung 5.5) gezeigt. Makroskopisch weisen
die Proben eine einheitliche Orientierung auf, sind aber in jedem Fall
polykristallin. Darauf weisen auch die beiden Aufnahmen des
Elektronenmikroskopes in Abbildung 6.5 hin. 3

Scanning Electron Microscope

Prüfung der Oberflächenqualität durch die Beugung langsamer Elektronen

39

Die linke Aufnahme zeigt das gespaltete Kristall im Probenhalter. Für
das rechte Bild wurde ein Ausschnitt um den Faktor 10 vergrößert.

Abbildung 6.5: Elektronenmikroskopaufnahme einer gekühlten Spaltung
einer HgCdTe-Probe. Deutlich sind Stufen von einigen µm Höhe zu
erkennen. Das Kristall zeigt innere Spannungen. Der Maßstab der Aufnahme
beträgt links 1 mm, rechts 100 µm.

Schon bei der Auflösung des Elektronenmikroskopes wird eine
uneinheitliche Oberflächenstruktur sichtbar. Man kann nicht von einer
homogenen Spaltung entlang einer (110)-Ebene sprechen. Vielmehr sind
Stufen von einigen Mikrometern Höhe erkennbar sowie innere Spannungen
des Kristalls. Diese Spannungen haben den Verlauf der Spaltung
entscheidend bestimmt. Bei der Interpretation winkelaufgelöster
Messungen von Photoelektronenspektren benötigt man natürlich eine
Referenzebene. Offensichtlich ist diese nur bedingt gegeben, man sollte
in den Spektren Anteile unterschiedlicher Oberflächenorientierungen
erwarten. Neben Effekten der Oberflächen-Relaxation werden zusätzliche
Anregungen von weiteren Oberflächenzuständen zu sehen sein. Die
Anforderungen an die Oberflächenqualität für eine Photoemissionsmessung
gehen über den Mikrometerbereich weit hinaus. Es sind gute Eigenschaften
auf atomarer Ebene erforderlich. Diese werden mit der nächsten Methode
überprüft.

6.4

Prüfung der Oberflächenqualität durch die Beugung langsamer Elektronen

Eine übliche Methode, periodische Strukturen (z. B. Kristalloberflächen)
auf atomarem Niveau zu untersuchen, ist die sogenannte Beugung langsamer
Elektronen oder LEED4 . Die wenig verheißungsvollen Ergebnisse des
Elektronenmikroskopes bestätigten sich auch bei der Untersuchung mittels
LEED. Mehrere Oberflächen wurden im Vakuum präpariert, indem sie unter
normalen Bedingungen gespalten wurden. Mit diesen Probenoberflächen war
es nicht möglich, ein Beugungsbild auf dem Phosphorschirm zu erhalten.
Aus der Untersuchung ähnlicher Kristalle {[}7{]} war bekannt, dass eine
gekühlte Spaltung weitaus bessere Ergebnisse liefern kann. So wurde eine
Probe im Kryostat-Manipulator der AR65 auf 100 K (-170 ◦ C) gekühlt und
mit dem Wobbelstick gespalten. Danach war es erstmals 4

Low Energy Electron Diffraction

40

Präparation der (110) Oberfläche

möglich, das typische Beugungsmuster einer LEED-Aufnahme zu erkennen.
Bei einer Anregungsenergie von 156 eV entstand das Bild, welches in
Abbildung 6.6 gezeigt ist.

Abbildung 6.6: Erstes erfolgreiches LEED-Beugungsbild nach einer
gekühlten Spaltung bei 100 Kelvin. Die Elektronenenergie beträgt 156 eV.

Aus dem Beugungsbild kann man die Periodizität des Oberflächengitters
berechnen. Für die Berechnung benötigt man außerdem die Geometrie der
Apparatur sowie die Elektronenenergie. Es sind zwei unterschiedliche
Gitterkonstanten zu erkennen, obwohl für HgCdTe stets nur eine
Gitterkonstante angegeben wird. Das liegt daran, dass eine
(110)-Oberfläche untersucht wurde. Da diese Ebene nicht parallel zu
Flächen des Basisgitters verläuft, sondern dia√ gonal durch diese, ist
eine Gitterkonstante um den Faktor 2 verkürzt. Die Relationen der
Oberflächenperiodizität zur Gitterkonstante betragen: 1 b1 = a 2

1 b2 = √ a 8

Mit der bekannten Anregungsenergie von 156 eV und der Geometrie der
LEED-Apparatur können die zugehörigen Längenwerte für b1 und b2 aus
Abbildung 6.6 berechnet werden: b1 = (3, 5 ± 0, 2)Å

b2 = (2, 4 ± 0, 2)Å

Aus den Werten b1 und b2 ergibt sich eine Gitterkonstante von 7,0 Å bzw.
6,8 Å. Der Sollwert für diese Probe (x=0,2) beträgt a = 6, 464 Å. Die
Ergebnisse der Beugung langsamer Elektronen weichen somit um 10 \% von
den realen Werten ab. Bei der Berechnung hat die Geometrie bzw. deren
exakte Einhaltung einen großen Einfluss auf das Endergebnis. In unserem
Experiment genügt bereits eine Abweichung der Probenposition um 7 mm vom
Zentrum des Phosphorschirmes für einen relativen Fehler von 10 \%
{[}23{]}. In Abbildung 6.6 ist zu erkennen, dass die Beugungsreflexe
nicht exakt auf einer Linie liegen. Offensichtlich befindet sich die
Probe also nicht an der Referenzposition. Mit dem Manipulator der AR65
kann die Probe sehr genau relativ positioniert werden, über die absolute
Positionierung kann jedoch keine Aussage gemacht werden. Eine weitere
Ungenauigkeit liegt in der Höhe des Probenstempels sowie der Position
des Kristalls in diesem.

Kapitel 7

Ergebnisse der Photoemission 7.1

Energie der gemessenen Zustände

In typischen Photoemissionsexperimenten bestimmt ein Analysator die
Intensitäten der emittierten Elektronen bei verschiedenen Energien. Ein
angeregter Zustand des Festkörpers erscheint in diesen Spektren bei
einer definierten kinetischen Energie. Der absolute Wert der gemessenen
kinetischen Energie für diesen Zustand ist allerdings von der
Anregungsenergie der Photonen abhängig (siehe Abbildung 3.2). In der
Praxis werden daher diese Energien relativ zu einem gemeinsamen
Bezugspunkt angegeben, um Spektren und Energiewerte leichter miteinander
vergleichen zu können. Damit befinden sich gleiche Zustände in
unterschiedlichen Messungen bei unterschiedlichen Photonenenergien bei
der gleichen relativen Energie. In den üblichen Arbeiten zur
Photoemission finden insbesondere zwei Referenzpunkte Verwendung. In
vielen Fällen werden die kinetischen Energien auf die Fermi-Energie
bezogen. Bei Halbleitern ist es auch üblich, sich auf das
Valenzbandmaximum (VBM) zu beziehen. Es ist mit einigen Besonderheiten
verbunden diese Bezugspunkte zu bestimmen. Der Energieanalysator ist
innen mit Graphit beschichtet (siehe Seite 13). Normalerweise würde man
erwarten, die Fermi-Kante stets bei exakt derselben Energie im Abstand
von der Photoenenergie zu messen. Ausgehend von der einfachen Beziehung
(3.1) erwartet man daher: EF = hν − ΦGraphit = hν − 4, 14 eV In der
Realität stimmt diese Gleichung jedoch nicht mit den Messergebnissen
überein. Die Spannungsversorgungen der Elektronenlinsen und des
Halbkugelanalysators besitzen eine Spannungsstabilität von wenigen µV.
Dennoch sind die gemessenen Energiewerte von den Umgebungsbedingungen
abhängig. Diese driften im Verlauf von Tagen um einige Millivolt. Daher
wird bei hochauflösenden Messungen, wie sie bei
Hochtemperatursupraleitern durchgeführt werden, jedes Mal vor oder nach
der Messung die exakte Position der Fermi-Energie bestimmt. Diese
Kalibrierung sollte auch bei weniger genauen Messungen regelmäßig
durchgeführt werden. Unbedingt erforderlich ist eine solche Messung,
wenn die Anlage abgebaut und an anderer Stelle (z. B. am Synchrotron)
wieder aufgebaut wird. So wurde im Frühjahr 2007 mit der Heliumlampe (hν
= 21, 22 eV) eine Fermi-Energie von EF = (17, 085±0, 004) eV bestimmt.
Daraus ergibt sich eine gemessene Austrittsarbeit Φ = (4, 135 ± 0, 005)
eV. Die Messungen im

41

42

Ergebnisse der Photoemission

August 2007 bei BESSY ergaben jedoch eine Austrittsarbeit von Φ = (4,
180 ± 0, 005) eV. Diese Angaben beziehen sich auf eine Passenergie von
10 eV. Für eine Passenergie von 5 eV verschiebt sich die Fermi-Kante um
weitere 130 meV (Messung Januar 2007). Ein großer Vorteil der Messung an
der Beamline eines Synchrotrons ist die einstellbare Photonenenergie.
Aus dem kontinuierlichen Spektrum der Synchrotronstrahlung wird mittels
eines Gittermonochromators die gewünschte Wellenlänge ausgewählt. Die
relativen Energiepositionen des Monochromators werden reproduzierbar mit
einer Genauigkeit von 5 meV angefahren {[}56{]}. Für die
BUS-XUV-Beamline wurde diese Angabe im Jahre 2007 nochmals überprüft.
Allerdings bezieht sich diese Genauigkeit nicht auf die absoluten
Energiewerte. So wurde im Januar 2007 mit der AR65 eine Austrittsarbeit
von Φ = (3, 92 ± 0, 03) eV bestimmt. Dieser Wert unterscheidet sich um
260 meV von dem Ergebnis im August desselben Jahres.

Abbildung 7.1: Gemessene Austrittsarbeit der AR65 in Abhängigkeit von
der Anregungsenergie. Sie wurde durch die Lage der Fermi-Kante von
polykristallinem Gold bei verschiedenen Anregungsenergien im Januar 2007
bei BESSY gemessen. Mittelwert: Φ = (3, 92 ± 0, 03) eV.

Die gemessene Fermi-Energie ist außerdem sehr stark von der
Photonenenergie abhängig (siehe Abbildung 7.1), einzelne Werte weichen
um 100 meV ab. Für die Auswertung der Spektren haben wir daher den Wert
verwendet, der bei der jeweiligen Photonenenergie empirisch bestimmt
worden war. Eine Vielzahl von Arbeiten bezieht ihre Energieangaben auf
das Valenzbandmaximum. Bei Halbleitern tritt dieses Maximum jedoch nur
am Γ-Punkt auf. Um diesen Energiewert bestimmen zu können, muss die
exakte Position des Γ-Punktes bekannt sein. Außerdem muss es möglich
sein, diesen Punkt mit der Messung zu erreichen. Diese Bedingung
schließt sowohl den richtigen Winkel relativ zur Oberfläche als auch die
passende Anregungsenergie ein. Durch die Dispersion der Energiebänder
liegen die Maxima von anderen Punkten der Brillouin-Zone bei niedrigeren
Energien. (siehe Abbildung 7.14). Dieser Abstand kann mehr als ein
Elektronenvolt betragen. Bei Materialien mit einer geringen effektiven
Masse der Löcher ist die Position des Γ-Punktes sehr scharf begrenzt.
Bei den Messungen an HgCdTe ist es nur sehr selten gelungen, eine klar
definierte Kante als Maximum des Valenzbandes zu messen. Als Beispiel
diene die Messung vom 16. 7. 2007 in Abbildung 7.2 an Probe I (x=0,07).
Es ist die erste Messung mit der neuen Stickstoff-

Energie der gemessenen Zustände

43

Nachfüllanlage. Daher war das Spektrum mit unzähligen „Spikes`` von den
Schaltvorgängen des Magnetventils übersät. Der typische inelastische
Untergrund hingegen trat vergleichsweise schwach in Erscheinung. Die
sichtbare Kante liegt bei einer Energie von -0,23 eV relativ zur
Fermi-Energie. Mit der Anpassung an eine Gaussfunktion lässt sich dem
Valenzbandmaxium ein Zustand bei -0,44 eV zuordnen.

Abbildung 7.2: Aus diesem Spektrum wurden nur die Spikes entfernt.
Deutlich ist ein Valenzbandmaximum erkennbar. Die Anregung erfolgte mit
der He-Lampe. Es wurde Probe I (x=0,07) untersucht.

Bei den meisten anderen Messungen war es nicht möglich, einen
Energiewert für das Valenzbandmaximum zu bestimmen. Zwar war es möglich,
eine gut sichtbare Dispersion der Volumenzustände zu detektieren. Das
trifft zum Beispiel auf die k⊥ -Messung an Probe I (x=0,07) zu, die in
Abbildung 7.14 gezeigt ist. Aufgrund der Dispersion lässt sich der
Γ-Punkt der Energie von 115 eV zuordnen, der X-Punkt liegt bei 65 eV.
Doch im Gegensatz zu Abbildung 7.2 kann am Γ-Punkt kein klares Maximum
ausgemacht werden. Einen Ausschnitt des Spektrums an diesem Messpunkt
zeigt Abbildung 7.3.

Abbildung 7.3: Valenzband HgCdTe: Intensität der Photoemission beim
Γ-Punkt (Anregungsenergie 115 eV).

Wie deutlich erkennbar ist, nimmt die Zustandsdichte oberhalb von 0,9 eV
Bindungsenergie stetig ab, um bei der Fermi-Energie (hier als 0
definiert) ihr Minimum zu erreichen. Im weiteren

44

Ergebnisse der Photoemission

ist daher bei allen Auswertungen von Spektren die zuvor bestimmte
Energie der Fermi-Kante als Referenzwert („Null``) angesetzt. Aus
anderen Arbeiten geht hervor, dass auch dort die Relativierung der
kinetischen Energien über die Fermi-Energie erfolgt. Zuvor wurde die
Energie des Valenzbandmaximums (VBM) relativ zu dieser Energie bestimmt.
Dieser Abstand entspricht einer konstanten Energie, die der
Fermi-Energie abgezogen werden kann. In der Auswertung der Spektren wird
dann das VBM als Nullpunkt gesetzt. Der Nullpunkt kann damit durchaus
ein Elektronenvolt unterhalb der Fermi-Energie liegen. Somit könnten
auch Zustände mit positiver Bindungsenergie auftauchen ({[}7{]}, Seite
46). Das ist aber kein Widerspruch, die Energie liegt noch deutlich
unterhalb der Fermi-Energie. Es handelt sich um Zustände in der
Energielücke, die zum Beispiel von Oberflächenzuständen (siehe Seite 6)
hervorgerufen werden.

7.2

Allgemeine Charakteristika

Bei unseren Messungen an der BUS-XUV-Beamline bei BESSY standen uns
Photonenenergien bis zu 125 eV zur Verfügung (Gitter 1 - 500 Linien).
Mit dieser Anregungsenergie haben wir Probe IV (x=0,183) bestrahlt. Die
emittierten Elektronen besitzen ein für HgCdTe typisches Spektrum, wie
es in Abbildung 7.4 gezeigt wird. Auf die einzelnen Bestandteile möchten
wir zunächst kurz eingehen. In diesem Übersichtsspektrum erkennt man
zunächst die Spin-Bahn-aufgespaltenen Kernniveaus des Tellur bei ca. 40
eV Bindungsenergie. Das Te4d5/2 -Level liegt bei 40,1 eV, während das
Te4d3/2 -Level um 1,5 eV abgespalten bei 41,6 eV liegt. Der Unterschied
in den Intensitäten entspricht der Besetzung der jeweiligen Orbitale mit
10 bzw. 6 Elektronen. Dieser Energiebereich ist noch einmal extra in
Abbildung 7.5 gezeigt. Sehr nahe am Valenzband befinden sich weitere
Kernniveaus des Quecksilber und Cadmium. In den untersuchten Proben
dominierte der Quecksilberanteil. Die 5d-Level sind ebenfalls Spin-Bahn
aufgespalten und befinden sich bei den Energien E(Hg5d5/2 ) = 8,27 eV
bzw. E(Hg5d3/2 ) = 10,08 eV. Auch hier ergeben sich die relativen
Intensitäten aus der Besetzung der Orbitale. Mit steigendem
Cadmium-Anteil sind auch die Cd4d-Level im Spektrum sichtbar. Ihre
Energien betragen E(Cd4d5/2 ) = 10,92 eV bzw. E(Cd4d3/2 ) = 11,49 eV.
Das Photoemissionsspektrum bis zu 13 eV Bindungsenergie ist in Abbildung
7.6 genauer dargestellt. Diese Messung erfolgte an Probe V (x=0,1955).
Im Energiebereich von 4 bis 6,5 eV überlagern sich s-artige {[}48{]}
Valenzelektronen vom Quecksilber und Cadmium. Der Zustand des Cadmium
besitzt eine Energie von 5,17 eV und das Quecksilber-Level 5,95 eV. Die
Lage sowie der Abstand von 780 meV stimmen mit den Ergebnissen von
Silberman et. al.~{[}53{]} überein. Anzumerken ist, dass sich diese
Arbeit auf das VBM bezog und daher Energiewerte erhält, die 0,5 eV
geringer sind. Im Bereich zwischen 4 eV Bindungsenergie und der
Fermi-Energie liegen p-artige Zustände {[}48{]}, {[}57{]}, die eine
starke Dispersion aufweisen (siehe Abbildung 7.14). Die Messung in
normaler Emission wird sich hauptsächlich auf diesen Bereich
konzentrieren.

Allgemeine Charakteristika

45

Abbildung 7.4: Übersichtsspektrum einer Probe HgCdTe mit 125 eV
Anregungsenergie bei BESSY.

Abbildung 7.5: Durch Spin-BahnKopplung aufgespaltene 4d-Kernniveaus des
Tellur.

Abbildung 7.6: Valenzbandspektrum von Probe V (x=0,1955) mit den
Kernniveaus von Quecksilber und Cadmium.

46

Ergebnisse der Photoemission

7.3

Energie der Kernniveaus

7.3.1

Tellur 4d

Kernniveaus zeichnen sich durch eine hohe Zustandsdichte aus, daher sind
die gemessenen Elektronenintensitäten bei diesen Energien besonders
hoch. Das Tellur-4d-Niveau bei 40 eV Bindungsenergie ist ein solcher
charakteristischer Zustand. Während der Messungen am Synchrotron war
dieses Niveau gut geeignet, eine optimale Position der zu untersuchenden
Probe in der Messapparatur zu finden.

Abbildung 7.7: Doppelstruktur des Te4dKernniveaus bei einer Messung von
Probe I (x=0,07). Offensichtlich liegt eine veränderte chemische
Umgebung vor.

Abbildung 7.8: Spin-Bahn-aufgespaltenes Kernniveau Te4d von Probe IV
(x=0,183) in ungestörter Umgebung. Die Energie der Aufspaltung beträgt
(1,46±0,01) eV.

Neben der Positionierung eignen sich die Te4d-Kernniveaus auch, um die
chemische Umgebung zu überprüfen. Bei einer Messung (Abbildung 7.7)
zeigten sich eine Überlagerung von zwei Spin-Bahn-aufgespaltenen
Kernniveaus mit einer Energiedifferenz von ∆=(0,53±0,02) eV. Dieser
sogenannte „chemical shift`` deutet auf eine veränderte chemische
Umgebung hin {[}58{]}. Die untersuchte Probe weist also keine idealen
Eigenschaften auf. In manchen Fällen ist eine solche Verdopplung ein
Hinweis auf eine Gitteränderung (z. B. β-MoTe2 ).

7.3.2

Quecksilber 5d und Cadmium 4d

Bereits in früheren Photoemissionsmessungen von Shih et. al.~wird von
einer Verschiebung der Kernniveaus Hg5d und Cd4d in Abhängigkeit von der
Zusammensetzungs beobachtet {[}54{]}. Auch unsere Messungen zeigen
unterschiedliche energetische Lagen der Kernniveaus in den verschiedenen
Proben. Einige Messwerte sind in Tabelle 7.1 zusammengetragen. X 0,40
0,18 0,07 0,06

Hg5d5/2 8,35 8,30 8,28 8,05

Hg5d3/2 10,15 10,09 10,04 9,87

Cd4d5/2 10,68 10,69 10,66

Cd4d3/2 11,37 11,38

Tabelle 7.1: Lage der Kernniveaus von Cadmium (Cd4d) und Quecksilber
(Hg5d) in Abhängigkeit von der Komposition. Der absolute Fehler beträgt
10 meV.

Energie der Kernniveaus

47

In unseren Messungen beobachten wir mit steigendem Cd-Anteil eine
Verschiebung des Hg5d-Niveaus um 300 meV zu größeren Bindungsenergien .
Allerdings bezieht sich die Messung von Shih auf das VBM (siehe linke
Ecke in Abbildung 7.9), während wir die Fermi-Energie als Referenz
verwenden. Der Unterschied könnte darin begründet liegen.

Abbildung 7.9: Die Hg5d und Cd4d Kernniveaus für HgTe, Hg0.7 Cd0.3 Te
und CdTe, bezogen auf das VBM. Das Hg 5d Level verschiebt sich um 0.1 eV
zu kleineren Bindungsenergien und das Cd 4d Level um 0.25 eV zu höheren
Bindungsenergien in ternären Mischungen {[}54{]}.

Abbildung 7.10: Die Hg5d und Cd4d Kernniveaus für Hg1-x Cdx Te (Werte
für x zwischen 0,05 und 0,4) bezogen auf die Fermie-Energie. Das Hg 5d
Level verschiebt sich um 0.3 eV zu größeren Bindungsenergien, Cd 4d
Level um 0.12 eV ebenfalls zu höheren Bindungsenergien.

Spin-Bahn-Aufspaltung Aus mehreren Messungen haben wir die
Spin-Bahn-Aufspaltung der erwähnten Kernniveaus bestimmt. Sie
entsprechen den Ergebnissen von Silberman et. al.~{[}53{]}. Tellur 4d
(1,47±0,02) eV

Cadmium 4d (0,65±0,05) eV

Quecksilber 5d (1,79±0,08) eV

48

7.4

Ergebnisse der Photoemission

Das Valenzband - Messung in normaler Emission

Die Photoelektronenspektroskopie in senkrechter Richtung zur
Kristalloberfläche wird als Messung in normaler Emission oder k⊥
-Messung bezeichnet. Die Parallelkomponente des Wellenvektors ist daher
null (siehe Formel (3.8)). Im Rahmen dieser Diplomarbeit wurden vier
solcher Messungen an unterschiedlichen Proben durchgeführt.

Abbildung 7.11: Erste Messung in normaler Emission. Es wurde Probe II
(x=0,4) untersucht. Die Spaltung erfolgte bei Raumtemperatur.

Abbildung 7.12: Valenzbandspektrum von Probe V (x=0,1955) in normaler
Emission nach gekühlter Spaltung.

Unsere erste Messung erfolgte mit Probe II (x=0,4). Das Ergebnis der
ungekühlten Spaltung ist in Abbildung 7.11 zu sehen. Bei einer
Anregungsenergie von 42 und 43 eV erscheinen Intensitäten oberhalb der
Fermi-Energie. Dabei handelt es sich um Anregungen der Te4dKernniveaus
durch die zweite Beugungsordnung des Gittermonochromators. Die
kinetische Energie dieser Kernanregungen entspricht den kinetischen
Energien der Valenzelektronen, die durch die erste Beugungsordnung des
Gitters angeregt werden. Störende zusätzliche Intensitäten treten auch
noch bei niedrigeren Energien auf. Auf die Schwierigkeiten der
Oberflächenpräparation geht bereits Kapitel 6 ein. Da die Probe bei
Raumtemperatur gespalten wurde, ist mit einer uneinheitlichen Oberfläche
zu rechnen. Die Photoemissionsmessung bestätigt dies, denn es sind keine
scharfen Intensitätsmaxima zu erkennen. Vielmehr scheint es sich um eine
Überlagerung vieler Zustände zu handeln. Das Spektrum verliert damit
seine Richtungsinformation und summiert unterschiedliche
Wellenvektorwerte aufgrund der Oberflächenstruktur auf. Das Valenzband
enthält im Bereich von 0 bis 4 eV keine auswertbare Dispersion.

Das Valenzband - Messung in normaler Emission

49

Im Spektrum ist ein Band zwischen 4 und 6 eV Bindungsenergie sichtbar.
Wir können es s-artigen Zuständen der Kationen zuordnen {[}48{]},
{[}11{]}. Dieses Band zeigt keine erkennbare Dispersion, nur eine
geringe Intensitätsmodulation ist zu sehen. In theoretischen
Bandstrukturrechnungen (siehe Abbildung 7.18 und 7.19) finden sich in
diesem Energiebereich die SpinBahn-abgespaltenen Lochbänder für die X, W
und L-Punkte der Brillouinzone. Zum Γ-Punkt hin weisen sie eine sehr
starke Dispersion auf. Aufgrund der daraus folgenden geringen
Zustandsdichte sind diese nur sehr schwer aufzulösen. Da die Zustände im
Energiebereich von 4 bis 6 eV offensichtlich keine Dispersion zeigen
(oder diese noch nicht aufgelöst werden kann), haben wir uns im
Folgenden entschlossen, nur den Energiebereich von der Fermi-Energie bis
zu 4 eV Bindungsenergie zu untersuchen. Eine zweite Messung in normaler
Emission erfolgte nach nunmehr gekühlter Spaltung an Probe V (x=0,1955).
Das Ergebnis zeigt Abbildung 7.12. In diesem Spektrum ist deutlich ein
Zustand bei 2 eV Bindungsenergie zu erkennen. Aufgrund seiner fehlenden
Dispersion können wir davon ausgehen, dass es sich um einen
Oberflächenzustand handelt (siehe KapiAbbildung 7.13: Bild der
gelungenen Spaltung tel 2.3 auf Seite 6). Ein weiterer dispersionsloser
von Probe I. Die k⊥ -Messung an dieser ProbenZustand findet sich bei ca.
3,4 eV in allen Spek- oberfläche findet sich in Abbildung 7.14. tren in
normaler Emission. Bei seiner Untersuchung von HgTe stellte N. Orlowski
{[}7{]} einen solchen Zustand bei 3,8 eV fest. Da der Oberflächenzustand
die Messung überdeckt, kann das Spektrum nicht weiter ausgewertet
werden. Die beiden nachfolgenden Spaltungen und Messungen waren hingegen
erfolgreich. Zunächst wurde Probe IV (x=0,183) gekühlt gespalten. Die
Dispersion der Zustände ist in Abbildung 7.15 zu sehen. Man erkennt zum
Beispiel ein Maximum des Valenzbandes bei einer Energie von 115 eV. Auch
die Dispersion weist auf einen hochsymetrischen Punkt hin. In Verbindung
mit der folgenden Auswertung kann dieser Energie dem Γ-Punkt zugeordnet
werden. Die letzte Messung erfolgte an Probe I (x=0,07). Die gekühlte
Spaltung war erfolgreich, wie die Aufnahme mit dem
Rasterelektronenmikroskop in Abbildung 7.13 erkennen lässt. In diesem
Fall wählten wir Anregungsenergien bis zu 125 eV, obwohl die Intensität
an dieser Stelle durch die Beamline bereits stark eingeschränkt ist.
Aufgrund von Symmetriebedingungen ist es klar, dass sich der Γ-Punkt bei
115 - 120 eV befindet. Ein weiteres lokales Maximum findet sich bei 65
eV. Diese Energie entspricht dem X-Punkt. Insgesamt konnten die
Messungen den hohen Erwartungen nicht gerecht werden. Verglichen mit den
Ergebnissen von N. Orlowski {[}34{]} an HgTe {[}59{]} und C. Janowitz an
CdTe {[}63{]} fällt sofort die geringe erreichte Auflösung an der
Valenzbandkante auf. Bedingt durch die geringe Anzahl Proben und die
eingeschränkten Möglichkeiten zur Messung kann auch die Auswertung
(Abschnitt 7.6) nur begrenzt stattfinden.

50

Ergebnisse der Photoemission

Abbildung 7.14: Messung in k⊥ an Probe I (x=0.07). Deutlich ist die
Dispersion des VBM zu erkennen mit symmetrischen Punkten bei 120 und 65
eV Anregungsenergie. Durch Anpassung können diese dem Γ-Punkt bzw.
X-Punkt zugeordnet werden.

Das Valenzband - Messung in normaler Emission

51

Abbildung 7.15: Messung in k⊥ an Probe I (x=0.16). Das Valenzbandmaximum
bei 115 eV Anregungsenergie kann dem Γ-Punkt zugeordnet werden. Die
Dispersion bestätigt diese Zuordnung.

52

7.5

Ergebnisse der Photoemission

Winkelaufgelöste Photoemissionsmessungen

Die Photoemissionsanlage AR65 wurde gebaut, um winkelaufgelöste
Messungen in hoher Auflösung zu ermöglichen. Daher wurde auch mit HgCdTe
versucht, winkelabhängige Spektren der Photoelektronen zu messen. Die
ersten Ergebnisse zeigen die folgenden Abbildungen.

Abbildung 7.16: Winkelabhängige Messung des Valenzbandes und von
Kernniveaus. Es zeigt sich keine Dispersion.

Abbildung 7.17: Messung der Winkelabhängigkeit in größerer Genauigkeit
und über einen weiteren Winkelbereich.

Der Energiebereich von der Fermi-Energie bis zu 12 eV Bindungsenergie
deckt die Kernniveaus von Quecksilber und Cadmium ab. Das in Abbildung
7.16 gezeigte Ergebnis der winkelaufgelösten Messung zeigt
erwartungsgemäß keine Dispersion für die Kerniveaus. Auch das s-artige
Band von 4 bis 6,5 eV zeigt keine Abhängigkeit vom Wellenvektor. Für das
Valenzband ist eine leichte Dispersion zu erkennen. Allerdings reicht
die Auflösung nicht aus, um eine Auswertung durchzuführen. Eine weitere
Messung wurde über einen Winkelbereich von -15◦ \textless{} φ
\textless{} 30◦ durchgeführt. Dabei wurden nur Energien bis 9 eV
untersucht, um insbesondere das Verhalten der Valenzelektronen zu
untersuchen. Jedoch ist keine Dispersion erkennbar. Wahrscheinlich sind
durch eine schlechte Oberflächenqualität sämtliche
Richtungsinformationen verlorengegangen und ergeben somit eine
winkelintegrierte Intensität. Auch weitere winkelaufgelöste Messungen
waren nicht erfolgreich. Selbst nach gekühlter Spaltung konnten keine
besseren Ergebnisse als die gezeigten erzielt werden. Eine Messung an
der Ausbildungsbeamline BESSY mit dem neuen Scienta der Arbeitsgruppe
EES würde weitaus schellere winkelabhängige Ergebnisse liefern und
könnte so auch schneller über erfolglose Spaltungen Auskunft geben.

Bandstruktur und Theorie

7.6

53

Bandstruktur und Theorie

Die Bandstrukturrechnungen zu CdTe und HgTe weisen große Ähnlichkeiten
auf {[}61{]}. Zwar unterscheiden sich diese beiden Halbleiter im Betrag
ihrer fundamentalen Lücke. Mit der Photoelektronspektroskopie sind
jedoch nur besetzte Zustände einer Probe sichtbar. Diese liegen
unterhalb der Fermi-Energie und bilden das Valenzband. In den
Abbildungen 7.18 und 7.19 sind es die Zustände negativer Energie. In
diesem Bereich gibt es auch die meisten Gemeinsamkeiten im
wellenvektorabhängigen Verhalten der Zustände.

Abbildung 7.18: Bandstruktur von CdTe {[}61{]}

Abbildung 7.19: Bandstruktur von HgTe {[}61{]}

Die drei Materialien CdTe, HgTe und HgCdTe sind sich in ihrem
Tellur-Anteil von 50 \% gleich. Das Valenzband wird an der Fermi-Energie
von p-artigen Elektronen gebildet {[}57{]}. Diese Zustände liefert das
Tellur, daher ist die Ähnlichkeit nicht verwunderlich. Erst ab einem
Kompositionsverhältnis unter 0,07 für Hg1-x Cdx Te erreicht das s-artige
Leitungsbandniveau des Kations die Fermi-Energie und invertiert sein
Verhalten. Es klappt unter die Fermi-Energie. Mit der
Photoelektronspektroskopie war es möglich, diese invertierte
Bandstruktur direkt zu beobachten {[}59{]}. Insofern sollte man bei
Untersuchungen von HgCdTe mittels Photoemission keine großen
Veränderungen innerhalb des Valenzbandes erwarten, sondern das
Auftauchen eines zusätzlichen Bandes zwischen den Γ8 - und Γ7 -Bändern
(siehe Skizze 7.20). Eine Besonderheit der sogenannten
„small-gap``-Halbleiter ist das veränderte Dispersionsverhalten direkt
am Γ-Punkt. Das Kane-Modell {[}62{]} beschreibt dieses
nicht-parabolische Verhalten bei Halbleitern, deren Bandlücke kleiner
als 0,35 eV ist. Neben den schon beschriebenen Problemen, überhaupt ein
Valenzbandmaximum zu bestimmen, war auch das rechnerische Anpassen der
gemessenen Spektren an eine Verteilung

54

Ergebnisse der Photoemission

Abbildung 7.20: Schematische Darstellung der Bandstruktur von CdTe und
HgTe am Γ-Punkt.

von Zuständen keineswegs trivial. Da die Spektren nicht aus einzelnen,
separaten Peaks bestanden, konnten die automatischen Funktionen („Fit
Multi-peaks``) von Origin {[}25{]} nicht verwendet werden. Jeder
Versuch, mit dieser Automatik dennoch die Messergebnisse anzupassen,
führte zu wertlosen und nicht reproduzierbaren Ergebnissen. Daher musste
die anzupassende Funktion selbst geschrieben („Non-linear Curve Fit -
Advanced Fitting Tool``) und manuell mit Werten versehen werden. Die
automatische Anpassung der Fitparameter mit dem
Levenberg-Marquardt-Algorithmus führte wieder zu unbrauchbaren
Ergebnissen. Die Anpassung musste zum großen Teil „von Hand`` erfolgen.
Erst in der Feinabstimmung konnten Automatiken eingesetzt werden.
Allerdings war es nicht möglich, mit den drei erwarteten Bändern eine
befriedigende Anpassung der gemessenen Spektren zu erreichen.

Abbildung 7.21: Dispersion des Punktes maximaler Intensität aus der K⊥
-Messung an Probe I, eingetragen in eine Bandstrukturrechnung.

Bandstruktur und Theorie

55

Zum Vergleich mit der Bandstrukturrechnung wurden daher nur die
Ergebnisse der k⊥ Messungen von Probe I herangezogen. Man beobachtet in
Abbildung 7.14 für Photonenenergien von 120 bis 85 eV eine Dispersion zu
höheren Bindungsenergien. Im Bereich von 80 bis 65 eV sieht man hingegen
eine Dispersion zu niedrigeren Bindungsenergien. Dieses Verhalten kann
mithilfe unterschiedlicher reziproker Gittervektoren in Formel (3.8)
erklärt werden. Im \textasciitilde{} 1 = 2π (4, 4, 0) angewendet und im
zweiten Fall G \textasciitilde{} 2 = 2π (3, 3, 1). Damit ist ersten Fall
wird G a a die Energie von 115 eV dem Γ-Punkt zugeordnet und die Energie
von 65 eV dem X-Punkt. In den Einzelspektren der Messung in normaler
Emission findet sich jeweils ein Punkt maximaler Intensität, der sehr
genau bestimmt werden kann. Dieser Punkt weist außerdem eine starke
Dispersion auf. Er repräsentiert daher mit großer Wahrscheinlichkeit ein
Volumenband. Unabhängig vom der weiteren Intensitätsverteilung im
gemessenen Spektrum wurde die Energie dieses Punktes bestimmt. Mit den
gefundenen passenden reziproken Gittervektoren wurde jeder dieser Punkte
einem Punkt im reziproken Raum zugeordnet. In Abbildung 7.21 sind diese
Punkte mit einer Bandstrukturrechnung überlagert. Das Intensitätsmaxima
kann demnach nicht nur einem einzigen Band zugeordnet werden.

Kapitel 8

Zusammenfassung Im Rahmen dieser Diplomarbeit bestand die Aufgabe, die
elektronische Struktur einiger Proben (Abbildung 5.4) von Hg1-x Cdx Te
zu bestimmen. Die Zusammensetzung der Kristalle variiert zwischen x=0,07
und x=0,4. In einem ersten Schritt mussten die Proben charakterisiert
werden. Die Orientierung konnte mit der Röntgenbeugung nach Laue
(Epigramm) sehr genau bestimmt werden. Eine Transmissionsmessung mit
dieser Methode legte deutlich den polykristallinen Charakter der Proben
dar. Jedoch ist die makroskopische Orientierung der Kristalle
einheitlich. Die Untersuchungen mit energiedispersiver
Röntgenspektroskopie ergab einen noch ungeklärten Mangel an Tellur. Doch
die Daten der Zusammensetzung konnten bestätigt werden. Die nächste
Herausforderung ist die Präparation einer Oberfläche, die für
Photoemissionmessungen geeignet ist. Das Verfahren des Sputterns und
Annealens kann nicht verwendet werden. Entlang der (110)-Richtung lassen
sich die Kristalle hingegen sehr gut spalten. Diese Spaltrichtung ist
allgemein von Kristallen in Zinkblendestruktur bevorzugt. Noch bessere
Ergebnisse konnten erzielt werden, wenn die Proben vor dem Spalten
gekühlt wurden. Die Qualität einer durchgeführten Spaltung wurde mit der
Beugung langsamer Elektronen sowie mit einem Rasterelektronenmikroskop
überprüft und bestätigt. Weiterhin zeigte die Untersuchung der
Probenoberflächen mit einem Atomkraftmikroskop, dass sich Fehlstellen in
HgCdTe schon bei Raumtemperatur aus dem Volumen an die Oberfläche
bewegen können. Auch die eigentliche Photoemissionsmessung war
erfolgreich. Bei der Untersuchung der energetischen Lage der Kernniveaus
wurden Einflüsse der chemischen Umgebung sichtbar. So konnte zum
Beispiel eine Verdopplung der Spin-Bahn-aufgespaltenen Te4d-Niveaus
gemessen werden (Abbildung 7.7). Des weiteren ließ sich eine
Verschiebung der Kernniveaus nahe dem Valenzband in Abhängigkeit von der
Zusammensetzung messen. Die Messungen an gut präparierten
(110)-Oberflächen in normaler Emission zeigten eine gut sichtbare
Dispersion, die sich eindeutig symmetrischen Punkten der Brillouin-Zone
zuordnen lassen. In einigen Punkten stimmte der Verlauf dieser
Dispersion mit den theoretischen Rechnungen überein. Die Kernniveaus
zeigten erwartungsgemäß keinerlei Dispersion. Ihre energetische Lage ist
winkelunabhängig.

56

Anhang A

A.1

Messdaten EDX April 2007

Die ersten Messungen der Komposition mittels EDX erfolgten im April 2007
an fünf ausgewählten Proben. Die Messergebnisse wurden auf zwei Atome in
der Einheitszelle normiert. Damit lässt sich der Anteil Cd direkt dem
x-Wert der Summenformel Hg1-x Cdx Te zuordnen. Der angegebene Sollwert
stammt von den Unterlagen aus Moskau, die den Proben mitgegeben wurden.
Dort wurden die Proben bereits vor 2004 charakterisiert und mithilfe
einer anderen Methode ihre Zusammensetzung sehr genau bestimmt. Nr. 1

2

3

4

5

Cd 0.07 0.06 0.05 0.13 0.11 0.15 0.39 0.40 0.38 0.20 0.18 0.17 0.16 0.16
0.16

rel.Fehler 29.6\% 40.6\% 45.5\% 17.9\% 20.5\% 16.1\% 10.4\% 9.9\% 10.1\%
15.2\% 16.3\% 19.0\% 19.7\% 19.0\% 18.4\%

Hg 1.00 1.02 1.05 0.99 0.99 0.93 0.67 0.65 0.69 0.89 0.90 0.90 0.90 0.89
0.91

rel. Fehler 18.9\% 19.6\% 19.1\% 19.3\% 19.3\% 18.5\% 21.9\% 22.2\%
21.2\% 20.3\% 20.3\% 21.7\% 20.8\% 23.1\% 22.8\%

Te 0.93 0.92 0.90 0.89 0.90 0.92 0.94 0.95 0.93 0.91 0.92 0.93 0.93 0.95
0.93

rel.Fehler 8.2\% 8.5\% 8.8\% 8.8\% 8.7\% 7.8\% 8.4\% 7.8\% 7.9\% 8.6\%
8.6\% 8.8\% 8.8\% 8.8\% 8.9\%

Probe I

x (soll) 0.07

V

0.1955

II

0.4

V

0.1955

V

0.1955

Tabelle A.1: Zusammensetzung ausgewählter fünf Proben, Ergebnisse von
EDX. Es wurden je drei unterschiedliche Punkte der Proben untersucht. An
den Proben wurde zuvor mittels Photoemission gemessen.

Weitere Messungen in Photoemission wurden im Juli und August 2007
vorgenommen. Bei der Präparation ist eine Probe bereits vor der Messung
gespalten. Sie ist auf der folgenden Seite mit Probe 10 bezeichnet.
Unter der Tabelle finden sich zwei SEM-Aufnahmen dieser Probe.

57

58

A.2

Anhang

Messdaten EDX August 2007

Nach Beendigung der Messungen bei BESSY im August 2007 wurden auch diese
Proben zur Untersuchung mittels EDX gegeben: Nr. 1

2

3

4

5

6

7

8

9

10

Cd 0.16 0.09 0.11 0.39 0.38 0.40 0.07 0.04 0.05 0.11 0.11 0.10 0.12 0.11
0.08 0.07 0.07 0.09 0.17 0.16 0.17 0.17 0.15 0.17 0.06 0.05 0.06 0.06
0.06 0.07

rel.Fehler 20.7\% 20.5\% 21.2\% 8.9\% 9.6\% 9.1\% 40.9\% 54.4\% 46.0\%
23.5\% 18.8\% 25.5\% 22.1\% 22.4\% 32.5\% 38.2\% 39.5\% 27.0\% 17.9\%
19.2\% 18.8\% 17.3\% 16.7\% 16.4\% 31.4\% 46.2\% 26.3\% 35.8\% 37.4\%
39.2\%

Hg 0.95 1.00 0.96 0.64 0.65 0.66 0.98 1.08 1.07 1.00 0.97 0.99 0.96 0.94
1.03 1.05 1.03 1.00 0.93 0.91 0.90 0.90 0.94 0.91 1.02 1.05 1.04 1.05
1.02 1.00

rel. Fehler 21.9\% 17.4\% 20.8\% 19.1\% 19.9\% 20.0\% 22.0\% 20.3\%
19.0\% 20.2\% 18.5\% 20.1\% 21.0\% 21.3\% 20.6\% 19.4\% 20.0\% 22.1\%
20.5\% 21.8\% 22.5\% 19.9\% 18.4\% 17.6\% 18.3\% 19.9\% 16.8\% 19.8\%
19.6\% 20.9\%

Te 0.89 0.91 0.92 0.97 0.96 0.94 0.95 0.88 0.88 0.89 0.92 0.91 0.92 0.94
0.89 0.88 0.90 0.91 0.90 0.93 0.93 0.93 0.91 0.92 0.92 0.90 0.90 0.89
0.92 0.93

rel.Fehler 9.0\% 7.7\% 9.0\% 7.1\% 7.5\% 7.2\% 9.0\% 8.9\% 8.7\% 8.9\%
7.8\% 9.0\% 9.0\% 8.4\% 9.0\% 8.7\% 8.8\% 9.0\% 8.8\% 8.9\% 8.8\% 8.4\%
8.1\% 7.6\% 7.6\% 8.7\% 7.2\% 8.7\% 8.5\% 8.9\%

Probe VI

x (soll) 0.105

II

0.4

I

0.07

VI

0.105

VI

0.105

I

0.07

V

0.1955

IV

0.183

I

0.07

I

0.07

Tabelle A.2: EDX-Ergebnisse an den Proben, die im Juli und August 2007
per ARPES untersucht wurden.

Literaturverzeichnis {[}1{]} W. D. Lawson, S. Nielson, E. H. Putley, and
A. S. Young. Preparation and properties of HgTe and mixed crystals of
HgTe-CdTe. J. Phys. Chem. Solids 9, 325-329 (1959). {[}2{]} D. Long and
J. L. Schmit. Mercury-cadmium telluride and closely related alloys.
Semiconductors and Semimetals, Vol. 5, pp.~175-255, edited by R. K.
Willardson and A. C. Beer, Academic Press, New York (1970). {[}3{]} Gert
Finger, J. Garnett, N. Bezawada, R. Dorn, L. Mehrgan, M. Meyer, A.
Moorwood, J. Stegmeier, G. Woodhouse. Performance evaluation and
calibration issues of large format infrared hybrid active pixel sensors
used for ground- and space-based astronomy. Nuclear Instruments and
Methods in Physics Research A 565 (2006) 241-250. {[}4{]} Derek Ives,
Nagajara Bezawada. Large area near infra-red detectors for astronomy.
Nuclear Instruments and Methods in Physics Research A 573 (2007)
107-110. {[}5{]} S. Gillessen, et. al.~The Messenger 120 (2005) 26-32.
{[}6{]} S. H. Groves, R. N. Brown. C. R. Pidgeon. Interband
Magnetoreflection and Band Structure of HgTe. Phys. Rev.~161 (1967),
779. {[}7{]} N. Orlowski. Untersuchung der elektronischen Struktur von
HgSe und HgTe mittels winkelaufgelöster Photoemission. Diplomarbeit, AG
EES, (2000). {[}8{]} W. M. Higgins, G. N. Pultz, R. G. Roy, R. A.
Lancaster, J. L. Schmit. Standard relationships in the properties of
Hg1-x Cdx Te. Vakuum Science and Technology A 7 (1989), p 271-275.
{[}9{]} E. Preuss, B. Krahn-Urban, R. Butz. Laue Atlas. Plotted Laue
Back-Reflection Patterns of the Elements, the Compounds RX and RX2 .
Bertelsmann Universitätsverlag, Düsseldorf, 1974. {[}10{]} A. Tanaka, Y.
Masa, S. Seto, T. Kawasaki. Zinc and selenium co-doped CdTe substrates
lattice matched to HgCdTe. J. Cryst. Growth vol.~94 (1989) p.~166-70.
{[}11{]} K.-U. Gawlik. Untersuchung der elektronischen Struktur von
II-VI-Verbindungshalbleitern mit direkter und inverser Photoemission.
Dissertation, (1996). {[}12{]} W. K. Ford, T. Guo, D. L. Lessor, C. B.
Duke. Dynamical low-energy electron-diffraction analysis of bismuth and
antimony epitaxy on GaAs(110). Phys. Rev.~B 42, 14 (1990), 8952.
{[}13{]} C. B. Duke. Structure and bonding of tetrahedrally coordinated
compound semiconductor cleavage faces. J. Vac. Sci. Technol. A 10(4),
2032 (1992). {[}14{]} H. Hertz. Über den Einfluss des ultravioletten
Lichtes auf die elektrische Entladung. Ann. Phys. 31, 983 (1887).

59

60

LITERATURVERZEICHNIS

{[}15{]} W. Hallwachs. Über den Einfluss des Lichtes auf elektrostatisch
geladene Körper. Ann. Phys. 33, 303 (1888). {[}16{]} A. Einstein. Über
einen die Erzeugung und Verwandlung des Lichtes betreffenden
heuristischen Gesichtspunkt. Annalen der Physik 17 (1905), 132-148.
{[}17{]} W. E. Spicer. Photoemissive, Photoconductive, and Optical
Absorption Studies of AlkaliAntimony Compounds Phys. Rev.~112, 117
(1958). {[}18{]} Stefan Hüfner. Photoelectron Spectroscopy, Principles
and Applications, Third Edition. Springer-Verlag Berlin Heidelberg New
York 2003. {[}19{]} R. Heimburger. Elektronische Eigenschaften und
Phasentransformation von β-MoTe2 . Diplomarbeit, AG EES, (2007).
{[}20{]} M. P. Seah, W. A. Dench. Quantitative Electron Spectroscopy of
Surface: A Standard Data Base for Electron Inelastic Mean Free Paths in
Solids. Surface and Interface Analysis, Vol. 1, Issue 1, p 2-11 (1979).
{[}21{]} D. A. Shirley. High-Resolution X-Ray Photoemission Spectrum of
the Valence Bands of Gold. Phys. Rev.~B 5, 4709-4714 (1972). {[}22{]} S.
Tougaard. Deconvolution of loss features from electron spectra. Surface
Science 139 (1984) pp.~208-218. {[}23{]} M. Kauert. Vanadiumoxide.
Herstellung, Charakterisierung und elektronische Struktur. Diplomarbeit,
AG EES, (2007). {[}24{]} A. Savitzky, Marcel J.E. Golay. Smoothing and
Differentiation of Data by Simplified Least Squares Procedures.
Analytical Chemistry, 36: 1627-1639 (1964). doi:10.1021/ac60214a047
{[}25{]} Softwarepaket: Origin 7.5 SR5, http://www.OriginLab.com (2004)
{[}26{]} T. Plake. Aufbau und Inbetriebnahme des
Photoemissionsexperimentes HIRE-PES: Charakterisierung und erste
Untersuchungen an Bi2 Sr2 CuO6 -Hochtemperatursupraleitern.
Diplomarbeit, AG EES, (1998). {[}27{]} C. Janowitz, R. Müller, T. Plake,
Th. Böger, R. Manzke. New high-resolution photoemission station for
synchrotron radiation at BESSY. Journal of Electron Spectroscopy and
Related Phenomena 105, 43-49 (1999). {[}28{]} E. Purcell. Phys. Rev.~54
(1938) 818. {[}29{]} G. Mante. Doktorarbeit, Universität Kiel (1992).
{[}30{]} Instruction Manual VUV Discharge Lamp HIS 13. Omicron
Vakuumphysik GmbH (www.omicron.de). {[}31{]}
Platin-Widerstandsthermometer Pt100 nach IEC 751 / DIN EN 60751.
{[}32{]} AR65view. Java-Software zur Analyse und Manipulation der
Messdaten von AR65 und WESPHOA. sourceforge.net/projects/ar65view/.
{[}33{]} M. Martins, G. Kaindl, N. Schwentner. Design of the
high-resolution BUS XUV-beamline for BESSY II. Journal of Electron
Spectroscopy and Related Phenomena 101-103, 965-969 (1999).

LITERATURVERZEICHNIS

61

{[}34{]} N. Orlowski, J. Augustin, Z. Golacki, C. Janowitz, R. Manzke.
Direct evidence for the inverted band structure of HgTe. Phys. Rev.~B,
Rapid Communications, 61, R5058-R5061, (2000). {[}35{]} J. C. Brice.
Properties of Mercury Cadmium Telluride. EMIS Datareviews Series No 3.
INSPEC, IEE, p.~3 (1994). {[}36{]} G. L. Hansen, J. L. Schmit, and T. N.
Casselman. Energy gap versus alloy composition and temperature in Hg1-X
CdX Te. J. Appl. Phys. 53, 7099-7101 (1982). {[}37{]} P. Norton. HgCdTe
infrared detectors. Opto-Electronics Review 10(3), 159-174 (2002).
{[}38{]} A. W. Vere, B. W. Straughan, D. J. Williams, N. Shaw, A. Royle,
J. S. Gough, J. B. Mullin. Growth of CdX Hg1-X Te by a pressurised
cast-recrystallise-anneal technique J. Cryst. Growth vol.~59 (1982)
p.~121-129. {[}39{]} L. Colombo, A. J. Syllaios, r. W. Perlaky, M. J.
Brau. Growth of large diameter (Hg,Cd)Te crystals by incremental
quenching. J. Vac. Sci. Technol. A 3(1), 100-104 (1985). {[}40{]} Yue
Wang, Quanbao Li, Qinglin Han, Qinghua Ma, Bingwen Song, Wanqi Jie,
Yaohe Zhou, Yuko Inatomi. A two-stage technique for single crystal
growth of HgCdTe using a pressurized Bridgman method. Journal of Crystal
Growth 263 (2004) 273-282. {[}41{]} L. Colombo, R. R. Chang, C. J.
Chang, B. A. Baird. Growth of Hg-based alloys by the traveling heater
method. J. Vac. Sci. Technol. A 6(4), 2795-2799 (1988). {[}42{]} T. C.
Harman. Optically pumped LPE-grown Hg1-X CdX Te lasers. J. Electron.
Mater. vol.~8 (1979) p.~191-200. {[}43{]} B. H. Koo, Y. Ishikawa, J.F.
Wang, M. Isshiki. Crowth of Hg1-x (Cd1-y Zny )x Te epilayers on (100)
Cd1-y Zny Te/GaAs substrates by ISOVPE. Materials Science and
Engineering, B66, 7074, (1999). {[}44{]} Yue Wang, Quanbao Li, Qinglin
Han, Qinghua Ma, Rongbin Ji, Bingwen Song, Wanqi Jie, Yaohe Zhou, Yuko
Inatomi. Growth and properties of 40mm diameter Hg1-X CdX Te using the
two-stage Pressurized Bridgman method. Journal of Crystal Growth 263
(2004) 54-62. {[}45{]} Vladimir Nikiforov (Nikiforov Vladimir
Nikolaevix), MSU, Private communication. {[}46{]} P. Capper. Properties
of narrow gap Cadmium-based compounds. INSPEC, the Institution of
Electrical Engineers (1994). {[}47{]} D. J. Williams, A. W. Vere.
Sub-grain boundaries in CdX Hg1-X Te and CdTe. J. Cryst. Growth vol.~83
(1987) p.~341-352. {[}48{]} W. E. Spicer, J. A. Silberman, and I.
Lindau. Band gap variation and lattice, surface, and interface
``instabilities`` in Hg1-X CdX Te and related compounds. J. Vac. Sci.
Technol. A 1(3) pp.~1735-1743 (1983). {[}49{]} P. Morgen, J. Silberman,
I. Landau, W. E. Spicer, J. A. Wilson. Stability of an atomically clean
Hg1-X CdX Te surface in vacuum and under O2 exposure. Journal of Crystal
Growth 56, 493-497 (1982). {[}50{]} R. Mitdank. Mikrosondenanalyse.
Beispiele für Untersuchungen an der Elektronenstrahlmikrosonde.
http://htc.physik.hu-berlin.de/∼mitdank/semq.htm.

62

LITERATURVERZEICHNIS

{[}51{]} M. A. Berding, A. Sher, A-B Chen and R. Patrick. Vacancies and
surface segregation in HgCdTe and HgZnTe. Semicond. Sci. Technol. vol.~5
pp.~S86-S89 (1990). {[}52{]} C. Ribbat. Aufbau einer MBE-Anlage zur
Herstellung von MolybdändichalkogenidSchichtkristallen für
photovoltaische Anwendungen. Diplomarbeit, AG EES, (1998). {[}53{]} J.
A. Silberman, P. Morgen, I. Lindau, and W. E. Spicer. UPS study of the
electronic structure of Hg1-X CdX Te: Breakdown of the virtual crystal
approximation. J. Vac. Sci. Technol., 21(1) pp.~142-145 (1982). {[}54{]}
C. K. Shih and W. E. Spicer. Photoemission studies of core level shifts
in HgCdTe, CdMnTe, and HgZnTe. J. Vac. Sci. Technol. A, 5(5)
pp.~3031-3034 (1987). {[}55{]} C. K. Shih, J. A. Silberman, A. K. Wahi,
G. P. Carey, I. Lindau, and W. E. Spicer. Angle resolved photoemission
study of the alloy scattering effect in Hg1-X CdX Te. J. Vac. Sci.
Technol. A, 5(5) pp.~3026-3030 (1987). {[}56{]} Dr.~Ralph Püttner, FU
Berlin. Beamlinebetreuer BUS. Private communication. {[}57{]} Chang-Youn
Moon, Su-Huai Wei. Band gap of Hg chalcogenides:
Symmetry-reductioninduced band-gap opening of materials with inverted
band structures. Phys. Rev.~B 74:045205 (2006). {[}58{]} J. F. Moulder,
W. F. Stickle, P. E. Sobol, K. D. Bomben. Handbook of X-ray
Photoelectron Spectroscopy. Perkin-Elmer Corporation, 1992. {[}59{]} C.
Janowitz, N. Orlowski, R. Manzke, Z. Golacki. On the band structure of
HgTe and HgSe - view from photoemission. Journal of Alloys and
Compounds, 328, 84-89, (2001). {[}60{]} A. Fleszar, W. Hanke. Electronic
structure of IIB -VI semiconductors in the GW approximation. Phys.
Rev.~B 71:045207 (2005). {[}61{]} A-B Chen, Y-M Lai-Hsu, S.
Krishnamurthy, M. A. Berding, and A. Sher. Band structures of HgCdTe and
HgZnTe alloys and superlattices. Semicond. Sci. Technol. vol.~5
pp.~S100-S102 (1990). doi:10.1088/0268-1242/5/3S/021 {[}62{]} E. O.
Kane. Band structure of indium antimonide. J. Phys. Chem. Solids 1 (4),
249-261 (1957) {[}63{]} C. Janowitz, L. Kipp, R. Manzke. Experimental
surface band structure of CdTe(110). Surface Science 231 (1990) 25-31.

Danksagung Die Mitarbeit in einem wissenschaftlichen Team stellt den
krönenden Abschluss meines Studiums dar. Daher möchte ich mich bei der
Arbeitsgruppe EES für die freundliche Aufnahme und gute Zusammenarbeit
bedanken. Neben ARPES konnte ich so viele andere Geheimnisse näher
kennen lernen, die sich hinter Abkürzungen wie STM, XUV, LEED, EDX, SEM,
AFM, EPR und BESSY verbergen. Mein Dank gilt insbesondere: •
Prof.~Manzke für die Bereitstellung des interessanten Themas und die
Möglichkeit, in seiner Arbeitsgruppe ein so umfassendes Forschungsthema
eigenständig bearbeiten zu können. • Dr.~Krapf für die Unterstützung,
den dreimonatigen Forschungsaufenthalt in Moskau zu organisieren. • Hr.
Sölle für die Präparation und Zerteilung der Proben. • Der Werkstatt
(insbesondere Hr. Rausche, Hr. Fahnauer und Frau Rosinska) für die gute
Zusammenarbeit, die Hilfe beim Herstellen zahlloser Kleinteile und die
Unterstützung bei der Reparatur einiger Gerätschaften. • Telegina Inna
Vasiljevna für die umfassende Laue-Untersuchung meiner Kristalle. •
Nikiforov Vladimir Nikolaeviq für die Betreuung während des Moskauer
Forschungsaufenthaltes. • den Korrekturlesern Gabi, Rahela, Susan und
Horst für die große Mühe, all die kleinen Fehler zu entdecken, die sich
eingeschlichen hatten. Danke auch für die guten Vorschläge zu besseren
Formulierungen. • auch meiner Familie, die mit Geduld mein Studium
begleitet hat und ohne deren Unterstützung das alles hier nicht möglich
gewesen wäre.

Erklärung Hiermit versichere ich, die vorliegende Arbeit ohne unerlaubte
fremde Hilfe angefertigt zu haben. Es sind keine anderen als die
angegebenen Quellen und Hilfsmittel benutzt worden. Ich gestatte
Einsichtnahme in diese Diplomarbeit in der Fachbereichsbibliothek.

Berlin, Januar 2008

Matthias Kreier


